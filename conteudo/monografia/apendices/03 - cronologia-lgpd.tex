\chapter{Cronologia da LGPD e seu impacto nas igrejas}
\label{apendice:cronologia-lgpd}

Este apêndice apresenta uma linha do tempo com os principais marcos legais relacionados à proteção de dados no Brasil, desde os primeiros debates públicos até a entrada em vigor da \gls{lgpd}.  Também estão incluídos eventos que marcam o início da resposta institucional das igrejas ao novo cenário jurídico. A cronologia permite visualizar a defasagem entre o avanço legislativo e a adoção efetiva de medidas de conformidade no meio eclesiástico que até o momento da construção deste trabalho acadêmico, conforme \hyperref[tabela:panorama_adequacao_ipb]{Tabela~\ref*{tabela:panorama_adequacao_ipb}} vem avançando muito lentamente. Os dados deste apêndice provêm de elaboração própria com base em dados do \citeonline{serpro_linhadotempo_2022}, legislação vigente, trabalhos acadêmicos e reportagens de domínio público que constam nas notas de rodapé.

\rowcolors{2}{gray!10}{white}
\begin{longtable}{>{\raggedright\arraybackslash}p{2cm} >{\raggedright\arraybackslash}p{13cm}}
\caption{Linha do tempo da LGPD e sua relação com o contexto eclesiástico}\\

\toprule
\textbf{Ano} & \textbf{Evento} \\
\midrule
\endfirsthead

\toprule
\textbf{Ano} & \textbf{Evento} \\
\midrule
\endhead
\bottomrule
\endfoot

2010 & Consulta pública do Ministério da Justiça sobre anteprojeto de lei de proteção de dados pessoais. \\
2011 &
\parbox[t]{13cm}{
    \textbullet\ Sancionada a \gls{lai}\footnote{dados pessoais de acesso público}.\\
    \textbullet\ Proposto projeto de lei nº 2126 (Marco Civil da Internet) – direitos e deveres de usuários e provedores.
} \\
2012 &
\parbox[t]{13cm}{
    \textbullet\ Sancionada a Lei Carolina Dieckmann\footnote{tipificação de crimes cibernéticos, como compartilhar dados pessoais sem autorização} \\
    \textbullet\ Proposto, na Câmara, o \gls{pl} nº 4.060, sobre o tratamento de dados pessoais
} \\
2013 & Proposto, o \gls{pls} nº 330, sobre a proteção, o tratamento e o uso de dados pessoais. \\
2014 & Entra em vigor o Marco Civil da Internet \\
2015 & Aprovado na \gls{cct}, do Senado, o substitutivo do \gls{pls} nº 330/13 \\
2016 & 
\parbox[t]{13cm}{
    \textbullet\ Aprovação da \gls{gdpr}, na Europa \\
    \textbullet\ Nova consulta pública, pelo \gls{mj}, que resulta no PL nº 5.276/16, anexado ao PL nº 4.060/2012
} \\
2017 & Tramitação no Congresso de dois projetos: o PL nº 5.276/2016, e o \gls{pls} nº 330/2013, no Senado \\
2018 & 
\parbox[t]{13cm}{
    \textbullet\ Em março: escândalo ``Facebook-Cambridge Analytica''\footnote{de uso ilícito de dados de usuárias de rede social pela empresa de consultoria \cite{carvalho_fb_cabridge_2023}} \\
    \textbullet\ Em maio: entra em vigor, o \gls{gdpr}, na Europa \\
    \textbullet\ Em agosto: sancionada a \gls{lgpd}, após unificação dos textos da Câmara e do Senado no \gls{plc} nº 53
} \\
2019 & Criada a \gls{anpd} pela \gls{mp} nº 869. \\
2020 &
\parbox[t]{13cm}{
    \textbullet\ Aprovada a criação da \gls{anpd}, pela \gls{mp} nº 869 \\
    \textbullet\ Em discussão a \gls{pec} nº 17, que inclui a proteção de dados pessoais, inclusive digitais, entre os direitos fundamentais do cidadão
} \\
2021 & Aplicação efetiva das sanções pela \gls{anpd}. Publicação dos primeiros guias orientativos. \\
2022 & Algumas igrejas iniciam adequações básicas; tema da proteção de dados aparece em eventos teológicos\footnote{Oficina online promovida pela \gls{cbesp} em março de 2022 para apresentação da \gls{lgpd} a pastores e líderes \cite{cbesp_workshop_lgpd_2022}.} \\
2024 & Vazamentos que impactam igrejas ganham repercussão\footnote{Vazamento de dado que expôs dados de quase 1 milhão de pessoas \cite{almeida_inchurch_2024}}. Pressão por responsabilização e conformidade aumenta. \\
\bottomrule
\end{longtable}

\chapter{Adequação à \gls{lgpd} em sites de igrejas da \gls{ipb}}
\label{apendice:adequacao_sites}

Este apêndice apresenta um panorama da adequação dos sites de igrejas da IPB à LGPD\footnote{A coleta foi realizada manualmente, com apoio de ferramentas automatizadas de varredura e inspeção de conteúdo. O processamento dos dados foi concluído em 20 de junho de 2025. O código utilizado no processo de análise está disponível publicamente em \url{https://github.com/vitormattos/webscraping-anuario-igrejas-ipb}, com o objetivo de garantir transparência e reprodutibilidade da metodologia aplicada.}.

\begin{table}[H]
\centering
\begin{threeparttable}
\caption{Panorama da adequação à \gls{lgpd} em sites de igrejas da \gls{ipb}}

\label{tabela:panorama_adequacao_ipb}
\rowcolors{2}{gray!10}{white}

\begin{tabularx}{\textwidth}{>{\raggedright\arraybackslash}X c}
\toprule
\textbf{Indicador} & \textbf{Quantidade} \\
\midrule
Total de igrejas \gls{ipb} analisadas & 2935 \\
Igrejas com site informado no cadastro\tnote{a} & 640 \\
Endereços que não são redes sociais\tnote{b} & 532 \\
Endereços válidos e ativos\tnote{c} & 219 \\
Sites com qualquer menção à \gls{lgpd} & 22 \\
Com aviso de uso de cookies & 16 \\
Com Política de Privacidade publicada & 15 \\
Com e-mail genérico indicando canal de DPO\tnote{d} & 12 \\
Com formulário específico para solicitação de direitos do titular\tnote{e} & 5 \\
Com \gls{dpo} identificado\tnote{f} & 2 \\
\bottomrule
\end{tabularx}
\begin{tablenotes}
\footnotesize
\item[a] Embora 640 igrejas tenham informado endereços de páginas web, muitos desses estão completamente inválidos, com textos que não dizem respeito a uma página web.
\item[b] Sites com endereços de páginas web válidas do ponto de vista técnico, mas que não necessariamente levam a sites ativos.
\item[c] Sites que exibem algum conteúdo, mas não necessariamente estão em conformidade com a LGPD. Embora apresentem páginas da igreja, muitas delas contêm falhas de segurança, mensagens de erro ou até mesmo indícios de comprometimento parcial por terceiros.
\item[d] Por exemplo: adm@paginadaigreja.org.br, contato@paginadaigreja.org.br e não endereços de endereço eletrônico de pessoas.
\item[e] Refere-se à presença de um formulário eletrônico próprio que permite ao titular do dado pessoal exercer seus direitos previstos na \gls{lgpd}, como acesso, correção, exclusão ou portabilidade de dados.
\item[f] O \gls{dpo}, ou Encarregado de Dados como definido na \gls{lgpd}, foi identificado nominalmente, com endereço eletrônico para contato.
\end{tablenotes}
\end{threeparttable}
\end{table}
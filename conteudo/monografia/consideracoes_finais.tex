\newpage
\newcommand{\tituloConsideracoesFinais}{Considerações finais}

\chapter*{\tituloConsideracoesFinais}
\addcontentsline{toc}{section}{\MakeUppercase{\tituloConsideracoesFinais}}

A presente reflexão propôs caminhos teológicos e práticos para uma governança digital ética no contexto eclesiástico. O ponto de chegada desta jornada é um convite à ação. O conhecimento produzido ao longo deste trabalho só se justifica se for convertido em prática, com a igreja assumindo sua responsabilidade diante de Deus e da sociedade na forma como lida com informações sensíveis e com as pessoas por trás delas.

Ao longo dos três capítulos, buscou-se integrar fundamentos bíblico-teológicos com os desafios contemporâneos da era digital. No primeiro capítulo, apresentaram-se os fundamentos da comunhão cristã à luz da teologia reformada, com ênfase na relação entre confiança, cuidado e responsabilidade mútua. O segundo capítulo examinou os riscos éticos e espirituais provocados pela cultura da vigilância, pelo uso inadequado das tecnologias e pela negligência no tratamento de dados, com base em documentos jurídicos e reflexões teológicas. Já o terceiro capítulo apresentou diretrizes práticas para uma governança digital responsável nas igrejas, incluindo critérios de adoção tecnológica, indicadores de maturidade institucional e ações pastorais concretas.

Apesar do esforço de síntese teológica e normativa, reconhece-se que esta monografia possui limitações. A principal delas é a ausência de dados empíricos sobre a realidade das igrejas brasileiras em relação à gestão de informações pessoais, o que impede uma análise comparativa com maior fundamentação. Também não foram abordadas com profundidade questões emergentes como o uso de inteligência artificial em contextos pastorais e as implicações do compliance digital no âmbito eclesiástico.

Essas limitações, no entanto, apontam para possibilidades de investigações futuras. Sugere-se a realização de estudos empíricos com diferentes tradições eclesiásticas, tanto no Brasil quanto internacionalmente, bem como a análise da atuação de encarregados de dados (\gls{dpo}) nas igrejas à luz das resoluções da \gls{anpd}. Espera-se que, com esta reflexão, sejam abertas, permitindo que novas pesquisas explorem como princípios teológicos informam ou colidem com práticas digitais baseadas em algoritmos, vigilância automatizada e armazenamento em nuvem.

A resposta pastoral a esse desafio exige mais do que adequação técnica: exige um caminhar consciente. A igreja precisa entender que sua missão inclui também o testemunho ético nos detalhes mais cotidianos. Tratar informações com zelo, respeitar a privacidade e agir com transparência são atitudes que refletem uma espiritualidade cristã biblicamente orientada. Quando uma comunidade de fé organiza seus processos com responsabilidade, ela comunica ao mundo que leva a sério a dignidade humana, a justiça e a verdade.

Na teologia há a expressão \textit{coram Deo} que é viver integralmente diante de Deus, com integridade que não se fragmenta entre o púlpito e a administração, entre a liturgia e os servidores de dados. Não há área da vida cristã que esteja fora do olhar divino. Essa consciência reformada nos chama a agir com reverência inclusive nas escolhas tecnológicas e nas rotinas administrativas da igreja. E isso exige comprometimento coletivo. Não basta uma liderança bem-intencionada, uma boa exegese, uma boa hermenêutica, uma boa aplicação pastoral do texto bíblico ou uma boa oratória. É preciso envolver toda a comunidade na construção de uma cultura de responsabilidade digital. Isso passa por revisar práticas, instituir políticas internas, escolher ferramentas adequadas, capacitar pessoas, dialogar sobre limites, cultura cristã e, principalmente, cultivar uma visão pastoral que entenda que proteger dados é proteger relacionamentos, é preservar confiança, é zelar por vidas.

A igreja que assume esse compromisso torna-se um sinal do Reino e luz em meio à escuridão de uma sociedade marcada pelo descaso; sendo assim, sal na terra e luz no mundo. Ela demonstra que é possível unir fé e responsabilidade com dados, espiritualidade e adequação às leis, comunhão e cuidado com o próximo. E ao fazer isso, responde de forma inspiradora a um mundo sedento por coerência e humanidade.

Não se trata de burocratizar a igreja ou de abordar temas \textit{`não teológicos''}, mas de amadurecê-la, de viver \textit{coram Deo}, de formar uma comunidade que entende que boas práticas também são expressão do amor. Que a excelência no cuidado com o próximo inclui o cuidado com suas informações, seus registros, sua história. E que a obediência ao Senhor se manifesta também na forma como tratamos aquilo que nos foi confiado, inclusive dados sensíveis.

Concluímos com a esperança de que esta monografia sirva como instrumento de despertamento. Que pastores, presbíteros, diáconos, ministérios e membros encontrem aqui não apenas alertas, mas diretrizes a serem discernidas e aplicadas com sabedoria. E que essas diretrizes não sejam apenas implementadas, mas vividas. Para que a igreja, enquanto corpo de Cristo, siga sendo lugar de refúgio, responsabilidade, comunhão e testemunho. Sempre diante de Deus. Sempre \textit{coram Deo}.
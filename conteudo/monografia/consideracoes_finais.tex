\newpage
\newcommand{\tituloConsideracoesFinais}{Considerações finais}

\chapter*{\tituloConsideracoesFinais}
\addcontentsline{toc}{section}{\MakeUppercase{\tituloConsideracoesFinais}}

A presente reflexão propôs caminhos teológicos e práticos para uma governança digital ética no contexto eclesiástico. O ponto de chegada desta jornada é um convite à ação. O conhecimento produzido ao longo deste trabalho só se justifica se for convertido em prática, com a igreja assumindo sua responsabilidade diante de Deus e da sociedade na forma como lida com dados sensíveis e com as pessoas por trás deles.

A resposta pastoral a esse desafio exige mais do que adequação técnica. Exige um caminhar consciente. A igreja precisa entender que sua missão inclui também o testemunho ético nos detalhes mais cotidianos. Tratar dados com zelo, respeitar a privacidade e agir com transparência são atitudes que expressam a espiritualidade cristã em uma forma biblicamente orientada. Quando uma comunidade de fé organiza seus processos com responsabilidade e cuidado, ela comunica ao mundo que leva a sério a dignidade humana, a justiça e a verdade.

Na teologia há a expressão \textit{``coram Deo''} que é viver integralmente diante de Deus, com integridade que não se fragmenta entre o púlpito e a administração, entre a liturgia e os servidores de dados. Não há área da vida cristã que esteja fora do olhar divino. Essa consciência reformada nos chama a agir com reverência inclusive nas escolhas tecnológicas e nas rotinas administrativas da igreja. E isso exige comprometimento coletivo. Não basta uma liderança bem-intencionada, uma boa exegese, uma boa hermenêutica, uma boa aplicação pastoral do texto bíblico ou uma boa oratória. É preciso envolver toda a comunidade na construção de uma cultura de responsabilidade digital. Isso passa por revisar práticas, instituir políticas internas, escolher ferramentas adequadas, treinar pessoas, dialogar sobre limites, e principalmente, cultivar uma visão pastoral que entenda que proteger dados é proteger relacionamentos. É preservar confiança. É zelar por vidas.

A igreja que toma para si esse compromisso se torna sinal do Reino e luz na escuridão em meio a uma sociedade que cultiva uma cultura de descaso. Ela demonstra que é possível unir fé e responsabilidade, espiritualidade e competência, comunhão e organização. E ao fazer isso, responde de forma profética, bela e inspiradora a um mundo marcado pela negligência, superficialidade e desumanização tecnológica.

Não se trata de burocratizar a igreja ou de abordar temas ``não teológicos'', mas de amadurecê-la, de viver \textit{``coram Deo''}, de formar uma comunidade que entende que boas práticas também são expressão do amor. Que a excelência no cuidado com o próximo inclui o cuidado com suas informações, seus registros, sua história. E que a obediência ao Senhor também se revela em como tratamos aquilo que nos foi confiado, inclusive dados sensíveis.

Concluímos com a esperança de que esta monografia sirva como instrumento de despertamento. Que pastores, presbíteros, diáconos, ministérios e membros encontrem aqui não apenas alertas, mas diretrizes. E que essas diretrizes não sejam apenas implementadas, mas vividas. Para que a igreja, enquanto corpo de Cristo, possa seguir sendo lugar de refúgio, responsabilidade, comunhão e testemunho. Sempre diante de Deus. Sempre \textit{``coram Deo''}.
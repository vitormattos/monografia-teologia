\newcommand{\tituloIntroducao}{Introdução}

\chapter*{\tituloIntroducao}
\addcontentsline{toc}{section}{\MakeUppercase{\tituloIntroducao}}

Vivemos em uma era marcada por avanços tecnológicos acelerados, em que a digitalização atravessa todos os aspectos da vida em sociedade, inclusive a vivência comunitária e a organização das igrejas. Ferramentas digitais vêm sendo amplamente adotadas por comunidades de fé para facilitar a gestão de dados de membros, uso de \gls{ia}\footnote{a habilidade dos sistemas computacionais de emular capacidades análogas às humanas, como o raciocínio, a aprendizagem, o planejamento estratégico e a capacidade criativa.} para revisar comunicações internas, ferramentas para facilitar registros financeiros, formulários para coleta de dados, nuvem para guarda de arquivos, redes sociais e grupos para compartilhar informações, plataformas de transmissão de vídeo para cultos e outros aspectos da vida eclesiástica. Ao mesmo tempo, essa inserção da tecnologia traz consigo implicações éticas e teológicas que demandam reflexão, discernimento pastoral e compromisso com princípios cristãos fundamentais.

Porém, antes mesmo de considerarmos que há riscos legais, administrativos e tecnológicos, é preciso reconhecer que estamos lidando com uma ameaça espiritual à própria comunhão dos santos. A exposição indevida de informações, a negligência com a privacidade dos irmãos e o uso banalizado de dados sensíveis sem a mínima reflexão ou consciência de como lidar com tais informações são sinais de um distanciamento dos fundamentos bíblicos que deveriam sustentar o cuidado mútuo. Quando a igreja perde de vista que o corpo de Cristo é composto por pessoas reais, com histórias íntimas que são entregues à confiança comunitária, ela flerta com uma prática eclesiástica dessacralizada, que reduz o próximo a um dado de preciosidade irrelevante e instrumentaliza a membresia, atentando contra a imago Dei presente em cada indivíduo. Pior: afronta o próprio Deus, que zela por sua noiva com ciúmes santos. Nesse sentido, o primeiro alerta não é técnico, é teológico. E deve despertar o nosso temor.

A coleta, o armazenamento, o uso e a eventual exposição de dados sensíveis por instituições religiosas tocam diretamente na dignidade humana, no respeito à privacidade e na missão da igreja enquanto espaço de acolhimento, cuidado e verdade. Informações como nome, endereço, situação familiar, dados financeiros ou de saúde são confiadas às igrejas por pessoas que esperam ser tratadas com zelo. O mau uso dessas informações caracteriza-se não apenas como uma afronta a Deus, mas também como uma violação da segurança individual, da confiança relacional e da integridade do testemunho cristão passível de punição com multas para a igreja e todas as partes responsáveis que podem chegar até 50 milhões de reais com interdição de bens. Em tempos marcados por escândalos envolvendo vazamentos de dados, perseguições por ideologia, fragilidade institucional e desinformação, torna-se urgente repensar as práticas internas das igrejas à luz da ética bíblica. Ignorar este tema pode comprometer não apenas a credibilidade e a legalidade da atuação pastoral, mas também corromper os fundamentos basilares da comunhão cristã. O silêncio ou a omissão diante desse cenário não é prova de neutralidade ou ignorância quanto ao assunto, mas sim de cumplicidade com o descumprimento da lei.

A presente monografia propõe uma abordagem teológica e ética sobre a gestão de dados sensíveis no contexto eclesiástico, articulando princípios bíblicos com as demandas contemporâneas impostas pela era digital. Não se trata de um debate técnico ou jurídico, mas de uma reflexão pastoral que visa resgatar o valor da comunhão dos santos, marcada por responsabilidade e cuidado mútuo como resposta visível do amor ao próximo. A tecnologia, quando guiada por esse horizonte, passa a estar alinhada à missão da igreja, caminhando sob princípios bíblicos. Quando adotada de forma acrítica, porém, pode fragilizar vínculos e comprometer o propósito da igreja.

No contexto brasileiro, ainda que não haja um movimento institucionalizado ou expressivo de adoção de tecnologias com foco em privacidade, segurança e auditoria que atendam a princípios legais por parte das igrejas, é possível identificar algumas poucas igrejas que compreendem a importância da soberania tecnológica e da ética cristã no uso de ferramentas digitais onde vemos uma busca por soluções mais alinhadas a esses princípios legais, inclusive por meio do uso de tecnologias livres em suas práticas pessoais e comunitárias. Assim como são raras as igrejas que se dedicam intencionalmente a esses temas, também são escassos os recursos que ofereçam às lideranças eclesiásticas um direcionamento teológico consistente sobre proteção de dados, governança digital, soberania tecnológica e segurança da informação, alertando para os perigos da negligência desses temas. Esta lacuna motivou a formulação desta pesquisa.

A construção desta reflexão teológica fundamenta-se especialmente na tradição reformada, considerando a \gls{cfw} como base normativa. Destaca-se o capítulo XXIII, que estabelece como dever do povo orar pelos magistrados, honrá-los, pagar tributos, obedecer às suas ordens legais e sujeitar-se à sua autoridade. Esses princípios orientam a compreensão de que o cuidado com os dados não é apenas um dever social, mas expressão concreta de reverência ao Criador, ao próximo e de submissão legítima à autoridade civil quando esta ordena o que é justo.

Além da \gls{cfw}, esta monografia dialoga com autores da teologia sistemática e bíblica, como João Calvino, Herman Bavinck e Louis Berkhof, e também com pensadores contemporâneos que tratam da relação entre ética cristã, cosmovisão e tecnologia, como Francis Schaeffer, Dietrich Bonhoeffer e Timothy Keller. A perspectiva adotada parte da cosmovisão cristã reformada, segundo a qual Cristo é Senhor sobre todas as esferas da vida, inclusive a digital.

Dessa forma, a pesquisa propõe apresentar diretrizes práticas para uma governança digital ética nas igrejas, considerando os fundamentos bíblicos da dignidade humana, a responsabilidade pastoral e os riscos associados à era digital. O objetivo é contribuir para que a igreja exerça um testemunho coerente com o evangelho também em seus sistemas e processos informacionais.

Que esta introdução tenha gerado não apenas interesse acadêmico, mas inquietação pastoral e reverente temor. Que o desconforto provocado pelo reconhecimento dos riscos, da negligência e das consequências sirva como ponto de partida para um engajamento ético, teológico e prático com o tema. A boa notícia é que há caminhos possíveis, ancorados nas Escrituras, na tradição reformada e em práticas éticas que podem restaurar a confiança, proteger os vulneráveis e glorificar a Deus até mesmo na esfera digital.

Nos capítulos seguintes, serão apresentadas essas possibilidades, com o intuito de fomentar uma cultura de cuidado e reverência, tanto nos espaços físicos da igreja quanto em seus registros, ferramentas e estruturas digitais.

\newcommand{\tituloIntroducao}{Introdução}

\chapter*{\tituloIntroducao}
\markboth{\tituloIntroducao}{\tituloIntroducao}
\addcontentsline{toc}{section}{\MakeUppercase{\tituloIntroducao}}

% JUSTIFICATIVA
Vivemos em uma era marcada por avanços tecnológicos acelerados, em que a digitalização atravessa todos os aspectos da vida em sociedade, inclusive a vivência comunitária e a organização das igrejas. Ferramentas digitais vêm sendo amplamente adotadas por comunidades de fé para facilitar a gestão de dados de membros, o uso de \gls{ia}\footnote{A habilidade dos sistemas computacionais de emular capacidades análogas às humanas, como o raciocínio, a aprendizagem, o planejamento estratégico e a capacidade criativa.} na revisão de comunicações internas, registros financeiros, formulários de coleta de dados, armazenamento em nuvem, redes sociais, grupos de mensagens, plataformas de transmissão de cultos e outros aspectos da vida eclesiástica. Ao mesmo tempo, essa inserção da tecnologia traz consigo implicações éticas e teológicas que demandam reflexão, discernimento pastoral e compromisso com princípios cristãos fundamentais.

Antes mesmo de considerarmos riscos legais, de natureza administrativa ou falhas tecnológicas, é preciso reconhecer que estamos diante de uma ameaça espiritual à própria comunhão dos santos. A exposição indevida de informações, a negligência com a privacidade dos irmãos e o uso banalizado de dados sensíveis sem reflexão adequada  ou consciência de como lidar com tais informações são sinais de distanciamento dos fundamentos bíblicos que deveriam sustentar o cuidado mútuo. Quando a igreja perde de vista que o corpo de Cristo é composto por pessoas reais, com histórias íntimas entregues à confiança comunitária, ela flerta com uma prática eclesiástica dessacralizada, reduzindo o próximo a um dado irrelevante e atentando contra a \textit{imago Dei} presente em cada indivíduo. Pior: afronta o próprio Deus, que zela por sua noiva com ciúmes santos. O primeiro alerta não é técnico, mas é teológico. E deve despertar o nosso temor.

A coleta, o armazenamento, o uso e a eventual exposição de dados sensíveis por instituições religiosas tocam diretamente na dignidade humana, no respeito à privacidade e na missão da igreja enquanto espaço de acolhimento e cuidado. Informações como nome, endereço, situação familiar, dados financeiros ou de saúde são confiadas às igrejas por pessoas que esperam que seus dados sejam tratados com zelo. O mau uso dessas informações caracteriza-se não apenas como afronta a Deus, mas também como violação da segurança individual, da confiança relacional e da integridade do testemunho cristão, passível de punição com multas para a igreja e todas as partes responsáveis, que podem chegar a 50 milhões de reais \cite{lgpd2018}, além da interdição de bens. Em tempos marcados por escândalos envolvendo vazamentos de dados, perseguições por ideologia, fragilidade institucional e desinformação, torna-se urgente repensar práticas internas das igrejas à luz da ética bíblica. Ignorar este tema pode comprometer não apenas a credibilidade e a legalidade da atuação pastoral, mas também corromper questões fundamentais da comunhão cristã. O silêncio ou a omissão diante desse cenário não é prova de neutralidade ou ignorância quanto ao assunto, mas sim de cumplicidade com o descumprimento da lei.

% RELEVÂNCIA
A presente monografia propõe uma abordagem teológica e ética sobre a gestão de dados sensíveis no contexto eclesiástico, articulando princípios bíblicos com as demandas contemporâneas impostas pela era digital. Não se trata de um debate técnico ou jurídico, mas de uma reflexão pastoral que resgata o valor da comunhão dos santos, marcada por responsabilidade e cuidado mútuo como expressão visível do amor ao próximo. A tecnologia, quando guiada por esse horizonte, passa a estar alinhada à missão da igreja, caminhando sob princípios bíblicos. Quando adotada de forma acrítica, porém, compromete vínculos e enfraquece o testemunho cristão.

% CONTEXTO
No contexto brasileiro, ainda que não haja um movimento institucionalizado ou expressivo de adoção de tecnologias com foco em privacidade e à segurança de forma alinhada a princípios legais por parte das igrejas, é possível identificar algumas poucas igrejas que demonstram preocupação com soberania tecnológica e ética cristã no uso de ferramentas digitais onde vemos uma busca por soluções mais alinhadas a esses princípios legais, inclusive por meio do uso de tecnologias livres em suas práticas pessoais e comunitárias. Assim como são raras as igrejas que se dedicam intencionalmente a esses temas, também são escassos ou não populares os recursos que ofereçam às lideranças eclesiásticas um direcionamento teológico consistente sobre proteção de dados, governança digital, soberania tecnológica e segurança da informação, alertando para os perigos da negligência desses temas. Esta lacuna motivou a formulação desta pesquisa.

% FUNDAMENTAÇÃO TEÓRICA
A construção desta reflexão teológica fundamenta-se na tradição reformada, tendo a \gls{cfw} como referência normativa. Destaca-se o capítulo XXIII, que estabelece como dever do povo orar pelos magistrados, honrá-los, pagar tributos, obedecer às suas ordens legais e sujeitar-se à sua autoridade. Esses princípios orientam a compreensão de que o cuidado com os dados não é apenas dever social, mas expressão de reverência ao Criador, de amor ao próximo e de submissão legítima à autoridade civil quando esta ordena o que é justo.

Além da \gls{cfw} como base confessional, esta pesquisa dialoga com três núcleos teóricos complementares: (i) a teologia sistemática clássica, representada por João Calvino, Herman Bavinck e Louis Berkhof; (ii) a reflexão ética e pastoral contemporânea, com Francis Schaeffer, Dietrich Bonhoeffer e Timothy Keller; e (iii) a cosmovisão cristã reformada, que afirma o senhorio de Cristo sobre todas as esferas da vida, inclusive a digital. Essa articulação permite compreender o tema não apenas como questão técnica ou social, mas como dimensão integral da vida cristã.

% PROBLEMA
A pergunta que norteia esta pesquisa é: como nós, enquanto corpo de Cristo comprometido com a comunhão dos santos, podemos viver essa comunhão de maneira ética e biblicamente orientada, sem negligenciar os princípios de privacidade e proteção de dados, especialmente diante dos desafios da era digital?

% HIPÓTESE
Parte-se da hipótese de que o modo como os dados pessoais e sensíveis são tratados nas igrejas pode reforçar ou comprometer a comunhão cristã, dependendo de sua conformidade com os princípios bíblicos, com foco em especial a Atos 2.42-47 e na \gls{cfw}.

% OBJETIVO GERAL
Diante dessa realidade e dos dilemas apresentados, esta pesquisa busca desenvolver uma abordagem teológica e pastoral que ofereça orientação fundamentada nas Escrituras, atenta à realidade vivida pelos membros e líderes das comunidades cristãs e com comprometimento ético para termos uma vivência comunitária que também pense e responsabilize-se coletivamente pela gestão de dados em ambiente digital ou não digital.

% OBJETIVOS ESPECÍFICOS
Especificamente, pretende-se: (i) investigar os fundamentos bíblicos e confessionais da comunhão cristã, reconhecendo sua centralidade para a vida da igreja; (ii) examinar criticamente os riscos envolvidos no uso descuidado de dados sensíveis, tanto para a integridade do testemunho cristão quanto para a ética pastoral e a conformidade legal; e (iii) apresentar caminhos práticos e teologicamente consistentes que contribuam para uma cultura de cuidado com os dados, em coerência com a fé professada e com os desafios contemporâneos.

% METODOLOGIA
Esta pesquisa adota uma abordagem qualitativa, de caráter exploratório e crítico, por buscar compreender sentidos, definições, princípios e implicações da comunhão cristã diante dos desafios digitais. A metodologia baseia-se em três eixos principais: (i) revisão bibliográfica, a partir da tradição reformada e de produções contemporâneas sobre ética cristã e tecnologia; (ii) análise documental de normativas legais, com ênfase na \gls{lgpd} (Lei nº 13.709/2018) e em orientações da \gls{anpd}; e (iii) análise crítica de experiências registradas no contexto eclesiástico brasileiro, em especial da \gls{ipb}, com atenção a lacunas de reflexão, riscos éticos e casos de vulnerabilidade digital. O tom adotado é intencionalmente exortativo em alguns trechos, pois a reflexão acadêmica proposta visa também provocar uma postura pastoral prática e urgente diante do problema apresentado. Assim, este estudo não apenas descreve a realidade, mas propõe diretrizes teológicas e pastorais aplicáveis às comunidades cristãs, favorecendo uma cultura de cuidado coletivo fundamentado na responsabilidade e na fidelidade bíblica no que se refere à gestão de dados.

% ESTRUTURA DO TRABALHO
Este trabalho está organizado em três capítulos, além da introdução e da conclusão. O primeiro capítulo apresenta os fundamentos bíblicos e teológicos da comunhão dos santos, destacando que a responsabilidade por zelar pela dignidade humana não é exclusiva de pastores ou profissionais ligados à tecnologia, mas é de todos os que participam ativamente da vida comunitária na igreja. O segundo capítulo investiga os riscos éticos e jurídicos relacionados à coleta, uso e exposição de dados sensíveis no contexto eclesiástico. O terceiro capítulo propõe caminhos para uma cultura de responsabilidade digital coerente com a fé cristã, a partir de princípios reformados e de boas práticas de cuidado com dados.

Ao final, serão apresentados anexos com sugestões de ferramentas práticas que podem auxiliar igrejas e comunidades na identificação de riscos e no fortalecimento da proteção de dados. Esses materiais incluem: listas de verificação para diagnóstico de segurança da informação, recomendações iniciais de boas práticas para mitigação de riscos e uma curadoria de sistemas e soluções acessíveis que podem ser adotados de forma responsável e coerente com os princípios discutidos ao longo do trabalho.

Que esta introdução tenha gerado não apenas interesse acadêmico, mas inquietação pastoral e reverente temor. Que o desconforto provocado pelo reconhecimento dos riscos e das consequências sirva como ponto de partida para um engajamento ético, conduzido por fundamentação teológica e resultando em atos pastorais e práticos fundamentados na Palavra de Deus. A boa notícia é que há caminhos possíveis, ancorados nas Escrituras, na tradição reformada e em práticas éticas que podem restaurar a confiança, proteger os vulneráveis e glorificar a Deus até mesmo na esfera digital.

Nos capítulos seguintes, essas possibilidades serão desenvolvidas com o intuito de fomentar uma cultura de cuidado e reverência, tanto nos espaços físicos da igreja quanto em seus registros e estruturas digitais.

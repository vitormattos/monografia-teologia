\newpage
\chapter{Riscos éticos da era digital que ameaçam a vivência da comunhão}

\section{Confiança e vulnerabilidade}

\subsection{Ambiente de confiança}

\subsection{Fragilidades nas relações}

\section{Práticas digitais e seus impactos}

\subsection{Uso irrefletido de ferramentas digitais nas igrejas}

\subsection{Naturalização de comportamentos antiéticos}

\subsection{Ausência de diretrizes claras}

\section{Dados sensíveis e riscos}

\subsection{Definição e contexto}

\subsection{Consequências da exposição inadequada}

\subsection{Responsabilidade institucional no tratamento dos dados}

\begin{citacao}
É dever do povo orar pelos magistrados, honrar as suas pessoas, pagar-lhes tributos e outros impostos, obedecer às suas ordens legais e sujeitar-se à sua autoridade.\cite[Cap. XXIII]{cfw}
\end{citacao}

\subsection{Privacidade, ética e limites legais na era da informação}

\begin{citacao}
as pessoas se disporão a trocar sua liberdade por segurança, especialmente se forem convencidas de que esse controle é científico e necessário para o bem-estar coletivo. \cite[p. 165]{schaeffer2002}
\end{citacao}

\section{Riscos na terceirização e múltiplas camadas digitais}

\subsection{Concentração de dados e poder institucional}

\subsection{Riscos ampliados pela terceirização digital}

\subsection{Rompimento do caráter relacional da comunhão}

\section{Desafios éticos na comunicação e na gestão de dados}

\subsection{Ausência de consentimento informado}

\subsection{Ética no compartilhamento de informações internas}

\subsection{Prestação de contas e responsabilidade confessional}

\textit{“O evangelho influencia todas as coisas”} \cite[p. 56]{keller2014}

\begin{citacao}
...há algumas circunstâncias, quanto ao culto de Deus e ao governo da Igreja, comum às ações e sociedades humanas, as quais têm de ser ordenadas pela luz da natureza e pela prudência cristã, segundo as regras gerais da palavra, que sempre devem ser observadas.” \cite[Cap. I, § VI]{cfw}
\end{citacao}

\textit{“Amar ao próximo é reconhecidamente um mandamento, não um conselho evangélico aleatório.”} \cite[p. 745]{calvino2022}

\subsection{Sem ética, não há arrependimento genuíno}

\textit{“Sem ética não há arrependimento genuíno. E sem arrependimento não há salvação”} \cite[p. 102]{stott2008}

\section{Fidelidade digital em prática}

\subsection{Síntese dos riscos éticos mapeados}

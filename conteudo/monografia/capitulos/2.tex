\chapter{Riscos éticos da era digital que ameaçam a vivência da comunhão}

\section{Entre a comunhão dos santos e o descuido ético}

\subsection{A tensão entre a liberdade e o controle na era digital}

Francis Schaeffer argumenta que o vácuo deixado pela perda de princípios cristãos não permanece neutro. Ele será inevitavelmente preenchido por alguma forma de controle. Segundo ele,
\begin{citacao}
Quando consideramos o surgimento de uma elite, um Estado autoritário, para preencher o vácuo deixado pela perda de princípios cristãos, não devemos ser ingênuos a ponto de achar que estamos nos referindo aos modelos de Stalin ou de Hitler. Devemos pensar num governo autoritário manipulador. Os governos modernos têm certas estratégias de manipulação à sua disposição que o mundo nunca conheceu antes. \cite[p.~181]{schaeffer2002}
\end{citacao}

Embora Schaeffer se referisse inicialmente ao Estado, sua análise é perfeitamente aplicável à realidade tecnológica contemporânea, marcada por empresas orientadas ao lucro, muitas vezes sem a devida reflexão ética sobre o impacto de suas práticas na segurança e privacidade dos indivíduos. Em uma sociedade que relativizou valores fundamentais, grandes corporações do setor tecnológico passaram a exercer um papel quase normativo, moldando comportamentos e promovendo um ambiente digital onde a ética e a razão são relativizadas, e a liberdade frequentemente ignorada em troca de conveniência, engajamento ou sensação de segurança, tudo visando o fortalecimento de produtos e marcas.

Schaeffer reconhece que \textit{``o computador é útil enquanto ferramenta, mas é neutro. Portanto, ele pode ser usado para o bem ou então para fins destrutivos''} \cite[p.~179]{schaeffer2002}. A questão, portanto, não está na existência dos meios digitais, mas na ausência de reflexão ética e teológica sobre seu uso, especialmente no contexto da igreja.

Esse cenário representa um risco ético direto à vivência da comunhão cristã, pois tais estruturas influenciam a forma como a igreja se comunica, se organiza e zela por sua membresia. O cuidado pastoral, nesse contexto, deve incluir também uma vigilância crítica quanto ao uso de tecnologias que podem comprometer a liberdade, a confidencialidade e o vínculo fraterno entre os crentes. Como observam Peixoto e Ehrhardt Jr., \textit{``essas mesmas tecnologias ajudaram a tornar cada vez mais indistinguível os limites entre o que é público e o que é privado''} \cite[p.~43]{peixoto2020ressignificacao}, o que reforça a necessidade de discernimento ético no tratamento de dados e informações sensíveis, mesmo dentro de comunidades de fé.

No contexto eclesiástico, tais riscos se manifestam por meio da coleta excessiva de dados, do uso indiscriminado de imagens de membros, da exposição de histórias pessoais em pregações sem consentimento ou da adoção de soluções tecnológicas sem análise ética. Ainda que motivadas por boas intenções, essas práticas comprometem a confiança mútua, que é fundamento da comunhão dos santos, e obscurecem o caráter relacional e voluntário da vida cristã. Como alerta Lécio Machado:
\begin{citacao}
    é necessário que os líderes religiosos estejam preparados para implantar a legislação protecionista de dados nas rotinas da igreja, garantindo um ambiente saudável aos seus fiéis, frequentadores, colaboradores e outros que de alguma forma têm seus dados tratados pela igreja. \cite[p.~5]{machado2020}
\end{citacao}

Esse uso acrítico dos meios digitais revela uma inclinação pecaminosa de confiar mais na eficiência do que no cuidado, e de negligenciar a responsabilidade ética diante do próximo. Trata-se de uma prática que ignora a dignidade do outro e compromete a comunhão, violando princípios que deveriam refletir o caráter de Deus no cuidado mútuo.

As contribuições contemporâneas de estudiosos do direito e da ética digital, como Peixoto, Ehrhardt Jr. e Machado, evidenciam a urgência de um cuidado pastoral informado e responsável frente aos desafios tecnológicos atuais. No entanto, esses desafios não se explicam apenas por descuidos administrativos ou operacionais, mas por uma mudança mais profunda na forma como a sociedade compreende o ser humano, a verdade e a liberdade. Nesse ponto, as advertências de Francis Schaeffer continuam relevantes, pois iluminam as bases filosóficas e espirituais que tornam possível o uso imprudente de meios tecnológicos no seio da igreja.

Essa crítica de Schaeffer encontra eco no que foi estabelecido no Capítulo 1 desta monografia: a dignidade da pessoa humana tem sua base na doutrina bíblica de que o ser humano foi criado à imagem de Deus e carece de comunhão, de relacionamento. A partir dessa cosmovisão, não é possível justificar o uso de qualquer meio, por mais eficiente ou inovador que pareça, sem considerar os princípios que orientam seu uso. Quando tecnologias são adotadas apenas por conveniência ou com o intuito de alcançar resultados que se acredita serem bons, sem reflexão ética, corre-se o risco de utilizar meios de forma inadequada, expondo aspectos da vida das pessoas que deveriam permanecer no âmbito particular, agindo, assim, de modo incompatível com o evangelho.

Schaeffer nos alerta que, na ausência de fundamentos cristãos, surgem formas sutis de controle que moldam o comportamento das pessoas em nome de objetivos diversos. Ele observa que \textit{``o homem moderno deseja ser livre para criar o seu próprio destino, por mais que imagine saber que esteja determinado''} \cite[p.~166]{schaeffer2002}. Essa contradição se expressa no desejo contemporâneo de resolver problemas com rapidez, adotando qualquer solução disponível sem refletir sobre os meios utilizados e os impactos que esses meios podem ter sobre os fins que se deseja alcançar. No contexto da igreja, isso se revela quando tecnologias são utilizadas sem critérios teológicos ou éticos, apenas por parecerem úteis ou inovadoras. Esse uso acrítico pode resultar na exposição de informações íntimas, na quebra da confiança pastoral e na deterioração da comunhão cristã. Por isso, é necessário que o uso de meios tecnológicos na vida eclesiástica seja orientado por discernimento bíblico, e não apenas por sua capacidade de produzir resultados.

\subsection{À luz da Confissão de Fé de Westminster}

A Confissão de Fé de Westminster nos fornece parâmetros para lidar com dilemas que, embora ausentes no contexto histórico da redação original, exigem respostas coerentes com os princípios reformados. Nela lemos que:

\begin{citacao}
    há algumas circunstâncias, quanto ao culto de Deus e ao governo da Igreja, comuns às ações e sociedades humanas, as quais têm de ser ordenadas pela luz da natureza e pela prudência cristã, segundo as regras gerais da Palavra \cite[Cap.~I, §~VI]{cfw}    
\end{citacao}

Isso implica reconhecer que nem tudo está explicitado nas Escrituras, mas que tudo deve ser regulado à luz dos seus ensinamentos, inclusive os desafios trazidos pela cultura digital e pelo tratamento de dados.

Nesse sentido, o cuidado com as informações pessoais na igreja, especialmente aquelas que revelam aspectos sensíveis da vida dos membros, não é apenas uma demanda jurídica oriunda de leis que tratam deste assunto, como a \gls{lgpd}, mas um reflexo de princípios bíblicos como o amor ao próximo, a justiça e a preservação da dignidade, que se expressam na prática da comunhão dos santos. A ética cristã não se limita a reagir aos problemas, mas se antecipa a eles com discernimento e temor de Deus. Essa postura preventiva está em harmonia com os princípios bíblicos que norteiam a vida comunitária e o cuidado pastoral.

A diferença de autoridade que naturalmente existe entre líderes e membros, especialmente no trato com dados pessoais, exige sabedoria pastoral e estruturas claras de prestação de contas. Nesse aspecto, a \gls{cfw} diz que \textit{``é dever do povo orar pelos magistrados, honrar as suas pessoas, pagar-lhes tributos e outros impostos, obedecer às suas ordens legais e sujeitar-se à sua autoridade''} \cite[Cap.~XXIII]{cfw}, exceto quando essas leis colidirem com a consciência cristã, o que claramente não é o caso aqui. Respeitar a \gls{lgpd}, portanto, não é questão exclusiva para empresas que visam o lucro, mas também uma forma de testemunhar que a igreja caminha de modo íntegro, mesmo quando isso envolve estruturas normativas externas.

A proteção de dados, quando exercida com transparência e zelo, é um modo de honrar a comunhão dos santos. Trata-se de reconhecer que, num ambiente comunitário, informações pessoais não podem ser tratadas como bem público, tampouco expostas sob a justificativa de espontaneidade pastoral. A prudência cristã, exigida pela \gls{cfw}, demanda que o acesso e o uso de dados sejam mediados por critérios éticos, alinhados com o evangelho e com a responsabilidade que decorre do cuidado de almas.

\section{Descuido, exposição e quebra da confiança}

Segundo notícia publicada pelo portal Cátedras, o Tribunal de Justiça de São Paulo manteve a condenação de uma igreja ao pagamento de R\$ 20 mil por danos morais, devido à divulgação não autorizada da imagem de uma fiel em vídeo que viralizou nas redes sociais. \cite{catedras2024}

\section{A lei denuncia, Cristo restaura}

\textit{“Amar ao próximo é reconhecidamente um mandamento, não um conselho evangélico aleatório.”} \cite[p. 745]{calvino2022}

\textit{“Sem ética não há arrependimento genuíno. E sem arrependimento não há salvação.”} \cite[p. 102]{stott2008}

\textit{“O evangelho influencia todas as coisas.”} \cite[p. 56]{keller2014}

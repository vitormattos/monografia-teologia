\chapter{Riscos éticos da era digital que ameaçam a vivência da comunhão}

\section{Entre a comunhão dos santos e o descuido ético}

\subsection{A tensão entre a liberdade e o controle na era digital}

A seguir, desenvolveremos a crítica de \citeonline{schaeffer2002} onde ele trata sobre ``\textit{principais características de nossa época em busca de soluções para os problemas que enfrentamos}''\footnote{Síntese do que trata o livro \textit{``Como viveremos?''} \cite{schaeffer2002}} em três pontos\footnote{Esta estrutura de 3 pontos foca nas necessidades deste trabalho monográfico dialogando com o capítulo de \textit{``Como viveremos?''} entitulado \textit{``Manipulação e a nova elite''}}: \textbf{1.} sua denúncia à manipulação no vácuo de fundamentos, \textbf{2.} os impactos dessa lógica na comunhão eclesiástica e, por fim, \textbf{3.} a necessidade de um discernimento ético e bíblico frente às tecnologias digitais.

\subsubsection{Manipulação digital diante da ausência de fundamentos cristãos}

Conforme Francis Schaeffer destaca, a ausência dos princípios cristãos gera um vácuo ético rapidamente ocupado por mecanismos de controle ou imposição, fenômeno especialmente evidente na atual realidade tecnológica, onde grandes empresas e plataformas digitais podem assumir papéis quase normativos. Com base na sua avaliação:

\begin{citacao}
    Quando consideramos o surgimento de uma elite, um Estado autoritário, para preencher o vácuo deixado pela perda de princípios cristãos, não devemos ser ingênuos a ponto de achar que estamos nos referindo aos modelos de Stalin ou de Hitler. Devemos pensar num governo autoritário manipulador. Os governos modernos têm certas estratégias de manipulação à sua disposição que o mundo nunca conheceu antes. \cite[p.~168]{schaeffer2002}
\end{citacao}

Embora Schaeffer se referisse inicialmente ao Estado, sua análise é perfeitamente aplicável à realidade tecnológica contemporânea, marcada por empresas orientadas ao lucro, muitas vezes sem a devida reflexão ética sobre o impacto de suas práticas na segurança e privacidade dos indivíduos. Em uma sociedade que relativizou valores fundamentais, grandes corporações do setor tecnológico passaram a exercer um papel quase normativo, moldando comportamentos e promovendo um ambiente digital onde a ética e a razão são relativizadas, e a liberdade frequentemente ignorada em troca de conveniência, engajamento ou sensação de segurança, tudo visando o fortalecimento de produtos e marcas.

\citeonline[p.~183]{schaeffer2002} reconhece que \textit{``o computador é útil enquanto ferramenta, mas é neutro. Portanto, ele pode ser usado para o bem ou então para fins destrutivos''}. A questão, portanto, não está na existência dos meios digitais, mas na ausência de reflexão ética e teológica sobre seu uso, especialmente no contexto da igreja.

\subsubsection{O impacto sobre a vivência da comunhão nas igrejas}

Esse cenário representa um risco ético direto à vivência da comunhão cristã, pois tais estruturas influenciam a forma como a igreja se comunica, se organiza e zela por sua membresia. O cuidado pastoral, nesse contexto, deve incluir também uma vigilância crítica quanto ao uso de tecnologias que podem comprometer a liberdade e a confiança essenciais para a vida em comunidade. Como observam Peixoto e Ehrhardt Jr., \textit{``essas mesmas tecnologias ajudaram a tornar cada vez mais indistinguível os limites entre o que é público e o que é privado''} \cite[p.~43]{peixoto2020ressignificacao}, o que reforça a necessidade de discernimento ético no tratamento de dados e informações sensíveis, mesmo dentro de comunidades de fé.

No contexto eclesiástico, tais riscos podem ser identificados, por exemplo, por meio da coleta excessiva de dados, no uso indiscriminado de imagens de membros, na exposição de histórias pessoais em pregações sem consentimento, em assuntos particulares que são expostos ou da adoção de soluções tecnológicas sem análise ética. Ainda que motivadas por boas intenções, essas práticas comprometem a confiança mútua, que é fundamento da comunhão dos santos, e obscurecem o caráter relacional e voluntário da vida cristã. Como alerta \citeonline[p.~5]{machado2020}:

\begin{citacao}
    é necessário que os líderes religiosos estejam preparados para implantar a legislação protecionista de dados nas rotinas da igreja, garantindo um ambiente saudável aos seus fiéis, frequentadores, colaboradores e outros que de alguma forma têm seus dados tratados pela igreja.
\end{citacao}

Esse uso acrítico dos meios digitais revela uma inclinação pecaminosa de manter-se acomodado em conservar hábitos comumente adotados em vez de expressar o amor ao próximo pelo cuidado com as informações. Agindo assim, incorre-se em negligência de responsabilidades éticas ao se ignorar os direitos do outro, comprometendo a comunhão e violando princípios que deveriam refletir o caráter de Deus no cuidado mútuo.

\subsubsection{Discernimento bíblico diante da lógica da produtividade}

As contribuições contemporâneas de estudiosos do direito e da ética digital, como \citeonline{peixoto2020ressignificacao}, evidenciam a urgência de um cuidado pastoral informado e responsável frente aos desafios tecnológicos atuais. No entanto, esses desafios não se explicam apenas por descuidos administrativos ou operacionais, mas por uma mudança mais profunda na forma como a sociedade compreende o ser humano, a verdade e a liberdade. Nesse ponto, as advertências de Francis Schaeffer continuam relevantes, pois iluminam as bases filosóficas e espirituais que tornam possível o uso imprudente de meios tecnológicos no seio da igreja.

Diante disto, não podemos ignorar a realidade de que não é possível justificar o uso de qualquer solução de tecnologia da informação e comunicação, por mais eficiente ou inovadora que pareça, sem considerar os princípios que orientam seu uso. Quando tecnologias são adotadas apenas por conveniência ou com o intuito de alcançar resultados que se acredita serem bons, sem reflexão ética, corre-se o risco de se utilizar meios de forma inadequada, expondo aspectos da vida das pessoas que deveriam permanecer no âmbito particular, agindo, assim, de modo incompatível com o evangelho.

Schaeffer nos alerta que, na ausência de fundamentos cristãos, surgem formas sutis de controle que moldam o comportamento das pessoas em nome de objetivos diversos. Ele observa que \textit{``o homem moderno deseja ser livre para criar o seu próprio destino, por mais que imagine saber que esteja determinado''} \cite[p.~169]{schaeffer2002}. Essa contradição se expressa no desejo contemporâneo de resolver problemas com rapidez, adotando qualquer solução disponível sem refletir sobre os meios utilizados e os impactos que esses meios podem ter sobre os fins que se deseja alcançar. No contexto da igreja, isso se revela quando tecnologias são utilizadas sem critérios para se observar critérios de avaliação tecnológica ou implicações éticas decorrentes do uso de certas tecnologias, e se segue em frente com as soluções adotadas apenas por parecerem úteis ou inovadoras. Esse uso acrítico pode resultar na exposição de informações íntimas, em quebra da confiança pastoral e na deterioração da comunhão cristã. Por isso, é imprescindível que o uso de meios tecnológicos na vida eclesiástica seja orientado de forma crítica por discernimento bíblico, e não apenas por sua capacidade de produzir resultados ou pela popularidade de uma solução tecnológica específica.

A crítica de Schaeffer foi escrita em 1976, porém ecoa no presente. Em uma era que idolatra a eficácia e minimiza a ética, a igreja corre o risco de incorporar tecnologias sem discernimento, reproduzindo, ainda que involuntariamente, as mesmas formas sutis de manipulação que ele denunciou. Recuperar fundamentos cristãos e exercer vigilância ética com soberania tecnológica são passos essenciais para que a comunhão não seja corrompida por soluções que prometem eficiência, mas comprometem a verdade e o cuidado pastoral.

\subsection{À luz da Confissão de Fé de Westminster}

A \gls{cfw} nos fornece parâmetros para lidar com dilemas que, embora ausentes no contexto histórico da redação original, exigem respostas coerentes com os princípios reformados. Nela lemos que:

\begin{citacao}
    há algumas circunstâncias, quanto ao culto de Deus e ao governo da Igreja, comuns às ações e sociedades humanas, as quais têm de ser ordenadas pela luz da natureza e pela prudência cristã, segundo as regras gerais da Palavra \cite[Cap.~I, §~VI]{cfw}    
\end{citacao}

Isso implica reconhecer que nem tudo está explicitado nas Escrituras, mas que tudo deve ser regulado à luz dos seus ensinamentos, inclusive os desafios trazidos pela cultura digital e pelo tratamento de dados.

Nesse sentido, o cuidado com as informações pessoais na igreja, especialmente aquelas que revelam aspectos sensíveis da vida dos membros, não é apenas uma demanda jurídica oriunda de leis que tratam deste assunto, como a \gls{lgpd}\footnote{Lei nº 13.709/2018 \cite{lgpd2018}}, mas um reflexo de princípios bíblicos como o amor ao próximo, a justiça e a preservação da dignidade, que se expressam na prática da comunhão dos santos. A ética cristã não se limita a reagir aos problemas, mas se antecipa a eles com discernimento e temor de Deus. Essa postura preventiva está em harmonia com os princípios bíblicos que norteiam a vida comunitária e o cuidado pastoral.

A diferença de autoridade que naturalmente existe entre líderes e membros, especialmente no trato com dados pessoais, exige sabedoria pastoral e estruturas claras de prestação de contas. Nesse aspecto, a \gls{cfw} diz que 

\begin{citacao}
    é dever do povo orar pelos magistrados, honrar as suas pessoas, pagar-lhes tributos e outros impostos, obedecer às suas ordens legais e sujeitar-se à sua autoridade \cite[Cap.~XXIII]{cfw}    
\end{citacao}

A única exceção é quando essas leis colidirem com a consciência cristã, o que claramente não é o caso aqui quando tratamos de zelo com dados sensíveis. Respeitar a \gls{lgpd}, portanto, não é questão exclusiva para empresas que visam o lucro, mas também uma forma de testemunhar que a igreja caminha de modo íntegro, mesmo quando isso envolve estruturas normativas externas.

A proteção de dados, quando exercida com transparência e zelo, é um modo de honrar a comunhão dos santos. Trata-se de reconhecer que, num ambiente comunitário, informações pessoais não podem ser tratadas como bem público, tampouco expostas sob a justificativa de espontaneidade pastoral. A prudência cristã, exigida pela \gls{cfw}, demanda que o acesso e o uso de dados sejam mediados por critérios éticos, alinhados com o evangelho e com a responsabilidade que decorre do cuidado de vidas.

\section{A negligência digital e os dados do Censo 2022}

O Censo Demográfico de 2022 realizado pelo \citeonline[p.~41]{ibge2025religiao}, em sua apresentação de resultados preliminares, revelou dados relevantes para esta pesquisa com informações sobre o acesso domiciliar à internet entre diferentes grupos religiosos no Brasil. De acordo com a \hyperref[fig:censo_demografico_2022]{Figura~\ref{fig:censo_demografico_2022}}, que se encontra no \hyperref[apendice:censo_demografico_2022]{Apêndice~\ref*{apendice:censo_demografico_2022}}, entre pessoas com 10 anos de idade ou mais que se identificam como evangélicos, 90,5\% vivem em domicílios com acesso à internet, superando inclusive a média nacional, que é de 89,2\%.

Esse alto percentual evidencia que a maioria dos fiéis evangélicos possui acesso à internet em suas residências, isto sem levar em consideração o acesso móvel (celulares), o que levanta questões importantes sobre o impacto dessa realidade na vivência da fé cristã. Embora o dado, por si só, seja algo que não ofereça algum tipo de juízo ou seja algo negativo, ele aponta para a necessidade de refletirmos sobre o uso da tecnologia dentro das igrejas. A internet se tornou um espaço comum de circulação de informações onde pessoas interagem e expressam assuntos diversos, inclusive questões religiosas. Mas será que esse uso tem sido acompanhado de critérios éticos e orientados por uma cosmovisão cristã?

Infelizmente, observa-se que muitas comunidades de fé utilizam tecnologias digitais sem uma reflexão teológica mais profunda. O uso de redes sociais, transmissões ao vivo, armazenamento de dados, sistemas diversos utilizados e compartilhamento de informações sensíveis muitas vezes ocorre de forma acrítica, sem considerar os riscos envolvidos nem os princípios que deveriam nortear o comportamento cristão. Essa ausência de reflexão pode comprometer diretamente a comunhão cristã. A exposição indevida de dados pessoais, a divulgação de imagens sem consentimento, a transmissão de testemunhos sem preparo ou a não proteção da identidade dos que frequentam uma comunidade de fé, tudo isso pode fragilizar vínculos, gerando constrangimentos e até configurando violações à privacidade e à dignidade humana que são passíveis de punição pelas leis de nosso país. Em vez de proteger a comunidade e servir como instrumento de se propagar a Palavra de Deus, a tecnologia mal utilizada pode se tornar instrumento de quebra de confiança e divisão, gerando feridas irreparáveis.

É importante lembrar que a responsabilidade por essa situação não recai apenas sobre os pastores, mas ela é de toda a liderança e membros das igrejas. Presbíteros, diáconos, professores de escola dominical, voluntários da mídia, gestores de redes sociais e todos os que lidam com informações, inclusive membros que não são responsáveis por alguma atividade específica, mas por apenas estarem na igreja já estão gerando dados de alguma forma, devem ser formados e conscientizados quanto aos limites e cuidados no uso da tecnologia.

Diante disso, a negligência na formação digital e ética das igrejas deixa de ser apenas um problema organizacional e passa a ser um desafio pastoral e missional pois impacta diretamente na missão de pregação do evangelho. A fidelidade à missão cristã exige o cuidado pastoral com os relacionamentos. É importante que pensemos e vivamos a proteção da intimidade promovendo uma cultura digital coerente com os valores do evangelho.

Essa reflexão nos conduz naturalmente à próxima seção, onde trataremos das implicações práticas da exposição indevida de dados e de como, uma vez violada a privacidade, os efeitos podem ser irreversíveis, trazendo danos profundos à comunhão e ao testemunho cristão.

\section{Descuido, exposição e quebra da confiança}

Uma ilustração que retrata muito bem a irreversibilidade de reparar plenamente os danos causados pela exposição indevida de informações é o conto popular da fofoca, citado na tese de doutorado de \citeonline[p.~141]{gondim2015}. Nele, um homem, conhecido por espalhar histórias alheias, é orientado por um sacerdote a subir à torre da igreja com um travesseiro de penas. Foi solicitado ao homem que, estando lá, rasgasse a fronha e deixasse que o vento levasse as penas, espalhando-as. Ao descer, o sacerdote lhe pede que recolha todas as penas e as coloque novamente no travesseiro. O homem protesta, dizendo ser impossível, pois o vento as dispersou para todos os lados. Assim como essas penas, os dados pessoais, uma vez expostos indevidamente, fogem ao controle e podem causar danos irreparáveis. A gravidade da fofoca, enquanto quebra de confiança e violação da privacidade, é também ressaltada nas Escrituras\footnote{A fofoca é um grande obstáculo à comunhão dos santos, como expressam textos como \gls{pv} 11.13: \textit{``O mexeriqueiro revela os segredos, mas o fiel de espírito os encobre.''} (\gls{naa}), e \gls{pv} 25.9–10: \textit{``Defenda a sua causa diretamente com o seu próximo e não revele o segredo do outro. Do contrário, quem o ouvir poderá envergonhá-lo, e você nunca se livrará dessa má fama.'' (\gls{naa})}}. A autora, em outro artigo, comenta:

\begin{citacao}
    Após um vazamento, não há como saber quem ou quantas pessoas os acessaram. Também, não é possível ter o controle de que serão apagados ou não serão compartilhados para outras pessoas. Por isso, fala-se em prejuízos incomensuráveis \cite{migalhas2023}.
\end{citacao}

No ambiente eclesiástico, essa imagem ajuda a refletir sobre a gravidade de práticas corriqueiras, mas arriscadas. Por mais que haja boas intenções, por exemplo, em fazer um pedido de oração durante uma transmissão ao vivo, os riscos são reais e geram processos com condenações inclusive. Em um caso o \gls{tjsp} manteve a condenação de uma igreja ao pagamento de R\$ 20 mil por danos morais, após a divulgação não autorizada da imagem de uma fiel que chorava durante um culto transmitido. A gravação viralizou nas redes sociais, provocando forte constrangimento e reconhecida violação de sua privacidade\footnote{A notícia diz: \gls{tjsp} mantém condenação a igreja por exposição de fiel sem autorização. Trata-se do processo nº 1004138-84.2023.8.26.0002. \cite{catedras2024}} Casos como esse evidenciam que realizar filmagens sem consentimento para uso de imagem, quando permitem a identificação de uma pessoa, mesmo com boas intenções, pode configurar grave violação à privacidade. Esta sentença do \gls{tjsp} nos mostra que devemos ter especial cuidado na gravação e transmissão de testemunhos, relatos de experiências pessoais, histórias do campo missionário que possam identificar pessoas em situação de vulnerabilidade social ou expondo intimidades e até mesmo na divulgação de fotos de cultos expondo os presentes. Ainda que haja uma cultura de confiança e informalidade dentro da comunidade, isso não dispensa a responsabilidade ética e legal na proteção da dignidade de cada indivíduo.

A exposição indevida de informações, mesmo que motivada por empatia ou zelo espiritual, pode gerar constrangimento, estigmatização e até danos emocionais. Além disso, uma vez publicado ou transmitido digitalmente, o conteúdo pode ser replicado incontáveis vezes, saindo completamente do controle da liderança. Como observa \citeonline{migalhas2023}, ao retomar o conto das penas: ``seria impossível recuperá-las''. O risco não está apenas na infração da \gls{lgpd}, mas na quebra de confiança que fere o próprio espírito da comunhão cristã.

O vazamento de dados no contexto religioso pode alcançar dimensões coletivas, afetando não apenas um indivíduo isoladamente, mas toda a comunidade envolvida. Quando informações sensíveis são expostas em ambientes como transmissões ao vivo, listas de oração, redes sociais ou materiais institucionais, a consequência não se limita à dor pessoal, mas compromete a confiança institucional, desestrutura vínculos e pode impactar o testemunho público da igreja. Nesses casos, é possível enquadrar o ocorrido como um \textit{dano enorme}\footnote{Uma lesão excepcional que atinge uma coletividade. Para este tipo de dano, consideram-se como requisitos: ``a) que se trate de danos de proporções catastróficas que causem considerável clamor social; b) que tenham causalidade múltipla, difusa ou indeterminada; c) que se relacionem ao modo de vida moderna'' (apud \citeonline{migalhas2023}).}, deixando marcas irreparáveis.

\section{Riscos legais}

É importante ressaltar que todos os agentes envolvidos na administração e liderança da igreja respondem solidariamente às sanções previstas na legislação vigente, inclusive nas esferas administrativa, civil e penal. Como exemplo, a \citeonline[Art.~52, inciso~II]{lgpd2018} estabelece a possibilidade de aplicação de multa simples de até dois por cento do faturamento anual bruto da igreja, por infração, limitada a cinquenta milhões de reais. O inciso III prevê ainda a aplicação de multa diária, respeitado o mesmo limite máximo por infração.

Isso significa que pastores, membros do conselho e diretorias da igreja podem, em caso de responsabilização, ter seus bens pessoais utilizados para o pagamento das penalidades, incluindo imóveis, veículos e contas bancárias para o pagamento da multa.

Em igrejas onde o tema da proteção de dados e segurança da informação nunca foi tratado com a devida seriedade, torna-se muito difícil estimar quantas infrações podem ter sido cometidas e, consequentemente, qual seria o valor total das multas. Como cada infração é considerada de forma individual, os valores das multas podem se multiplicar rapidamente.

Em uma igreja que normalmente tem uma diversidade de público que frequenta as atividades, o risco é ainda mais ampliado. Visitantes de outras denominações, pessoas de diferentes religiões ou mesmo sem vínculo religioso, muitas vezes em situações emocionais ou sociais delicadas, podem se sentir lesadas caso suas informações sejam utilizadas de maneira indevida.

Mesmo que algumas lideranças considerem remota a possibilidade de sofrerem esse tipo de penalidade por de forma equivocada acreditar que isto só serve para o mundo corporativo, caso ocorra, o impacto tende a ser devastador para toda a comunidade. A lógica da legislação não é apenas educativa ou orientadora, mas visa aplicar punições exemplares que inibam o descumprimento da lei.

Um exemplo emblemático dessa vulnerabilidade ocorreu com a empresa InChurch, especializada em soluções digitais para igrejas. Em 2024, um vazamento de dados expôs informações pessoais de milhares de fiéis, como nomes, e-mails e números de telefone, sem que houvesse garantias claras sobre a segurança ou o consentimento no uso dessas informações. O caso ganhou repercussão nacional e trouxe à tona os riscos da terceirização de serviços digitais sem o devido cuidado na escolha dos fornecedores, das tecnologias utilizadas e na governança dos dados sensíveis\footnote{Vazamento de dados: empresa InChurch expõe informações de fiéis, segundo empresa de segurança. \cite{almeida_inchurch_2024}}. A terceirização não exime as lideranças do dever de zelar pela confidencialidade e integridade das informações confiadas à igreja. Propostas de soluções tecnológicas para essa realidade serão apresentadas no próximo capítulo.

\section{A lei denuncia, Cristo restaura}

\textit{``Amar ao próximo é reconhecidamente um mandamento, não um conselho evangélico aleatório.''}\footnote{A afirmação de Calvino expressa a centralidade do amor ao próximo na vida cristã. Herman Bavinck, ao comentar \gls{lc} 17.10, reforça que mesmo que o ser humano cumprisse toda a Lei, ainda assim só lhe caberia dizer: ``Somos servos inúteis, porque fizemos apenas o que devíamos fazer'' (cf. \cite[p.~738]{bavinck2012}). Isso demonstra que o amor ao próximo é um dever e não uma opção moral facultativa.} \cite[p.~745]{calvino2022}

A denúncia da Lei aponta para a graça. Ela escancara a realidade da queda e da ruptura que separa o ser humano de Deus e do próximo. Por um lado, revela o padrão da justiça divina; por outro, evidencia a total incapacidade humana de cumpri-lo plenamente, mostrando que somente mediante a graça divina é possível restaurar nossa relação com Deus e com o próximo. Ao apontar para a graça, a Lei conduz a Cristo, o único mediador e cumprimento da justiça de Deus (\gls{gl} 3.24). É exatamente aí que o evangelho se apresenta como resposta, pois Cristo não veio para abolir a Lei, mas para cumpri-la (\gls{mt} 5.17), reconciliando, em sua cruz, não apenas Deus e os homens, mas também os homens entre si (\gls{ef} 2.14-16).

A Lei, embora justa e santa, não salva. Ela revela o pecado, condena o homem e o deixa sem defesa diante de Deus. Apenas em Cristo, como sacrifício expiatório, é possível sermos justificados. Como afirma Bavinck\footnote{Bavinck constrói este pensamento ao discorrer sobre a natureza da justificação e centralidade em Cristo como fundamento.}:

\begin{citacao}
    Portanto, já que, de acordo com a lei, Deus condena e tem de nos condenar por causa do nosso pecado, aprouve a ele revelar sua justiça, isto é, sua justiça judiciária, independente da lei e das obras da lei, somente através do evangelho. Deus propôs Cristo como meio ou sacrifício expiatório, mostrando-se, assim, justo e, ao mesmo tempo, capaz de justificar aqueles que têm fé em Jesus\footnote{Esse comentário de Bavinck está em consonância com \gls{rm} 3.21-26, enfatizando que a justiça de Deus é revelada em Cristo, independentemente das obras da Lei.}. \cite[p.~212]{bavinck2012}
\end{citacao}

Nesse sentido, a comunhão dos santos, tantas vezes ferida por egoísmos, abusos, omissões e práticas impensadamente danosas, encontra em Cristo seu ponto de restauração. Ele não apenas redime pessoas individualmente, mas as insere em um corpo coletivo, interdependente e espiritual, convidando cada membro a cuidar uns dos outros. O amor ao próximo não é um apêndice do cristianismo; é nele que Cristo resume toda a Lei (\gls{mt} 22.39). Como escreveu Calvino, trata-se de um mandamento, não de um conselho facultativo. Desprezá-lo é tornar o cristianismo estéril, mesmo quando adornado de ortodoxia.

Se Cristo é o fundamento e o agente dessa restauração, então ela precisa se manifestar na prática cotidiana da fé. Isso implica renunciar a posturas que ferem a comunhão, cultivar humildade para reconhecer limitações, buscar formação contínua e envolver pessoas capacitadas que nos auxiliem a viver com responsabilidade e graça no contexto comunitário. A restauração que recebemos de Cristo deve ser refletida no zelo com que tratamos uns aos outros, no cuidado mútuo e no compromisso de construir relacionamentos que expressem a reconciliação operada na cruz. A comunhão dos santos, assim compreendida, deixa de ser apenas um ideal teológico e se torna um chamado diário à conversão relacional e ao exercício ativo do amor.

\textit{``Sem ética não há arrependimento genuíno. E sem arrependimento não há salvação.''} \cite[p.~102]{stott2008}

A ética cristã não é uma formalidade, mas fruto direto da ação regeneradora do Espírito. Segundo John Stott, ela não é opcional. Onde há arrependimento verdadeiro, há transformação ética. E onde não há transformação, não há evidência de salvação (\gls{tg} 2.14-26). O evangelho não convida apenas à fé intelectual, mas à conversão moral, concreta e visível.

No contexto contemporâneo, esse chamado ganha um novo campo de aplicação: a esfera digital. A tecnologia tornou-se uma extensão do agir humano. Por isso, o testemunho cristão precisa alcançar também os ambientes onde interações, decisões e exposições acontecem por meio de dados e telas. Quando a igreja falha nesse aspecto, não está apenas infringindo princípios legais relacionados à proteção de dados, mas revela também incoerências teológicas. O uso descuidado de imagens, o sigilo quebrado, a coleta sem consentimento, o armazenamento inseguro de informações sensíveis e o uso de soluções tecnológicas que não atendem aos requisitos mínimos de privacidade\footnote{Vale aqui o comentário de que privacidade também diz respeito ao conceito de soberania tecnológica, que é a capacidade de ter posse total da tecnologia envolvida no tratamento dos dados, ponto que trabalharemos melhor no próximo capítulo.} comprometem o amor cristão.

Nesse cenário, a \gls{lgpd} pode ser compreendida como mais do que uma exigência jurídica. Suas diretrizes, quando iluminadas pela Escritura, tornam-se instrumentos de cuidado e responsabilidade. Proteger dados, adotar boas práticas digitais e respeitar a privacidade dos irmãos não é apenas uma questão institucional. É uma forma de amar o próximo e de demonstrar, na prática, o arrependimento ético que a fé exige.

Essa consciência se torna ainda mais urgente em tempos de vulnerabilidade exposta. A comunhão dos santos envolve compartilhar dons e serviços, mas também fragilidades. Hoje, essas fragilidades estão muitas vezes digitalizadas: confidências armazenadas, orientações pastorais registradas, histórias pessoais arquivadas. Ignorar o cuidado com essas informações pode gerar dor. Em alguns casos, pode comprometer relacionamentos ou causar escândalos evitáveis.

Cuidar dos dados é cuidar das pessoas. A ética digital, nesse sentido, é altamente pastoral. A confidencialidade no aconselhamento, a prudência na exposição pública e a cautela com imagens de crianças não são meras exigências técnicas. São expressões tangíveis de um cristianismo coerente com o evangelho que proclama.

Em um tempo que normaliza o descuido, a igreja é chamada a viver de forma distinta. O arrependimento que conduz à salvação precisa alcançar também a forma como lidamos com a vida digital. Se a comunhão dos santos foi restaurada por Cristo, ela deve ser vivida com temor, reverência e integridade em todas as suas dimensões, inclusive na maneira como utilizamos a tecnologia para armazenar, processar e proteger informações no contexto da vida comunitária.

\textit{``O evangelho influencia todas as coisas.''} \cite[p.~56]{keller2014}

Tim Keller resume com precisão a abrangência do senhorio de Cristo: nada escapa à influência do evangelho. Isso inclui o modo como a igreja se posiciona diante das tecnologias. Contudo, o maior desafio é que isso não seja apenas conformidade legal, mas convicção evangélica.

Em um mundo caído, onde a ética parece ser descartável e a exposição digital é a moeda da visibilidade e realização, a igreja é chamada a ser sal e luz também nas práticas digitais. Não se trata de adotar tecnologia com ingenuidade, mas com temor e responsabilidade. É possível, ainda que para muitos possa parecer difícil, utilizar recursos tecnológicos de maneira ética, consciente e pastoral quando formamos líderes que compreendam tanto os princípios da proteção de dados quanto os da mordomia cristã. A obediência não se contrapõe à graça; antes, dela é fruto.

Assim, o fechamento deste capítulo nos conduz a uma pergunta incômoda, porém essencial: haverá espaço, neste mundo viciado em dados, para um uso da tecnologia que glorifique a Cristo e edifique a comunhão dos santos?

Talvez, se a igreja se lembrar de que a ética cristã não é periférica, mas central; que a privacidade é extensão da dignidade; e que o cuidado com os dados é cuidado com as pessoas,  então podemos dizer que há esperança. Porque onde abundou o pecado, superabundou a graça (\gls{rm} 5.20). E a graça nos ensina a viver de modo sensato, justo e piedoso também na era digital lembrando do que diz \gls{tt} 2.11-13.

Para transformar essa pergunta reflexiva em ações práticas e trazer soluções ao desafio proposto, o próximo capítulo apresentará diretrizes que ajudem igrejas e lideranças a viver a comunhão cristã com integridade também no ambiente digital, refletindo o evangelho na forma como tratamos dados, pessoas e relações mediadas por tecnologia.
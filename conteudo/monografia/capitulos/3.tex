\newpage
\chapter{Propostas teológicas e éticas para preservar a comunhão em contextos digitais}

\begin{citacao}
A possibilidade de armazenamento de informações... faz com que literalmente não reste mais espaço para alguém se esconder ou ter qualquer privacidade\cite[p. 165]{schaeffer2002}
\end{citacao}

\section{Boas práticas com dados}
\subsection{Coleta, armazenamento e acesso}

\subsection{Uso ético e pastoral dos dados}

\subsection{Consentimento e transparência}

\subsection{Administração e segurança}

\subsection{Políticas de privacidade e governança}

\section{Software livre e comunhão}
\subsection{Ética do cuidado}

\subsection{Alternativas auditáveis}

\subsection{Tecnologia e teologia}

\section{Formação eclesiástica digital}
\subsection{Capacitação ética}

\subsection{Liturgia e cultura digital}

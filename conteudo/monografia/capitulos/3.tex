\chapter{Diretrizes éticas e pastorais para o cuidado com dados na igreja}
\section{Entre o alerta profético e a responsabilidade atual}

\subsection{A visão de Schaeffer sobre privacidade e controle}

\begin{citacao}
    A possibilidade de armazenamento de informações... faz com que literalmente não reste mais espaço para alguém se esconder ou ter qualquer privacidade\cite[p.~165]{schaeffer2002}
\end{citacao}

A advertência de Francis Schaeffer sobre o avanço tecnológico e suas implicações para a liberdade e a privacidade humanas é, sem dúvida, notavelmente antecipatória. Contudo, sua análise refletia uma visão inicial da tecnologia, limitada em parte por sua formação teológica e pela ausência de uma abordagem técnica mais aprofundada. Escrevendo ainda em 1976, Schaeffer alertava para o risco de uma sociedade cada vez mais controlada pela tecnologia, na qual o vácuo moral deixado pela rejeição de absolutos cristãos seria preenchido por estruturas de controle cada vez mais invasivas. Ele percebia, com sensibilidade profética, que o armazenamento crescente de informações pessoais poderia comprometer severamente a possibilidade de muitos indivíduos preservarem sua intimidade e aquilo que consideram privado.

À luz dos avanços ocorridos nas décadas seguintes, especialmente nos campos da proteção de dados e da engenharia de software, percebe-se que o armazenamento de informações, por si só, não é sinônimo de violação de privacidade ou motivo para se dizer que não há espaço para a privacidade. O ponto central da discussão está em como os dados são armazenados, quem os acessa, de que forma são tratados, com qual finalidade são utilizados, e se há mecanismos adequados de controle e transparência. Nesse sentido, privacidade não é apenas o direito de restringir o acesso ao que é particular, nem apenas a possibilidade de manter algo em segredo. Trata-se, sobretudo, do direito à autodeterminação informacional, ou seja, da capacidade de decidir sobre como seus próprios dados serão utilizados.

\subsection{Do medo à responsabilidade: o que mudou desde 1976}

Desde os tempos de Schaeffer, o debate sobre privacidade digital foi bastante ampliado, em especial nos últimos anos, impulsionado pelo crescimento de plataformas online, redes sociais, dispositivos conectados e a popularização de soluções de inteligência artificial. Muitas empresas de tecnologia passaram a exercer um papel central na coleta e no tratamento de dados pessoais. O problema, no entanto, não reside apenas na existência dessas ferramentas, mas na lógica que orienta seus desenvolvedores. Em muitos casos, os modelos de negócio são voltados à maximização do lucro por meio da exploração de informações, sem que haja o devido respeito à vontade ou ao conhecimento do usuário. Essa situação é agravada pelo fato de que grande parte da população, desinformada quanto a seus direitos digitais, aceita os termos impostos pelas tecnologias de massa sem questionamentos. Em geral, essas pessoas sequer têm ciência do que aceitam, pois muitas vezes veem a tecnologia apenas como um meio para alcançar uma finalidade e não se interessam pelos processos que a sustentam.

Apesar desse cenário, os dias atuais oferecem caminhos alternativos. O desenvolvimento de legislações como a \gls{gdpr} e a \gls{lgpd}, a expansão de ferramentas digitais voltadas à privacidade e o surgimento de movimentos que defendem a soberania tecnológica\footnote{Soberania tecnológica refere-se à capacidade de indivíduos, organizações ou Estados de controlar autonomamente os sistemas digitais dos quais dependem, inclusive em termos políticos, econômicos e culturais. Com base na formulação de \citeonline{pohle_soberania_2020}.} revelam que é possível e necessário exercer responsabilidade ética no uso das tecnologias. A privacidade tornou-se um valor tangível, cuja efetivação depende de decisões pessoais, institucionais e comunitárias, sempre relacionadas a princípios culturais e sociais mais amplos.

Para a igreja, não é coerente adotar uma postura de negação diante dos problemas relacionados à privacidade e à segurança digital, nem se eximir do debate como se essas questões fossem irrelevantes para a vida cristã. Ao contrário, esses desafios interpelam a ética cristã e exigem discernimento e responsabilidade. Isso implica realizar escolhas conscientes sobre as ferramentas utilizadas, refletir cuidadosamente sobre a forma de coleta e uso de dados e, sobretudo, dedicar-se à formação ética dos membros de maneira integral. Nesse contexto, a privacidade não deve ser tratada apenas como uma questão técnica ou jurídica. Ela é uma dimensão pastoral do cuidado, uma expressão concreta do amor ao próximo em um mundo em que a exposição constante se tornou a norma.

\subsection{Aprendizados de Atos 2 sobre corresponsabilidade}

O relato de Atos 2.42-47, analisado no primeiro capítulo, apresenta uma igreja viva e ativa, unida pela fé e marcada por relações de corresponsabilidade. A perseverança na doutrina dos apóstolos, a comunhão, o partir do pão e as orações eram expressões de uma comunidade comprometida com Deus e com o próximo. Cada membro assumia uma parte ativa na vida da igreja, inclusive na partilha de bens e no cuidado com os necessitados, evidenciando que a responsabilidade cristã não estava restrita à liderança institucional.

Essa dinâmica permanece plenamente relevante nos dias atuais. Aplicar esses princípios aos desafios contemporâneos da era digital, como os apresentados até aqui, permite afirmar que o cuidado com os dados também deve ser compartilhado por todos os que integram a comunidade de fé. Pastores e presbíteros exercem um papel de grande relevância e, em muitos casos, são o ponto de partida na orientação ética e na definição de diretrizes, pois suas condutas servem de referência para os demais membros quanto à postura individual. Líderes administrativos e voluntários, inclusive nas sociedades internas, colaboram de forma prática na proteção e manutenção das informações, promovendo uma cultura de responsabilidade e transformação. Os membros, por sua vez, devem ser capacitados para agir com sabedoria e discernimento em todas as situações que envolvam o manuseio de dados e a comunicação no contexto da igreja.

Essa compreensão pastoral cooperativa, marcada pela corresponsabilidade, fortalece a unidade do corpo e promove uma cultura comunitária de cuidado. Ao reconhecer que não apenas os líderes, mas todos são chamados a participar ativamente da vida da igreja, inclusive no uso ético das tecnologias, cria-se o ambiente propício para a adoção consciente de boas práticas, conforme será apresentado na seção seguinte.

\section{Boas práticas com dados}

\subsection{Políticas internas e cultura de governança}

\subsubsection{Ter uma política aprovada pelo conselho da igreja}

A elaboração de uma política de privacidade e proteção de dados, adequada ao contexto eclesiástico, deve ocorrer com a aprovação do conselho da igreja, órgão responsável pela governança institucional no caso de igrejas com liderança conciliar e representativa como a \gls{ipb}. Essa política deve estar alinhada aos princípios da \gls{lgpd}, assegurando o tratamento ético, transparente e responsável das informações pessoais e sensíveis dos membros e frequentadores. Sua formulação não apenas garante conformidade legal, mas também expressa o compromisso da igreja com a dignidade humana e com o cuidado pastoral em ambientes digitais. Exemplos relevantes de políticas desse tipo podem ser observados na \citeonline{ippinheiros2022} e na \citeonline{jardimdeoracao2025}.

Uma vez estabelecida, a política deve ser formalmente documentada, mantida atualizada conforme mudanças legais e tecnológicas, e amplamente comunicada aos líderes e voluntários da igreja. É essencial que o conteúdo esteja disponível em formatos físico e digital, preferencialmente divulgado em canais institucionais, como sites e murais internos. Recomenda-se também investir em ações formativas contínuas, por meio de treinamentos e campanhas de conscientização, com vistas a consolidar uma cultura organizacional sensível às implicações éticas e jurídicas do uso de dados, como será abordado em seção específica sobre este assunto mais à frente.

\subsubsection{Quem administra os sistemas e como isso é auditado?}

A boa governança digital requer definição clara sobre os responsáveis pela administração dos sistemas que armazenam, processam e transmitem dados sensíveis. Recomenda-se a nomeação de um encarregado de proteção de dados\footnote{Segundo a \gls{lgpd}, o encarregado é \textit{``pessoa indicada pelo controlador e operador para atuar como canal de comunicação entre o controlador, os titulares dos dados e a \gls{anpd}''} \cite[art.~5º, inciso~VIII]{lgpd2018}}, também conhecido como \gls{dpo}, conforme a \gls{gdpr}. Esse agente pode ser um membro capacitado da própria igreja ou um profissional externo contratado para essa finalidade.

Compete ao encarregado, também conhecido como \gls{dpo}, atuar como elo entre o controlador, os titulares e a \gls{anpd}, conforme dispõe a \gls{lgpd} \cite[art.~41]{lgpd2018}. Suas atribuições incluem: a) aceitar reclamações e comunicações dos titulares, prestar esclarecimentos e adotar providências; b) receber comunicações da autoridade nacional e tomar as medidas necessárias; c) orientar os funcionários e voluntários da igreja quanto às boas práticas em proteção de dados; e d) executar as demais atribuições determinadas pela liderança institucional ou por normas complementares.

O encarregado também pode atuar em frentes estratégicas, como: e) realizar o mapeamento e a documentação dos fluxos de dados; f) coordenar auditorias internas de conformidade; g) manter canais de transparência acessíveis aos titulares; e h) promover ações educativas e de conscientização na comunidade.

Conforme ressalta \citeonline[p.~19]{machado2020}, é função do \gls{dpo} \textit{``buscar meios de manter um constante treinamento aos colaboradores e líderes da igreja e disponibilizar as regras de privacidade e tratamento de dados a todos''}.

Entre suas atribuições específicas, destacam-se: a) o mapeamento de dados, identificando quais dados pessoais são coletados, onde são armazenados, quem tem acesso e por quanto tempo são mantidos, garantindo a rastreabilidade e o propósito legítimo do uso \cite[art.~6]{lgpd2018}; b) a realização de auditorias periódicas, que revisam práticas e sistemas para assegurar a conformidade com a legislação e corrigir falhas \cite[art.~50]{lgpd2018}; c) o funcionamento como canal de comunicação com a \gls{anpd}, inclusive para notificação de incidentes e orientação normativa \cite[art.~41]{lgpd2018}; d) a promoção de ações de treinamento e conscientização, voltadas a líderes e membros da igreja, com vistas ao fortalecimento de uma cultura de proteção de dados \cite[art.~50]{lgpd2018}; e) a garantia de transparência com os titulares, assegurando canais acessíveis para exercício de direitos como acesso, correção e exclusão de dados \cite[arts.~9º e 18º]{lgpd2018}.

A \gls{gdpr} também trata do papel do \gls{dpo} e da importância das auditorias de impacto, recomenda que a auditoria inclua uma descrição clara das operações de tratamento de dados, a justificativa para sua necessidade e proporcionalidade, uma análise dos riscos aos direitos dos titulares e as medidas previstas para mitigação desses riscos, incluindo garantias e mecanismos de conformidade com o regulamento. \cite[art.~35]{gdpr2016}

\subsection{Coleta, finalidade e consentimento}

As diretrizes internas de proteção de dados, quando bem estabelecidas, devem orientar não apenas a estrutura de governança, mas também os procedimentos práticos da organização, especialmente no que diz respeito à coleta e ao tratamento de informações pessoais. No contexto das igrejas, essa coleta ocorre com frequência por meio de fichas de membresia, formulários de inscrição em cursos, eventos e outras atividades eclesiásticas. Dados como nome, endereço, telefone, estado civil, histórico de participação e aspectos relacionados à fé são regularmente solicitados. Quando revelam convicções religiosas, tais informações passam a ser classificadas como dados pessoais sensíveis, conforme define o artigo 5º, inciso II, da \gls{lgpd}.

Diante disso, é fundamental que esses processos estejam embasados em princípios éticos e legais que assegurem o respeito à privacidade e à dignidade dos membros. A legislação brasileira impõe cuidados adicionais, como previsto nos artigos 7º e 11 da \gls{lgpd}, que condicionam o tratamento de dados sensíveis à obtenção de consentimento específico e destacado do titular\footnote{Segundo a \gls{lgpd}, o titular, ou pessoa natural, é \textit{``a quem se referem os dados pessoais que são objeto de tratamento''} \cite[art.~5º, inciso~V]{lgpd2018}.}, além da adoção de medidas técnicas e administrativas para sua adequada proteção.

Entre os princípios basilares da \gls{lgpd}, destaca-se o da minimização da coleta de dados. Recomenda-se evitar a solicitação de informações excessivas ou não essenciais, como CPF, RG ou número de telefone celular, quando não forem estritamente necessárias para o propósito informado. Sempre que houver coleta, a finalidade deve estar claramente definida e devidamente informada ao titular.

A própria \gls{lgpd} define tratamento de dados como:

\begin{citacao}
    Toda operação realizada com dados pessoais, como as que se referem à coleta, produção, recepção, classificação, utilização, acesso, reprodução, transmissão, distribuição, processamento, arquivamento, armazenamento, eliminação, avaliação ou controle da informação, modificação, comunicação, transferência, difusão ou extração.\cite[Cap.~I, art.~5º, X]{lgpd2018}
\end{citacao}

Essa amplitude do conceito evidencia a necessidade de atenção redobrada por parte das igrejas ao lidarem com qualquer aspecto relacionado ao manuseio de dados pessoais.

Para que o consentimento do titular seja considerado válido, não basta a simples marcação de uma caixa de seleção em um formulário. A \gls{lgpd} exige que essa manifestação de vontade seja livre, informada, específica e inequívoca\footnote{Conforme \citeonline[Cap.~II, art.~8º]{lgpd2018} \textit{``consentimento previsto no inciso I do art. 7º desta Lei deverá ser fornecido por escrito ou por outro meio que demonstre a manifestação de vontade do titular.''}}. O titular deve compreender claramente quais dados estão sendo solicitados, com que finalidade serão utilizados e quais são seus direitos. Além disso, o consentimento deve ser registrado de forma que permita sua posterior comprovação, caso necessário.

À luz dessas exigências, é altamente recomendável que as igrejas revisem suas fichas de cadastro, formulários e termos de consentimento, assegurando que contenham cláusulas objetivas quanto à finalidade da coleta, ao uso pretendido das informações e aos direitos dos titulares, como acesso, atualização e exclusão. Conforme alertam \citeonline{souzaMicheletti2024}, a ausência de critérios claros de finalidade, adequação e necessidade pode acarretar não apenas responsabilização civil da organização religiosa, mas também prejuízos à integridade emocional e espiritual de seus membros, pois pode conter questões íntimas.

\subsection{Armazenamento e acesso seguro}

A proteção dos dados pessoais no contexto eclesiástico requer a implementação de medidas técnicas e administrativas que possam minimamente garantir armazenamento e controle de acesso adequado às informações coletadas, respeitando a privacidade dos dados.

\subsubsection{Uso de senhas}

Senhas são pessoais e intransferíveis. O uso de senhas fortes é uma medida importante para proteger o acesso a sistemas e informações sensíveis. Recomenda-se a criação de senhas com, no mínimo, dez caracteres, combinando letras maiúsculas e minúsculas, números e símbolos. Deve-se evitar o uso de senhas padrão ou facilmente previsíveis, como datas de nascimento ou sequências numéricas simples. A utilização de senhas distintas para diferentes sistemas também é uma prática recomendada para ampliar a segurança dos dados armazenados. Sempre que possível, o compartilhamento de senhas deve ser evitado. Entretanto, quando esse compartilhamento se torna inevitável, é fundamental garantir que essas informações não sejam salvas em arquivos de texto simples, anotadas em papéis ou enviadas por e-mail ou mensagens instantâneas. Há soluções específicas para realizar o envio de senhas de forma que seja possível acesso único à senha e quem a receber possa salvá-la em algum local apropriado.

Para maior segurança, a utilização de um gerenciador de senhas\footnote{Segundo a Wikipédia: \textit{``Um gerenciador de senha é um programa que é usado para armazenar uma grande quantidade de nomes/senhas. O banco de dados onde esta informação é armazenada é criptografado usando uma única chave (senha mestre ou \foreignlanguage{english}{master password} em inglês), para que o usuário apenas tenha de memorizar uma senha para acesso a todas as outras. Isso facilita a administração de senhas e incentiva os usuários a escolherem chaves complexas sem medo de não ser capazes de lembrá-las mais tarde.''}\cite{wikipediaGerenciadorSenha}} confiável é altamente recomendada, pois essas ferramentas oferecem armazenamento criptografado e facilitam a criação e o uso de credenciais complexas sem comprometer a praticidade, seja para uso pessoal ou por mais de uma pessoa.

É importante também que senhas não sejam armazenadas no navegador de internet por meio do recurso de salvamento automático. Quando uma senha é salva no navegador, ela pode ser acessada por qualquer pessoa que utilize aquele dispositivo, comprometendo os princípios básicos de segurança e privacidade. Outro bom conselho sobre gestores de senhas é optar por soluções que não sejam gerenciadas por terceiros para evitar vazamento de dados\footnote{Vazamento de dados da LatsPass: \cite{goodin_lastpass_2022} \cite{vinton_lastpass_2015}}. Nesse sentido, recomenda-se, por exemplo, o uso do KeePassXC\footnote{\url{https://keepassxc.org/}} para computadores em geral, e do KeePassDX\footnote{\url{https://www.keepassdx.com/}} para dispositivos Android.

\subsubsection{Direito ao acesso aos próprios dados}

A \gls{lgpd} assegura aos titulares dos dados o direito de acessar, corrigir, portar e excluir suas informações pessoais\footnote{\textit{``Art. 18. O titular dos dados pessoais tem direito a obter do controlador, em relação aos dados do titular por ele tratados, a qualquer momento e mediante requisição''} \cite[Cap.~III, art.~18]{lgpd2018}.}. Esses direitos incluem, por exemplo, a confirmação da existência de tratamento, o acesso aos dados, a correção de informações incompletas, inexatas ou desatualizadas, e a eliminação dos dados pessoais tratados com base no consentimento do titular.

As igrejas, na condição de controladoras de dados pessoais\footnote{Segundo a \gls{lgpd}, controlador é \textit{``pessoa natural ou jurídica, de direito público ou privado, a quem competem as decisões referentes ao tratamento de dados pessoais''} \cite[art.~5º, inciso~VI]{lgpd2018}.}, devem estabelecer procedimentos claros e acessíveis para que os membros possam exercer esses direitos. Isso inclui a designação de responsáveis pelo atendimento às solicitações dos titulares, bem como a criação de canais de comunicação, garantindo o cumprimento dos prazos legais e a conformidade com a legislação vigente.

\subsection{Privacidade em cultos, reuniões e transmissões}

A era digital transformou profundamente as práticas eclesiásticas, tornando cultos, reuniões e outras expressões de comunhão acessíveis por meios eletrônicos. No entanto, essa exposição traz implicações éticas que exigem atenção, sobretudo quanto ao direito à privacidade dos participantes e à preservação do espaço sagrado como ambiente de confiança.

\subsubsection{Confidencialidade em reuniões}

Reuniões pastorais, conciliares e outras formas de aconselhamento espiritual muitas vezes envolvem relatos íntimos e decisões sensíveis. É dever da liderança zelar pelo sigilo, mesmo quando tais encontros ocorrem por videoconferência. Como afirma o Código de Disciplina da Igreja Presbiteriana do Brasil, a disciplina eclesiástica deve ser exercida com prudência, discrição e caridade, promovendo o bem do povo de Deus e a honra de Cristo. A gravação ou retransmissão dessas reuniões deve ser evitada, salvo em casos excepcionais, com aviso prévio e consentimento dos participantes.

\subsubsection{Conselhos para gravações}

\subsubsubsection{Avisos sobre gravação de cultos}

Imagens que permitam a identificação de indivíduos são consideradas dados pessoais. Quando essas imagens revelam, ainda que de forma indireta, a filiação religiosa ou a participação em práticas de fé, passam a ser classificadas como dados pessoais sensíveis, conforme define o artigo 5º, inciso II, da \gls{lgpd}. O contexto eclesiástico, por envolver aspectos da intimidade, convicções e pertencimento religioso, exige cuidado redobrado quanto à captação e divulgação de imagens.

Embora as igrejas sejam locais de acesso público, acolhem pessoas de diferentes origens, histórias e condições, inclusive aquelas que podem estar em situação de vulnerabilidade ou sob algum tipo de risco social. Casos noticiados pela imprensa evidenciam que a exposição pública pode gerar constrangimentos e consequências indesejadas tanto para os envolvidos quanto para a própria igreja, sobretudo quando não há consentimento claro e informado\cite{silva2023}. O descumprimento das diretrizes legais pode acarretar sanções administrativas, incluindo multas expressivas.

Uma boa prática é anexar à ficha de membresia um termo de consentimento para uso de imagem. Nesse documento, a igreja deve informar que os cultos e demais eventos poderão ser gravados, transmitidos por plataformas digitais e arquivados para fins de registro histórico. Esse consentimento deve ser claro, conforme estabelecido pela \gls{lgpd}.

Segundo orienta Machado:

\begin{citacao}
    O ideal é não divulgar as imagens de pessoas que estejam participando da celebração sem a autorização escrita de cada uma delas. E se a igreja optar por divulgar imagens das pessoas presentes na celebração, que seja de modo a não identificá-las, utilizando-se de imagens embaçadas e de costas.\cite[p.~12]{machado2020}
\end{citacao}

No caso de visitantes, a transparência deve ser ainda mais zelosa, pois são pessoas que não preencheram previamente termo de consentimento para uso de imagem. Recomenda-se que a igreja informe antecipadamente sobre a gravação por meio de comunicados verbais no início da celebração e durante os avisos, bem como por sinalizações visuais em pontos estratégicos do templo e notas informativas nos boletins ou formulários de inscrição, especialmente em eventos que exijam registro prévio. Nesse contexto, o excesso de notificações não representa um problema, mas sim uma salvaguarda ética e jurídica.

Sempre que possível, é aconselhável oferecer aos visitantes uma ficha de consentimento específica, permitindo-lhes manifestar de forma clara sua vontade quanto ao uso de sua imagem. Tal medida assegura o respeito à liberdade individual e protege especialmente aqueles que, por razões pessoais, legais ou profissionais, optam por não aparecer em registros públicos, preservando assim seu direito ao anonimato.

Como testemunho pessoal, em certa ocasião convidei uma pessoa para participar de um culto. Durante o momento de oração, esta pessoa foi fotografada, e a imagem, de fato, estava muito bonita. No entanto, a foto foi publicada nas redes sociais da igreja sem que houvesse consentimento prévio. Dias depois, ao visualizar a publicação, essa pessoa reconheceu a beleza do registro, mas relatou ter se sentido profundamente invadida em sua privacidade.

\subsubsubsection{Espaços sem filmagem, práticas respeitosas online}

É aconselhável que a igreja estabeleça espaços previamente sinalizados onde não haja captação de imagem. Essa iniciativa oferece uma alternativa para quem desejar participar da celebração presencialmente, mas preferir não aparecer em registros audiovisuais por razões pessoais, legais, políticas, sociais ou profissionais.

Essas áreas devem ser respeitadas pela equipe de mídia e comunicadas de forma visível, de modo que qualquer pessoa possa exercer sua liberdade de escolha com clareza. Tal medida contribui para o ambiente acolhedor que se espera no contexto eclesiástico, especialmente diante da pluralidade de situações representadas entre os presentes.

Também é importante que as equipes responsáveis por transmissões e registros recebam orientações específicas quanto à ética da imagem, evitando planos fechados que individualizem crianças, idosos ou pessoas em situações de vulnerabilidade. Quando for necessário registrar momentos que envolvam tais grupos, o uso de ângulos amplos, desfoque ou enquadramentos neutros deve ser preferido.

No ambiente digital, esse mesmo princípio de cuidado se aplica à edição e à divulgação dos conteúdos. Conforme exemplificado nas políticas de igrejas como \citeonline{ippinheiros2022} e a \citeonline{jardimdeoracao2025}, e informado pela \gls{lgpd}, a imagem vinculada à prática de fé constitui dado sensível e, portanto, deve ser tratada com especial cautela, respeitando os direitos do titular.

Recomenda-se que a igreja mantenha um canal acessível para solicitações de exclusão ou anonimização de imagens publicadas. Essa abertura ao diálogo fortalece a confiança da comunidade, expressa o compromisso pastoral com a dignidade de cada pessoa e contribui para a construção de uma cultura digital respeitosa e coerente com os valores cristãos.

\subsubsection{Práticas respeitosas online}

No ambiente digital, é essencial que a igreja atue com discernimento e zelo, especialmente diante dos riscos de exposição indevida e de comunicações impróprias que possam comprometer a comunhão e a edificação mútua. Isso inclui a moderação de comentários durante transmissões ao vivo, o cuidado com a divulgação de informações pessoais em chats, redes sociais ou grupos de mensagens, bem como a orientação aos membros quanto à forma adequada de compartilhar trechos de cultos e eventos. Um exemplo crítico a ser evitado é a divulgação de detalhes sobre campos missionários, que, embora motivada por boas intenções, pode expor dados sensíveis e colocar em risco a integridade e a segurança de pessoas envolvidas.

O Código de Disciplina da \gls{ipb} estabelece:

\begin{citacao}
    \textbf{Art. 4º}. Falta é tudo que, na doutrina e prática dos membros e concílios da igreja, não esteja de conformidade com os ensinamentos da Sagrada Escritura, ou transgrida e prejudique a paz, a unidade, a pureza, a ordem e a boa administração da comunidade cristã.

    \textbf{Parágrafo único}. Nenhum tribunal eclesiástico poderá considerar como falta, ou admitir como matéria de acusação aquilo que não possa ser provado como tal pela Escritura, segundo a interpretação dos Símbolos da Igreja \cite[art.~4]{ipb_codigodisciplina}
\end{citacao} 

A exposição indevida de dados sensíveis, como imagens sem consentimento, falas particulares ou informações de contato, além de violar a legislação civil, o código de disciplina da \gls{ipb} e princípios confessionais no que diz respeito à comunhão dos santos, em primeiro lugar infringe os princípios bíblicos sobre o zelo, a responsabilidade e a preservação da comunhão cristã. A Escritura ensina que devemos ser submissos às autoridades constituídas (\gls{rm} 13.1-7; \gls{1pe} 2.13-17), desde que estas não contrariem os mandamentos de Deus. Assim, quando a legislação vigente protege a dignidade e a privacidade dos indivíduos, cumpri-la é expressão concreta de obediência ao ensino bíblico e de amor ao próximo.

\subsection{Dependência digital e riscos à missão eclesiástica}

A crescente digitalização das atividades eclesiásticas trouxe consigo benefícios inegáveis, como maior alcance da mensagem e eficiência na gestão. Contudo, essa transição tem ocorrido, muitas vezes, com uma dependência quase cega de soluções tecnológicas proprietárias, sem avaliação crítica dos riscos envolvidos. Essa dependência pode comprometer a missão da igreja, especialmente quando os dados dos membros, das finanças e das ações pastorais são terceirizados para plataformas sobre as quais a comunidade eclesiástica não possui controle efetivo.

A falta de soberania sobre os sistemas utilizados impede que a igreja tenha garantias sobre práticas básicas de segurança da informação, como a realização de backups regulares, o controle de acessos e a integridade dos dados. Além disso, não raramente, os termos de uso dessas plataformas permitem o acesso a dados sensíveis por terceiros, colocando em risco a privacidade e a confiança dos fiéis.

O risco de delegar a gestão dos dados pastorais a empresas terceirizadas foi evidenciado por casos como o da InChurch\cite{almeida_inchurch_2024}, já citado no capítulo anterior, que expôs milhares de registros de fiéis devido a falhas técnicas. A situação reforça a urgência de repensar a dependência tecnológica da igreja diante de sua responsabilidade ética e espiritual.

Tal vulnerabilidade não é apenas uma falha técnica, mas uma questão de mordomia cristã. A igreja é chamada a zelar pelo rebanho, o que inclui a proteção das informações que lhe são confiadas. Portanto, torna-se urgente que as lideranças eclesiásticas reflitam sobre os riscos da dependência digital e busquem alternativas que respeitem a soberania tecnológica e a responsabilidade pastoral diante de Deus e das pessoas.

\subsection{Panorama da conformidade digital na IPB}

Com o objetivo de avaliar a presença e a maturidade digital de igrejas da \gls{ipb} frente às exigências da \gls{lgpd}, foi realizada uma análise técnica sobre os dados públicos disponíveis no site \url{https://www.icalvinus.app/consulta_ipb/anuario_igrejas.html}, que reúne informações extraídas do Anuário da \gls{ipb}. O levantamento considerou os domínios informados por 2.935 igrejas cadastradas e verificou aspectos básicos de conformidade digital e a presença de políticas de privacidade.

A coleta foi realizada manualmente, com apoio de ferramentas automatizadas de varredura e inspeção de conteúdo. O processamento dos dados foi concluído em 20 de junho de 2025. O código utilizado no processo de análise está disponível publicamente em \url{https://github.com/vitormattos/webscraping-anuario-igrejas-ipb}, com o objetivo de garantir transparência e reprodutibilidade da metodologia aplicada.

A seguir, apresenta-se a tabela com os principais achados da amostra.

\begin{table}[H]
\centering
\begin{threeparttable}
\caption{Panorama da adequação à \gls{lgpd} em sites de igrejas da \gls{ipb}}
\rowcolors{2}{gray!10}{white}
\begin{tabularx}{\textwidth}{>{\raggedright\arraybackslash}X c}
\toprule
\textbf{Indicador} & \textbf{Quantidade} \\
\midrule
Total de igrejas \gls{ipb} analisadas & 2935 \\
Igrejas com site informado no cadastro\tnote{a} & 640 \\
Endereços que não são redes sociais\tnote{b} & 532 \\
Endereços válidos e ativos\tnote{c} & 219 \\
Sites com qualquer menção à \gls{lgpd} & 22 \\
Com aviso de uso de cookies & 16 \\
Com Política de Privacidade publicada & 15 \\
Com e-mail genérico indicando canal de DPO\tnote{d} & 12 \\
Com formulário específico para solicitação de direitos do titular\tnote{e} & 5 \\
Com \gls{dpo} identificado\tnote{f} & 2 \\
\bottomrule
\end{tabularx}
\begin{tablenotes}
\footnotesize
\item[a] Embora 640 igrejas tenham informado endereços de páginas web, muitos desses estão completamente inválidos, com textos que não dizem respeito a uma página web.
\item[b] Sites com endereços de páginas web válidas do ponto de vista técnico, mas que não necessariamente levam a sites ativos.
\item[c] Sites que exibem algum conteúdo, mas não necessariamente estão em conformidade com a LGPD. Embora apresentem páginas da igreja, muitas delas contêm falhas de segurança, mensagens de erro ou até mesmo indícios de comprometimento parcial por terceiros.
\item[d] Por exemplo: adm@paginadaigreja.org.br, contato@paginadaigreja.org.br e não endereços de endereço eletrônico de pessoas.
\item[e] Refere-se à presença de um formulário eletrônico próprio que permite ao titular do dado pessoal exercer seus direitos previstos na \gls{lgpd}, como acesso, correção, exclusão ou portabilidade de dados.
\item[f] O \gls{dpo}, ou Encarregado de Dados como definido na \gls{lgpd}, foi identificado nominalmente, com endereço eletrônico para contato.
\end{tablenotes}
\end{threeparttable}
\end{table}

Além dos dados quantitativos apresentados, foi possível identificar que pelo menos 8 dos sites analisados foram desenvolvidos pela empresa \mbox{InChurch}, já mencionada nesta monografia em razão do episódio de exposição de mais de 9,2 milhões de arquivos, que resultou no vazamento que resultou no vazamento de quase um milhão de informações sensíveis de usuários\cite{almeida_inchurch_2024}.

Esses 8 sites fazem parte de um grupo maior de aproximadamente 50 páginas desenvolvidas principalmente por quatro empresas\footnote{\mbox{InChurch}, \mbox{Inovaki}, \mbox{Igreja Smart} e \mbox{Eklesia}} que, à época da análise, foram identificadas no Anuário de Igrejas da \gls{ipb} e não apresentavam os elementos mínimos exigidos pela \gls{lgpd}, como política de privacidade, identificação do encarregado de dados ou canal específico para o exercício dos direitos dos titulares.

Essa constatação aponta que, embora igrejas estejam em processo de digitalização de seus canais de comunicação, a contratação de serviços de terceiros sem critérios claros de conformidade pode comprometer a proteção de dados dos membros e fragilizar o testemunho ético da igreja no ambiente digital.

Dos dois sites que apresentaram todos os elementos mínimos de conformidade com a \gls{lgpd}, um pertence à Igreja Presbiteriana da Freguesia (RJ) e outro a Igreja Presbiteriana de Alphaville (SP). Ambos disponibilizam política de privacidade, canal de atendimento ao titular e identificação nominal de encarregado.

Ainda assim, é importante destacar que a simples nomeação de um \gls{dpo} no site da igreja não garante que esse profissional esteja registrado junto à \gls{anpd}, como orienta o Guia de Boas Práticas da própria autoridade nacional. A atualização desses dados, bem como sua comunicação formal à \gls{anpd}, é responsabilidade da administração eclesiástica e deve ser compreendida como parte do testemunho de obediência da igreja à legislação vigente.

\subsection{O invisível que sustenta o visível}

Ao adotar soluções proprietárias e terceirizadas, a igreja abre mão do controle sobre a tecnologia e a gestão dos dados, delegando isto a terceiros. Quando sistemas de gestão, comunicação e armazenamento são hospedados por empresas externas, onde a igreja não é capaz de certificar se backups estão sendo feitos, se backups estão criptografados, a comunidade passa a operar sob uma lógica de dependência. A cada serviço que se utiliza sem transparência ou autonomia, aumenta-se o número de intermediários que podem ter acesso às informações da igreja. Isso fragiliza a privacidade dos membros, compromete a segurança dos dados e ameaça a continuidade da missão em caso de mudanças nos termos de serviço, falhas técnicas ou mesmo ataques.

\subsubsection{Por que o software importa?}

A escolha do software utilizado pela igreja não se limita a uma mera decisão técnica ou operacional ou a questões de gosto. Trata-se também de uma questão ética, espiritual e pastoral. Optar por soluções que não oferecem transparência no tratamento de dados, que impõem restrições ao acesso ou que podem, em algum momento, contrariar os interesses da comunidade, implica submeter a missão eclesiástica a riscos que afetam diretamente a comunhão dos santos. Esses riscos envolvem desde a perda da privacidade e da integridade das informações até o alinhamento, muitas vezes involuntário, a interesses incompatíveis com os valores do Reino de Deus, como o uso ou a comercialização indevida de dados dos fiéis, os quais são sustentados financeiramente por meio dos dízimos e ofertas.

Da mesma forma como a igreja valoriza o estudo das Escrituras, tornando-as acessíveis, compreensíveis e abertas à investigação, ela deve também buscar compreender as tecnologias que sustentam sua missão ou, ao menos, assegurar que pessoas com conhecimento técnico possam fazê-lo. Ter acesso ao código-fonte, entender a lógica de funcionamento dos sistemas utilizados, adaptá-los às necessidades do ministério e compartilhá-los com outras comunidades são expressões práticas dos princípios cristãos de mordomia e mutualidade. Essa abordagem é viabilizada pelo uso do software livre, modelo que assegura liberdades fundamentais como usar, estudar, modificar e distribuir programas computacionais. Essas liberdades promovem a colaboração entre comunidades e fortalecem a soberania tecnológica da igreja.

Na história do cristianismo há um notório exemplo que ocorreu em meados do século VI, quando um monge irlandês chamado Finnian regressou de Roma portando uma rara e preciosa cópia da Bíblia. O manuscrito chamou a atenção de outro monge chamado Colmcille que, encantado com a Bíblia, começou a frequentemente entrar de forma secreta na igreja onde o exemplar da Bíblia estava e começou a copiar o livro com o intuito de disseminar o conteúdo sagrado em um \foreignlanguage{latin}{scriptorium} que administrava. Quando Finnian ficou sabendo, prontamente entrou com uma ação judicial junto ao Alto Rei Diarmaid, que julgou a favor de Finnian da posse exclusiva do exemplar original. Em sua defesa, Colmcille argumentou:

\begin{citacao}
    Livros são diferentes de outros bens móveis. Homens cultos como nós, que recebemos uma nova herança de conhecimento por meio de livros, têm a obrigação de disseminá-lo, copiando e distribuindo-os por toda parte. O conhecimento contido nos livros deve estar disponível para qualquer pessoa que queira lê-los [...] e é errado ocultar tal conhecimento ou tentar extinguir as coisas divinas que os livros contêm. \cite[p.~40-42]{owens_dorean_2021}\footnote{Tradução nossa}
\end{citacao}

Esse episódio, embora inserido em outro contexto histórico e não levando em consideração que Colmcille não teve sua argumentação acatada talvez por uma tentativa de impedir o avanço do cristianismo, a história ilustra o perigo de tratar recursos espirituais como propriedades exclusivas. Trancar o acesso a sistemas e limitar seu uso por meio de licenças proprietárias reproduz, na linguagem digital, a mesma lógica de controle que compromete a vocação da igreja de partilhar livremente o que recebeu pela graça.

Ainda que nem todos os pastores ou presbíteros possuam conhecimento ou formação técnica para acompanhar ou realizar diretamente tais tarefas, é fundamental que a igreja se mantenha aberta a esta necessidade de compreensão e interação com os meios tecnológicos e legais relacionados à segurança e privacidade de dados. Da mesma forma que nem todos dominam grego ou hebraico para realizar uma exegese detalhada das Escrituras, elaborar um estudo histórico rico em detalhes sobre o contexto dos textos bíblicos, análise de forma e outros elementos de estudo, mas ainda assim quem participa de um culto ou estudo se beneficia dos trabalhos feitos por teólogos partindo dos textos em idioma original, o acesso ao conhecimento técnico por membros capacitados amplia a autonomia e o discernimento da comunidade. Lutero, ao traduzir a Bíblia para a linguagem comum, libertou o povo da dependência exclusiva do clero. O mesmo princípio se aplica ao uso consciente da tecnologia.

Em contrapartida, o uso de software proprietário pode ser comparado a um culto realizado em língua desconhecida: cumpre parcialmente seu propósito estético e ritualístico, mas limita profundamente o entendimento e o envolvimento da comunidade. Quando a igreja investe recursos financeiros e humanos em soluções fechadas, corre o risco de perceber, apenas após a implementação, que essas ferramentas não atendem plenamente às suas necessidades. Nesses casos, a migração para outra solução se torna necessária; porém, em se tratando de softwares proprietários, esse processo é frequentemente complexo ou inviável, especialmente quando os dados estão armazenados em formatos fechados ou incompatíveis com outros sistemas. Além disso, por sua própria natureza, tais softwares não permitem auditorias completas, o que pode ocultar vulnerabilidades ou até mecanismos de coleta indevida de dados pessoais, comprometendo a conformidade com legislações de proteção de dados.

A adoção de soluções livres, por sua vez, facilita o cumprimento das exigências legais, garantindo que dados sensíveis sejam adequadamente tratados e protegidos, com processos transparentes e auditáveis. Ao utilizar software livre, a igreja conquista soberania tecnológica verdadeira, tornando-se de fato detentora das soluções que utiliza. Isso permite contar com membros capacitados, colaboração com outras igrejas ou suporte especializado para realizar as customizações necessárias, sem depender exclusivamente da empresa desenvolvedora. Dessa forma, a igreja mantém controle legal e técnico sobre seus dados e processos, beneficiando-se também da atuação de uma ampla comunidade global de desenvolvedores para manutenção, suporte, personalização e integração das soluções conforme as demandas específicas do ministério local.

\subsubsection{Ferramentas visíveis para uma missão invisível}

A missão da igreja no mundo digital exige discernimento não apenas quanto ao conteúdo que compartilha, mas também aos meios pelos quais realiza sua comunicação, gestão e cuidado pastoral. Se os instrumentos forem incoerentes com os valores cristãos, comprometem a integridade do testemunho. Por isso, a seleção criteriosa de ferramentas digitais não deve ser um ato técnico neutro, mas uma expressão de mordomia e fidelidade ao chamado ministerial.

Para além da adequação à \gls{lgpd}, a escolha de tecnologias deve levar em conta princípios como soberania sobre os dados, privacidade dos fiéis, transparência na gestão e sustentabilidade comunitária. Isso inclui evitar dependência de empresas que operam fora da jurisdição nacional\footnote{Isto reduz riscos relacionados à vazamento de dados, como o que legalmente ocorre por conta do \textit{\gls{cloudact}} que é a lei norte-americana que autoriza o governo dos Estados Unidos a acessar dados armazenados por empresas sob sua jurisdição, mesmo quando localizados em servidores fora do território americano\cite{harvard_cloudact_2018,brooklyn_cloudact_2020}.}, seja de forma direta ou por terceiros\footnote{A \gls{lgpd}, em seus artigos 33 a 35, estabelece critérios rigorosos para a transferência internacional de dados, exigindo garantias de proteção adequadas por parte de países estrangeiros, cláusulas contratuais específicas ou autorização da \gls{anpd}.}, e priorizar soluções livres que possam ser auditáveis e estejam alinhadas com a missão da igreja.

Soluções de tecnologia da informação e comunicação exigem pessoas preparadas, ou dispostas a aprender para implementá-las com sabedoria e responsabilidade. É igualmente necessário estabelecer rotinas mínimas de segurança digital, como controle de acesso, auditoria de registros, backups frequentes e revisão periódica dos termos de uso e consentimento.

Uma lista dessas soluções está disponível no \hyperref[apendice:solucoes]{Apêndice~\ref*{apendice:solucoes}}, onde elas estão organizadas por áreas de atuação, como gestão de documentos, comunicação pastoral, transmissões, finanças, assinatura digital e presença online. Cada solução está acompanhada de comentários sobre sua aplicabilidade. Todas foram selecionadas com base em critérios de conformidade legal e coerência com os valores cristãos. O objetivo é oferecer um ponto de partida confiável, alinhado aos valores éticos abordados nesta monografia, para comunidades que desejam amadurecer sua presença digital com responsabilidade e discernimento pastoral.

\section{Formação digital da igreja}

\subsection{Capacitação eclesiástica sobre privacidade}

Diante do exposto, torna-se evidente o impacto que o uso não refletido de meios tecnológicos pode causar na vida comunitária das igrejas. Isso exige uma reflexão séria e ação sobre o tratamento de dados pessoais, o uso de ferramentas digitais e os ambientes de informação. A proteção de dados não é apenas uma obrigação legal, mas também uma dimensão do cuidado pastoral. Considerando que as igrejas lidam diariamente com informações sensíveis de seus membros e visitantes, e diante do rigor da legislação quanto ao uso indevido desses dados, evidencia-se a urgência de uma formação interna que capacite lideranças e colaboradores a compreenderem os riscos envolvidos e a adotarem práticas responsáveis que se antecipem aos possíveis problemas.

Essa capacitação não exige que todos se tornem especialistas, mas pressupõe que a liderança identifique membros com perfil e interesse em atuar na igreja com foco no uso ético de sistemas digitais, na prevenção de falhas e na resposta a incidentes. Essa atuação deve incluir também a definição e revisão periódica da política de privacidade da igreja, a atualização desse documento conforme exigências legais, a avaliação de processos internos que envolvam dados sensíveis e a orientação em todos os momentos onde haja registro e o compartilhamento de informações, tanto de forma escrita quanto oral. Para que essa pessoa ou grupo possa cumprir adequadamente sua função, é essencial que tenha liberdade para propor e implementar ajustes nos processos da igreja em conformidade com a \gls{lgpd}. A presença de alguém responsável por essa área, conforme previsto na legislação brasileira, deve ser entendida como uma condição indispensável para que a igreja atue de maneira legal e fiel aos valores cristãos no tempo presente.

O custo financeiro para adequar-se a essa realidade é mínimo, quase desprezível, especialmente porque, em boa parte das igrejas, já existem pessoas interessadas em se qualificar ou que até mesmo possuem formação profissional na área e estão dispostas a contribuir voluntariamente com esse serviço.

\subsection{Discipulado digital e testemunho público}

A igreja também precisa reconhecer a importância de instruir sua comunidade sobre temas como o tratamento de dados sensíveis, o uso ético de ferramentas digitais, o comportamento nas redes sociais e o cuidado com informações que circulam em momentos da vida comunitária, como reuniões de oração, encontros de pequenos grupos, sociedades internas ou qualquer outra atividade que haja troca de informações. Promover uma cultura de consciência sobre segurança da informação e sobre como essas práticas comunicam valores ao mundo é parte do compromisso da igreja com o evangelho. Em tempos de crescente desconfiança institucional, demonstrar responsabilidade no ambiente digital é uma forma concreta e necessária de testemunhar a fé cristã.

Nesse contexto, o discipulado, que é uma ordenança de Cristo registrada em \gls{mt} 28.19, não pode ser negligenciado, e sua aplicação deve incluir também a formação digital dos membros. Utilizar-se de palestras constantes, instruções nas classes de ensino da igreja são oportunidades para ensinar o uso sábio dos meios digitais, o cuidado com a privacidade e a rejeição de práticas descuidadas ou invasivas. O cuidado com os dados não se limita à liderança ou à esfera jurídica, mas é expressão de uma fé coerente com a prática. O que se publica, compartilha e administra digitalmente também revela os valores que a igreja afirma crer e seguir.
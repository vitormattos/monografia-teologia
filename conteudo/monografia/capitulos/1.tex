\chapter{A Comunhão dos Santos de Gênesis a teologia contemporânea}

\section{Perspectivas bíblicas}

\subsection{No Antigo Testamento}

Antes da queda, essa comunhão perfeita era caracterizada pela harmonia entre Criador e criatura, refletida no fato de que "o SENHOR Deus andava pelo jardim" em relacionamento próximo com o homem (\gls{gn} 3.8), em um ambiente de plena transparência, onde "ambos estavam nus e não se envergonhavam" (\gls{gn} 2.25). A ordem dada diretamente por Deus (\gls{gn} 2.16-17) e a responsabilidade confiada ao homem de cuidar do Éden (\gls{gn} 2.15) evidenciam uma vida partilhada em amor, verdade e justiça, que foi declarada "muito boa" pelo próprio Criador (\gls{gn} 1.31). Esta condição original de santidade e unidade é reconhecida também pelos Cânones de Dort, que descrevem que:

\begin{citacao} no princípio, o homem foi criado à imagem de Deus, adornado em seu entendimento com o verdadeiro e salutar conhecimento de Deus e de todas as coisas espirituais, com a vontade e o coração retos, e os afetos puros; portanto, era o homem completamente santo.\cite{cdd}
\end{citacao}

Essa comunhão desimpedida expressava desde o princípio o chamado do ser humano à vida comunitária sob a orientação de Deus, apontando para aquilo que seria restaurado de maneira real em Cristo por meio da comunhão dos santos, embora ainda estejamos em um estado corruptível e em um processo contínuo de santificação \footnote{\gls{rm} 8.23 ou \gls{2co} 4.16}, e que alcançará sua plenitude na eternidade: um povo reunido em vínculo com Deus e entre si, crescendo no conhecimento do Senhor, até que esse conhecimento e essa unidade sejam plenamente consumados na glória futura.

Essa realidade é reafirmada com a constituição do povo de Israel. Em \gls{ex} 19.5-6, Deus chama os israelitas para serem um reino de sacerdotes e uma nação santa, estabelecendo a aliança como fundamento para uma vida comunitária regida pela justiça e pelo amor (\gls{lv} 19.18).

A Bíblia de Estudos de Genebra, comentando \gls{jz} 20.1, registra que \textit{pela primeira vez, Israel agiu unido, mas o propósito era guerrear contra seus irmãos''}\footnote{Trecho na \gls{naa}: \textit{como se fosse um só homem''}.} \cite[p. 443]{bibliaGenebra}, evidenciando que a unidade, embora visível, pode ser mal orientada quando desconectada dos princípios divinos.

O Salmo 133 enfatiza o valor da verdadeira comunhão entre os crentes. No versículo 1, Davi declara: \textit{``Oh! Como é bom e agradável viverem unidos os irmãos''} (\gls{naa}). A Bíblia de Estudos de Genebra observa que este cântico de romagem, embora possa aludir a irmãos biológicos, provavelmente se refere às famílias tribais e nacionais reunidas para adorar ao Senhor \cite[p. 1053]{bibliaGenebra}. Ainda segundo essa fonte, a união celebrada no salmo aponta para o relacionamento entre os crentes no contexto da adoração no monte Sião \cite[p. 1053]{bibliaGenebra}. Matthew Henry, ao comentar este salmo, destaca que, embora Davi tenha inicialmente em vista a unidade das tribos de Israel, o princípio enunciado se aplica à comunhão dos santos em todos os tempos, sendo esta a principal bênção exaltada no texto sagrado \cite[p. 1077]{mathewHenry}.

A literatura sapiencial reforça a importância do companheirismo na caminhada humana. Em \gls{ec} 4.9-12, destaca-se que \textit{``melhor é serem dois do que um''} (\gls{naa}), pois a cooperação proporciona apoio mútuo e maior resistência diante das adversidades.

A expectativa de uma unidade restaurada entre os povos de todas as nações já aparece nas visões proféticas. Isaías profetiza que \textit{os estrangeiros que se unirem ao Senhor''} (\gls{is} 56.6-7, \gls{naa}) seriam aceitos em sua casa de oração, e Miquéias profetiza que \textit{muitos povos subirão ao monte do SENHOR''} (\gls{mq} 4.1-2, \gls{naa}). Essas promessas não apenas anunciam a inclusão dos gentios, mas também apontam para a reconstrução da relação rompida no Éden, agora sendo restabelecida em Cristo, que reúne em si todos os que são chamados pelo nome do Senhor em uma única comunidade de fé.

A promessa da nova aliança fundamenta a consolidação de uma vida comunitária renovada. Em \gls{jr} 31.33-34, o profeta anuncia: \textit{``Não ensinará jamais cada um ao seu próximo, nem cada um ao seu irmão, dizendo: 'Conheça o SENHOR!' Porque todos me conhecerão''} (\gls{naa}). Ainda que, em Cristo, o conhecimento de Deus já tenha sido inaugurado entre o seu povo na igreja visível, o cultivo da comunhão mútua permanece essencial, pois é por meio do relacionamento entre os crentes que o conhecimento do Senhor é aprofundado e vivido na prática da fé.

\subsection{No Novo Testamento}

\item A igreja primitiva perseverava na comunhão (Atos 2:42)
\item ``Da multidão dos que criam, era um o coração e a alma'' (Atos 4:32)
\item Paulo descreve a comunhão como partilha no corpo e sangue de Cristo (1 Coríntios 10:16-17)
\item Unidade no Espírito, diversidade de membros, todos necessários (1 Coríntios 12:12-27)
\item A unidade e edificação da igreja como expressão da comunhão (Efésios 2:14-22; 4:1-6)
\item Participação ativa no evangelho como comunhão (Filipenses 1:5-7)
\item Vivência da comunhão com gratidão, ensino e amor (Colossenses 3:12-17)
\item A comunhão é vivida como edificação mútua e perseverança na fé (Hebreus 10:24-25)
\item A harmonia cristã como forma de comunhão (1 Pedro 3:8)
\item Comunhão com Deus e com os irmãos como marca da vida cristã (1 João 1:3-7)
\item A comunhão é escatológica: os que estão em Cristo vivem mesmo que morram (Apocalipse 1:17-18)


\section{Na igreja primitiva}

\subsection{Partilha e solidariedade entre os santos}

\subsection{Tertuliano e a expressão \textit{mater fidelium}}
\begin{citacao}
Desde os dias de Tertuliano, todos os cristãos chamavam a igreja não apenas de comunidade reunida (\textit{coetus}), mas também de mãe dos crentes (\textit{mater fidelium}). [...] A instituição da igreja, pelo menos de acordo com a confissão reformada, absolutamente não é um produto da comunidade de fé, mas uma obra do próprio Cristo. \cite[335]{bavinck2012}.
\end{citacao}

\subsection{A formulação do credo apostólico}

\begin{citacao}
Agostinho dá este conselho: ``que corrijam compassivamente o que podem; o que não podem, tolerem pacientemente, e com amor, deplorem e lamentem, até que Deus ou emende e corrija, ou, na colheita, arranque as cizânias e joeire as palhas.'' \cite[v. 4, p. 102]{calvino2022}
\end{citacao}

\section{Na Reforma Protestante}

\subsection{Comunhão como sustentação da igreja visível}
\textit{``A comunhão dos santos [...] exprime excelentemente a natureza da Igreja [...] que todos e quaisquer benefícios que Deus lhes confira, entre si, mutuamente, compartilhem''} \cite[v. 4, p. 75-76]{calvino2022}.

\begin{citacao}
``Por esta razão cremos na Igreja, que estejamos seguramente persuadidos de que somos seus membros. [...] nossa salvação se apoia em suportes seguros e sólidos, de sorte que, ainda quando seja abalada toda a máquina do orbe, ela própria não se mova e tombe por terra: primeiro, ela se sustém com a divina eleição [...]; então, de certo modo associada com a firmeza de Cristo, que não mais permitirá que seus fiéis sejam de si alijados [...]'' \cite[v. 4, p. 75-76]{calvino2022}
\end{citacao}

\begin{citacao}
``O abandono da Igreja é negação de Deus e de Cristo [...] Porque, enquanto nos esforçamos, quanto está em nós, por fomentar a ruína da verdade de Deus, somos dignos de que ele dardeje seus raios com todo o ímpeto de sua ira, a fim de fazer-nos em pedaços.'' \cite[v. 4, p. 92]{calvino2022}
\end{citacao}

\textit{``não renunciemos à comunhão da Igreja, nem perturbemos nela a paz e a disciplina devidamente exercitada''} \cite[v. 4, p. 95]{calvino2022}

\subsection{A importância da igreja visível}

\subsection{Nos símbolos de fé}

\subsubsection{A Confissão de Fé de Westminster} 
\begin{citacao}
Todos os santos que pelo seu Espírito e pela fé estão unidos a Jesus Cristo, seu Cabeça, têm com Ele comunhão nas suas graças, nos seus sofrimentos, na sua morte, na sua ressurreição e na sua glória, e, estando unidos uns aos outros no amor, participam dos mesmos dons e graças e estão obrigados ao cumprimento dos deveres públicos e particulares que contribuem para o seu mútuo proveito, tanto no homem interior como no exterior. \cite{cfw}.
\end{citacao}

\subsubsection{Catecismo Maior} 
\begin{citacao}
P. 63. O que é a comunhão dos santos? \\
R. A comunhão dos santos é aquela comunhão que os crentes têm com Cristo, e uns com os outros, na graça, sofrimentos, morte, ressurreição e glória dele; e sendo assim, são obrigados a amar uns aos outros como membros de um mesmo corpo, e a exercer comunhão uns com os outros em coisas espirituais e temporais segundo suas possibilidades e necessidades. \cite{catecismoMaior}.
\end{citacao}

\subsubsection{Catecismo de Heidelberg}
\begin{citacao}
P. 55. Como você entende as palavras: ``a comunhão dos santos''?
R. Primeiro: entendo que todos os crentes, juntos e cada um por si, têm, como membros, comunhão com Cristo, o Senhor, e todos os seus ricos dons. Segundo: que todos devem sentir-se obrigados a usar seus dons com vontade e alegria para o bem dos outros membros \cite{heidelberg}
\end{citacao}

\subsubsection{Segunda Confissão Helvética} 
\textit{``A Igreja é chamada de comunhão dos santos, porque, reunidos na verdadeira fé, todos os crentes participam de todas as bênçãos espirituais que Deus outorga à sua Igreja.''} \cite{helvetica}

\subsection{Contribuições de Louis Berkhof}

\subsubsection{Igreja como corpo espiritual e comunidade dos santos}
\begin{citacao}
    a igreja da qual a Bíblia diz coisas tão gloriosas não é a igreja considerada como instituição externa, mas a igreja como corpo espiritual de Jesus Cristo, que é essencialmente invisível no presente, [...] e esteja destinada a ter uma perfeita encarnação visível no fim dos séculos. \cite[p. 644]{berkhof2012}.
\end{citacao}

\subsubsection{Igreja visível e invisível como uma só essência}
\begin{citacao}
Estas não são duas igrejas, mas uma somente e, portanto, têm apenas uma única essência. [...] A igreja visível é a igreja como o homem a vê [...] julgados como sendo a comunidade dos santos. \cite[p. 642]{berkhof2012}.
\end{citacao}

\subsection{Contribuições de Herman Bavinck}

\subsubsection{Igreja como organismo e instituição}

\subsubsection{Diversidade de dons e mutualidade}
\begin{citacao}
Toda igreja local [...] é uma comunicação de santos na qual todos sofrem e se alegram uns com os outros e usam seus dons especiais ``pronta e agradavelmente para o serviço e o enriquecimento dos outros membros''. \cite[p. 380]{bavinck2012}.
\end{citacao}

\subsubsection{A comunhão como graça presente e escatológica}

\section{Desdobramentos éticos contemporâneos}

\subsection{Comunhão real: confissão, escuta, presença}
\begin{citacao}
O primeiro serviço que alguém deve ao outro na comunidade é ouvi-lo. Assim como o amor a Deus começa quando ouvimos a sua Palavra, assim também o amor ao irmão começa quando aprendemos a escutá-lo. [...] Esquecem que ouvir pode ser um serviço maior do que falar. \cite[pp. 75--76]{bonhoeffer1997}.
\end{citacao}

\begin{citacao}O pecado oculto separava-o da comunhão, desmentia toda a comunhão aparente; o pecado professado ajudou-o a encontrar a verdadeira comunhão com os irmãos em Jesus Cristo.'' \cite[p. 80]{bonhoeffer1997}.
\end{citacao}

\subsection{A comunhão como responsabilidade mútua}

\subsection{Relevância para a era digital}

\subsection{Síntese das compreensões ao longo da história}

\subsection{A Comunhão dos Santos como realidade}

\begin{citacao}
\foreignlanguage{english}{Essentially the church can be understood only as a divine act, that is, in the utterance of faith; only upon this basis can it be understood as an 'experience'; only faith comprehends the church as a community established by God.}\footnote{Tradução: ``Essencialmente, a igreja só pode ser compreendida como um ato divino, ou seja, na expressão da fé; somente sobre essa base ela pode ser entendida como 'experiência'; só a fé compreende a igreja como uma comunidade estabelecida por Deus.''} \cite[p. 195]{bonhoeffer1963}
\end{citacao}

\chapter{A Comunhão dos Santos de Gênesis a teologia contemporânea}

\section{Perspectivas bíblicas}

\subsection{No Antigo Testamento}

Antes da queda, essa comunhão perfeita era caracterizada pela harmonia entre Criador e criatura, refletida no fato de que ``o SENHOR Deus andava pelo jardim'' em relacionamento próximo com o homem (\gls{gn} 3.8), em um ambiente de plena transparência, onde ``ambos estavam nus e não se envergonhavam'' (\gls{gn} 2.25). A ordem dada diretamente por Deus (\gls{gn} 2.16-17) e a responsabilidade confiada ao homem de cuidar do Éden (\gls{gn} 2.15) evidenciam uma vida partilhada em amor, verdade e justiça, que foi declarada ``muito boa'' pelo próprio Criador (\gls{gn} 1.31). Esta condição original de santidade e unidade é reconhecida também pelos Cânones de Dort, que descrevem que:

\begin{citacao}
    no princípio, o homem foi criado à imagem de Deus, adornado em seu entendimento com o verdadeiro e salutar conhecimento de Deus e de todas as coisas espirituais, com a vontade e o coração retos, e os afetos puros; portanto, era o homem completamente santo. \cite{cdd}
\end{citacao}

Essa comunhão desimpedida expressava desde o princípio o chamado do ser humano à vida comunitária sob a orientação de Deus, apontando para aquilo que seria restaurado de maneira real em Cristo por meio da comunhão dos santos, embora ainda estejamos em um estado corruptível e em um processo contínuo de santificação \footnote{\gls{rm} 8.23 ou \gls{2co} 4.16}, e que alcançará sua plenitude na eternidade: um povo reunido em vínculo com Deus e entre si, crescendo no conhecimento do Senhor, até que esse conhecimento e essa unidade sejam plenamente consumados na glória futura.

Essa realidade é reafirmada com a constituição do povo de Israel. Em \gls{ex} 19.5-6, Deus chama os israelitas para serem um reino de sacerdotes e uma nação santa, estabelecendo a aliança como fundamento para uma vida comunitária regida pela justiça e pelo amor (\gls{lv} 19.18).

A Bíblia de Estudos de Genebra, comentando \gls{jz} 20.1, registra que \textit{pela primeira vez, Israel agiu unido, mas o propósito era guerrear contra seus irmãos''}\footnote{Trecho na \gls{naa}: \textit{como se fosse um só homem''}.} \cite[p. 443]{bibliaGenebra}, evidenciando que a unidade, embora visível, pode ser mal orientada quando desconectada dos princípios divinos.

O Salmo 133 enfatiza o valor da verdadeira comunhão entre os crentes. No versículo 1, Davi declara: \textit{``Oh! Como é bom e agradável viverem unidos os irmãos''} (\gls{naa}). A Bíblia de Estudos de Genebra observa que este cântico de romagem, embora possa aludir a irmãos biológicos, provavelmente se refere às famílias tribais e nacionais reunidas para adorar ao Senhor \cite[p. 1053]{bibliaGenebra}. Ainda segundo essa fonte, a união celebrada no salmo aponta para o relacionamento entre os crentes no contexto da adoração no monte Sião \cite[p. 1053]{bibliaGenebra}. Matthew Henry, ao comentar este salmo, destaca que, embora Davi tenha inicialmente em vista a unidade das tribos de Israel, o princípio enunciado se aplica à comunhão dos santos em todos os tempos, sendo esta a principal bênção exaltada no texto sagrado \cite[p. 1077]{mathewHenry}.

A literatura sapiencial reforça a importância do companheirismo na caminhada humana. Em \gls{ec} 4.9-12, destaca-se que \textit{``melhor é serem dois do que um''} (\gls{naa}), pois a cooperação proporciona apoio mútuo e maior resistência diante das adversidades.

A expectativa de uma unidade restaurada entre os povos de todas as nações já aparece nas visões proféticas. Isaías profetiza que \textit{os estrangeiros que se unirem ao Senhor''} (\gls{is} 56.6-7, \gls{naa}) seriam aceitos em sua casa de oração, e Miquéias profetiza que \textit{muitos povos subirão ao monte do SENHOR''} (\gls{mq} 4.1-2, \gls{naa}). Essas promessas não apenas anunciam a inclusão dos gentios, mas também apontam para a reconstrução da relação rompida no Éden, agora sendo restabelecida em Cristo, que reúne em si todos os que são chamados pelo nome do Senhor em uma única comunidade de fé.

A promessa da nova aliança fundamenta a consolidação de uma vida comunitária renovada. Em \gls{jr} 31.33-34, o profeta anuncia: \textit{``Não ensinará jamais cada um ao seu próximo, nem cada um ao seu irmão, dizendo: 'Conheça o SENHOR!' Porque todos me conhecerão''} (\gls{naa}). Ainda que, em Cristo, o conhecimento de Deus já tenha sido inaugurado entre o seu povo na igreja visível, o cultivo da comunhão mútua permanece essencial, pois é por meio do relacionamento entre os crentes que o conhecimento do Senhor é aprofundado e vivido na prática da fé.

\subsection{No Novo Testamento}

Nos primeiros escritos do Novo Testamento, a identidade dos cristãos como ``os santos'' (\textit{\foreignlanguage{greek}{οἱ ἅγιοι}}) já é estabelecida. Os crentes são constantemente referidos como santos em Cristo Jesus ou chamados a viver de maneira separada (santa), como se vê, por exemplo, em \textit{``a todos os que em qualquer lugar invocam o nome de nosso Senhor Jesus Cristo, Senhor deles e nosso''} (\gls{1co} 1.2, \gls{naa}) e em \textit{``a todos os amados de Deus que estão em Roma, chamados para serem santos''} (\gls{rm} 1.7, \gls{naa}). Essa designação revela que a comunhão dos santos, embora ainda não formalizada como expressão doutrinária, já está presente como realidade espiritual entre os fiéis. Ela abrange tanto a união entre os crentes em suas comunidades locais quanto a participação na igreja como um todo, fundada na fé comum e na vocação santa recebida em Cristo Jesus.

Essa realidade espiritual da vocação comum dos crentes se afirma na história da redenção com a formação da igreja primitiva. Em Atos dos Apóstolos, logo após o Pentecostes, a nova comunidade de fé é descrita como perseverante \textit{``no ensino dos apóstolos, na comunhão, no partir do pão e nas orações''} (\gls{at} 2.42, \gls{naa}), o que mostra que a comunhão cristã não é algo acessório ou restrito a momentos específicos, mas elemento essencial que molda a vida e a identidade do povo de Deus. Nesse contexto, Calvino observa que \textit{``a doutrina é o vínculo da comunhão fraterna entre nós''}\cite{calvinoAtos2}, ressaltando que a verdadeira comunhão nasce da fidelidade ao ensino apostólico e é sustentada pela verdade de Deus. A palavra grega utilizada para comunhão (\textit{κοινωνία}) denota, segundo Albert Barnes, \textit{``ter coisas em comum, ou participação, sociedade, amizade''} \cite{barnesAtos2}, indicando que a união entre os crentes envolvia partilha de missão e vida, e não se limitava a uma associação exterior. Barnes também comenta que \textit{``eles se uniram aos apóstolos e participaram com eles de tudo o que lhes aconteceu''} \cite{barnesAtos2}, evidenciando a profundidade dessa união, e conclui que \textit{``o efeito de um reavivamento da religião é unir cada vez mais os cristãos e trazer aqueles que antes eram separados à união e ao amor''}\cite{barnesAtos2}, destacando o agir do Espírito Santo como fonte e sustentação dessa comunhão.

Percebe-se em Atos que \textit{``da multidão dos que criam, era um o coração e a alma''} (\gls{at} 4.32, \gls{naa}), que esta unidade transcende aparências e brota de uma transformação interior. João Calvino observa que \textit{``a unidade interior das mentes precede como raiz, e o fruto dela é a partilha de bens''}\cite{calvinoAtos4}, enfatizando que a verdadeira comunhão nasce de um coração transformado e se expressa em ações práticas. John Wesley comenta que \textit{``cada pessoa era de um só coração e uma alma – seu amor, suas esperanças, suas paixões se uniam [...] essa era uma consequência necessária dessa união de coração''}\cite{wesleyAtos4}, indicando que amar também é cuidar, e cuidar é se importar com tudo o que possa de alguma forma tocar a vida do próximo. Tal princípio lança luz sobre a responsabilidade ética do cristão, inclusive no zelo pelas informações e dados que envolvem a vida comunitária, pois a verdadeira comunhão exige uma atitude de proteção, respeito e edificação mútua em todas as esferas da convivência cristã.

O Novo Testamento apresenta tanto a prática da oração individual, como ensinado por Jesus em Mateus 6.6, quanto a prática da oração comunitária, como registrado em Atos 1.14 e 2.42. Em ambas as situações, nota-se que a oração ocorre em um ambiente de confiança e comunhão, respeitando o caráter pessoal da relação entre o crente e Deus, e a edificação mútua entre os irmãos. Essa perspectiva evidencia que a vida comunitária cristã sempre foi marcada por responsabilidade e zelo na preservação da dignidade, da honra e da intimidade dos irmãos em Cristo, elementos que integram a realidade da comunhão dos santos.

A doutrina apostólica expande ainda mais essa compreensão da comunhão nos escritos de Paulo. Em sua primeira carta aos Coríntios, ele descreve a comunhão como participação real no corpo e no sangue de Cristo, afirmando: \textit{``o cálice da bênção que abençoamos não é a comunhão do sangue de Cristo? O pão que partimos não é a comunhão do corpo de Cristo?''} (\gls{1co} 10.16, \gls{naa}). Segundo Albert Barnes, este versículo ensina que \textit{``ao participar da Ceia, os cristãos se dedicam a Cristo e reconhecem sua união com Ele e uns com os outros''}\cite{barnes1co10}. João Calvino também comenta que \textit{``o cálice da bênção é separado para ser um emblema do sangue de Cristo, pela palavra da promessa, quando os crentes se reúnem para celebrar a lembrança de sua morte''}\cite{calvino1co10}, ressaltando que a comunhão, como um sacramento, não é um mero ritual simbólico, mas um selo espiritual de união com o Salvador.

Esta verdade confere à comunhão dos santos uma dimensão de ainda maior responsabilidade e zelo: participar da comunhão implica não apenas desfrutar dos benefícios da redenção, mas também viver em coerência com essa união, de forma santa e diligente. A Ceia do Senhor, como expressão da comunhão mística com Cristo e da comunhão visível entre os crentes, exige discernimento e reverência, conforme Paulo adverte: \textit{``pois quem come e bebe sem discernir o corpo, come e bebe juízo para si''} (\gls{1co} 11.29, \gls{naa}). Assim, a comunhão sacramental chama o cristão a um viver ético, em que o cuidado e a proteção da comunhão visível — inclusive em aspectos sensíveis como a gestão de informações e relações — tornam-se expressões práticas da fidelidade a Cristo e do amor fraterno entre os santos.

A comunhão não é apenas uma relação horizontal entre crentes, mas está enraizada na união vital com o próprio Cristo. Essa verdade é ilustrada pela analogia do corpo: \textit{``assim como o corpo é um e tem muitos membros, e todos os membros, mesmo sendo muitos, constituem um só corpo, assim também é com respeito a Cristo''} (\gls{1co} 12.12, \gls{naa}). Paulo enfatiza que, embora haja diversidade de dons e funções, todos são necessários para a edificação do corpo (cf. \gls{1co} 12.12-27).

Essa diversidade de dons não se limita apenas às funções estritamente teológicas ou pastorais, mas abrange todo saber e capacidade que possa servir para o cuidado da vida comunitária. Assim como o zelo pastoral visa proteger e edificar espiritualmente, também o zelo ético — no trato de informações, dados sensíveis e responsabilidades administrativas — é expressão legítima da vida em comunhão. Ignorar conhecimentos específicos que Deus concede a alguns membros do corpo, seja na área jurídica, tecnológica ou organizacional, é desprezar dons que foram dados para a edificação mútua. Como Paulo ensina: \textit{``o olho não pode dizer à mão: 'Não preciso de você' ''} (\gls{1co} 12.21, \gls{naa}); portanto, reconhecer e valorizar a contribuição de cada membro é fazer com que a comunhão dos santos se manifeste de maneira prática, cumprindo a vontade de Deus em todas as áreas da vida. Em contrapartida, negligenciar essas contribuições é negligenciar a diversidade do agir de Deus em seu próprio corpo, desprezando dons que foram concedidos para a edificação mútua.

Em Efésios, o apóstolo destaca que essa unidade é fruto da reconciliação realizada por Cristo: \textit{``Ele é a nossa paz, o qual de ambos fez um e quebrou a parede de separação que estava no meio''} (\gls{ef} 2.14, \gls{naa}). João Calvino comenta que Cristo \textit{``unindo judeus e gentios em um só corpo, removeu a separação estabelecida pela lei cerimonial''} \cite{calvinoEf2}, ressaltando que a comunhão dos santos é uma realidade concreta, e não meramente um símbolo ou conceito distante. Essa realidade demanda a remoção prática de todo tipo de preconceito, bloqueio, resistência ou barreira.

É importante reconhecer que, assim como Cristo derrubou a parede que separava judeus e gentios, também somos chamados a derrubar as barreiras que nos impedem de exercer a comunhão plena no corpo de Cristo. Um bom teólogo, ou qualquer membro da igreja, precisa estar atento às próprias limitações e remover as divisões internas que desconectam a prática teológica e pastoral da vida real da comunidade. Quando se despreza a contribuição de irmãos que possuem dons e saberes que nem sempre são estritamente teológicos, ergue-se novamente, de maneira sutil, a parede que Cristo já derrubou. A comunhão dos santos exige diálogo, humildade e disposição para reconhecer que a edificação do corpo depende da contribuição de todos, sob a direção daquele que é a nossa paz.

A prática da comunhão também se manifesta no modo de viver dos crentes. Paulo dá graças pela \textit{``participação no evangelho desde o primeiro dia até agora''} (\gls{fp} 1.5, \gls{naa}), mostrando que a comunhão inclui a cooperação ativa na missão de Deus. Aos colossenses, ele instrui a vivência comunitária marcada por compaixão, humildade e gratidão: \textit{``revesti-vos, pois, como eleitos de Deus, santos e amados, de ternos afetos de misericórdia, de bondade, de humildade, de mansidão, de longanimidade''} (\gls{cl} 3.12, \gls{naa}), culminando na exortação para que \textit{``a paz de Cristo seja o árbitro no coração de vocês''} (\gls{cl} 3.15, \gls{naa}).

O autor de Hebreus reforça a importância da comunhão para a perseverança na fé: \textit{``consideremo-nos também uns aos outros, para nos estimularmos ao amor e às boas obras, não abandonando a nossa congregação''} (\gls{hb} 10.24-25, \gls{naa}). De modo semelhante, Pedro exorta: \textit{``sejam todos de igual ânimo, compadecidos, fraternalmente amigos, misericordiosos, humildes''} (\gls{1pe} 3.8, \gls{naa}). João, em sua primeira carta, destaca que a comunhão com Deus e com os irmãos é marca essencial da vida cristã: \textit{``o que vimos e ouvimos anunciamos também a vocês, para que vocês tenham comunhão conosco; e a nossa comunhão é com o Pai e com seu Filho, Jesus Cristo''} (\gls{1jo} 1.3, \gls{naa}).

Essa comunhão, ainda imperfeita neste mundo, será plenamente consumada na eternidade. Em Apocalipse, Cristo assegura à sua igreja: \textit{``Não tenha medo. Eu sou o Primeiro e o Último. Eu sou aquele que vive; estive morto, mas eis que estou vivo pelos séculos dos séculos''} (\gls{ap} 1.17-18, \gls{naa}). Os que estão em Cristo, mesmo enfrentando a morte, permanecem vivos n'Ele e participarão da comunhão perfeita na nova criação, onde não haverá mais separação, dor ou pecado. Assim, a comunhão, que hoje é vivida de maneira imperfeita, já é um testemunho antecipado da unidade perfeita que será plenamente revelada na eternidade.

\section{Período Patrístico}

A comunhão dos santos, vivida e testemunhada no Novo Testamento, continuou a ser refletida nos séculos seguintes, à medida que a Igreja enfrentava novos desafios teológicos e estruturais. No período patrístico, essa realidade desenvolveu-se também no campo da formulação doutrinária, enquanto as comunidades cristãs buscavam expressar de maneira clara sua fé comum e preservar a unidade da Igreja.

Já nos documentos mais antigos, observa-se essa preocupação comunitária. A \textit{Didaché}\footnote{em grego clássico: \foreignlanguage{greek}{Διδαχń}; ``ensino'', ``doutrina'', ``instrução'', também chamado Instrução dos Doze Apóstolos \cite{wikididaque}}, um dos primeiros escritos cristãos pós-apostólicos, orientava: \textit{``Vigie sobre a vida uns dos outros. Não deixe que sua lâmpada se apague, nem afrouxe o cinto dos rins. Fique preparado porque você não sabe a que horas nosso Senhor chegará''}\cite{didachePaulus2013}. Essa exortação reflete a consciência de que a comunhão dos santos implicava não apenas convivência, mas vigilância mútua e responsabilidade espiritual.

A necessidade de consolidar a fé comum também se evidencia na reflexão teológica de Santo Agostinho: \textit{``o cristianismo da primeira hora, no entanto, não tinha um símbolo propriamente dito; no século I, fórmulas trinitárias e cristológicas o precedem''}\footnote{Tradução, compilação e organização moderna realizadas pela equipe editorial da Paulus, conforme volume 32 da Coleção Patrística.} \cite[p. 8]{santos2013}. Essas fórmulas, usadas especialmente em contextos batismais e catequéticos, pavimentaram o caminho para o surgimento, nos séculos seguintes, das confissões que deram origem ao Credo Apostólico, tal como o conhecemos hoje.

A consciência da comunhão entre os santos, fundamentada na fé em Cristo, impulsionou também a necessidade de uma identidade compartilhada e publicamente confessada, tanto para fortalecer a vida interna da Igreja quanto para protegê-la diante das heresias.

Essa consciência levou à concepção da Igreja como mais do que uma simples reunião de crentes. Desde os dias de Tertuliano, ela passou a ser reconhecida como \textit{mater fidelium}\footnote{Tradução: mãe dos crentes}, não como fruto da iniciativa humana, mas como obra do próprio Cristo:
\begin{citacao}
Desde os dias de Tertuliano, todos os cristãos chamavam a igreja não apenas de comunidade reunida (\textit{coetus}), mas também de mãe dos crentes (\textit{mater fidelium}). [...] A instituição da igreja, pelo menos de acordo com a confissão reformada, absolutamente não é um produto da comunidade de fé, mas uma obra do próprio Cristo. \cite[335]{bavinck2012}.
\end{citacao}

A preservação da comunhão também exigia um ordenamento ético baseado no amor e na paciência. Santo Agostinho, ao refletir sobre a convivência dos crentes na Igreja visível, aconselha:

\begin{citacao}
Portanto, que corrija misericordiosamente o que pode, mas suporte pacientemente o que não pode, e com amor, que se lamente ou se empenhe, até que ou se emende e corrija do alto, ou adie até a colheita a remoção do joio e a separação da palha\footnote{Tradução livre a partir do texto latino: \textit{``\foreignlanguage{latin}{Misericorditer igitur corripiat homo quod potest, quod autem non potest patienter ferat et cum dilectione gemat aut lugeat, donec aut ille desuper emendet et corrigat aut usque ad messem differat eradicare zizania et paleam.}''} \cite[livro III, capítulo 1, seção 15]{agostinhoContraParmeniani}. João Calvino também cita esse trecho, no contexto em que combate o rigor de puristas extremados e ressalta que a comunhão da Igreja inclui suportar as fraquezas dos irmãos, rejeitando uma visão perfeccionista da comunidade cristã \cite[v. 4, cap. 1, seção 16, p. 1887]{calvino2022}.}.
\end{citacao}

Dessa forma, a comunhão dos santos, já nos primeiros séculos da Igreja, não era apenas uma experiência espiritual, mas envolvia responsabilidade mútua (\gls{jo} 17.21), identidade comunitária e compromisso ético diante de Deus e dos irmãos.

\section{Período Escolástico}

No desenvolvimento da fé cristã durante a Idade Média, comentando o Credo Apostólico, Tomás de Aquino reflete sobre a natureza da comunhão dos santos à luz da analogia paulina do corpo de Cristo. Ele afirma:
\begin{citacao}
Assim como no corpo natural a atividade de um membro subordina-se ao bem de todo o corpo, também no corpo espiritual acontece o mesmo, isto é, na Igreja. E porque todos os fiéis são um só corpo, o bem de um comunica-se ao outro. \cite[p. 79]{aquinoCredo2004}
\end{citacao}

Dessa maneira, ao citar explicitamente Paulo aos Romanos, Tomás de Aquino reafirma que a comunhão dos santos não é apenas uma unidade invisível, mas uma realidade relacional: cada membro, com seus dons e responsabilidades, deve atuar em benefício de todo o corpo eclesiástico. A negligência ou a decisão isolada de um membro contra a ordem estabelecida compromete o bem comum, ressaltando a importância do zelo comunitário e do respeito às estruturas que regem a vida da comunidade dos santos.

Embora a escolástica\footnote{A escolástica foi o método dominante de ensino teológico e filosófico na Idade Média, buscando sistematizar a fé cristã segundo categorias filosóficas, especialmente aristotélicas.} tenha refinado a compreensão da fé cristã e aprofundado a reflexão sobre a comunhão dos santos, progressivamente certos desvios e excessos comprometeram a simplicidade e a pureza da vida comunitária da Igreja. No próprio comentário de Tomás de Aquino sobre a comunhão dos santos, observa-se a associação desta realidade com a necessidade dos sete sacramentos para a participação na graça de Deus\footnote{Tomás de Aquino, ao tratar da comunhão dos santos no \textit{\foreignlanguage{latin}{Expositio in Symbolum Apostolorum}}, conecta esta comunhão aos sete sacramentos adotados pela \gls{icar} \cite{aquinoCredo2004}, apresentando-os como meios necessários de acesso à graça e à edificação do corpo eclesial.}, o que revela um processo de crescente sacramentalização da vida cristã. Diante desse cenário, a Reforma Protestante emergiu no século XVI como um movimento de retorno à centralidade da Palavra e à verdadeira comunhão dos crentes em Cristo, buscando restaurar a vida eclesial conforme o padrão bíblico.

\section{Na Reforma do séc. XVI}

A Reforma do século XVI representou, entre outras dimensões, uma recuperação da compreensão bíblica da comunhão dos santos. Ao invés de entender a comunhão como algo restrito à mediação hierárquica e sacramental da Igreja institucional, como predominava na escolástica, os Reformadores enfatizaram a comunhão como uma realidade espiritual operada diretamente por Deus, por meio da fé e da atuação do Espírito Santo.

João Calvino, ao tratar da Igreja visível\footnote{Calvino define Igreja visível como mãe dos fiéis. \cite[v. 4, p. 1865]{calvino2022}}, destaca que a comunhão dos santos não é uma mera associação externa, mas a expressão viva da partilha mútua dos benefícios espirituais concedidos por Deus. Ele afirma que:
\begin{citacao}
A comunhão dos santos [...] exprime excelentemente a natureza da Igreja [...] que todos e quaisquer benefícios que Deus lhes confira, entre si, mutuamente, compartilhem \cite[v. 4, p. 1863]{calvino2022}.
\end{citacao}

Essa compreensão fundamenta-se na certeza de que os dons espirituais não são propriedades individuais ou institucionais, mas sim destinados à edificação do corpo de Cristo.

A segurança dessa comunhão está na união com Cristo. Calvino reforça que pertencer à Igreja não se baseia na força humana, mas na eleição divina e na firmeza do Salvador:
\begin{citacao}
nossa salvação se apoia em suportes seguros e sólidos, de forma que, ainda quando seja abalada toda a máquina do universo, ela própria não se mova e tombe por terra: primeiro, ela se sustenta com a divina eleição, não pode variar ou falhar mais do que sua eterna providência; então, associada, de certo modo, com a firmeza de Cristo, que não permitirá que seus fiéis sejam afastados de si mais do que permitirá que sejam arrancados e despedaçados seus membros \cite[v. 4, pp. 1863-1864]{calvino2022}.
\end{citacao}

A Igreja, portanto, como comunhão dos santos, é o meio ordinário pelo qual Deus sustenta os crentes na fé e no amor.

Ao enfatizar a gravidade do abandono da comunhão visível, Calvino adverte:
\begin{citacao}
o abandono da Igreja é negação de Deus e de Cristo. Portanto, devemos evitar ainda mais essa ímpia separação. Pois enquanto nos esforçamos, com todas as nossas forças, em arruinar a verdade de Deus, merecemos que ele lance os seus raios com todo o ímpeto de sua ira, a fim de nos fazer em pedaços. Não se pode imaginar algo mais abominável do que o de violar com sacrílega traição o matrimônio que o Unigênito Filho de Deus contraiu conosco \cite[v. 4, pp. 1877-1878]{calvino2022}.
\end{citacao}

Tal advertência ressoa com os desafios contemporâneos enfrentados pela Igreja na gestão dos vínculos comunitários, especialmente em tempos de crescente individualismo e virtualização das relações. Preservar a comunhão dos santos, para Calvino, não é apenas um dever institucional, mas uma expressão de fidelidade a Deus e de zelo pelo próximo.

Assim, para Calvino, a comunhão dos santos não é mero conceito institucional, mas uma realidade viva na qual cada crente participa ativamente do bem espiritual do outro. A comunhão verdadeira é tanto dom quanto responsabilidade.

Essa compreensão da comunhão como expressão concreta da vida cristã é também enfatizada por Martinho Lutero. Comentando o Terceiro Artigo do Credo no \textit{Catecismo Maior}, Lutero afirma:
\begin{citacao}
Creio que existe na terra um santo grupinho e uma congregação compostos apenas de santos, sob uma só cabeça, Cristo, grupo congregado pelo Espírito Santo, em uma só fé, mente e um entendimento, com diversidade de dons, mas unânimes no amor, sem seitas e sem cismas \cite[p. 454]{livroConcordia}.
\end{citacao}

Lutero destaca que essa unidade espiritual é formada pela ação do Espírito Santo, e não por estruturas humanas. A comunhão dos santos, para Lutero, envolve a participação ativa dos crentes na vida e nos dons uns dos outros, unificados sob Cristo como cabeça da Igreja. Trata-se de uma comunhão que, ao mesmo tempo que é dom recebido, exige responsabilidade concreta em amor e serviço.

A Confissão de Augsburgo, importante símbolo de fé luterana, também reflete essa recuperação bíblica da natureza da Igreja ao afirmar:
\begin{citacao}
Além disso, ainda que a igreja cristã, propriamente falando, outra coisa não é senão a congregação de todos os crentes e santos, todavia, já que nesta vida continuam entre os piedosos muitos falsos cristãos e hipócritas, também, pecadores manifestos, os sacramentos nada obstante são eficazes, embora os sacerdotes que os administram não sejam piedosos. Conforme o próprio Cristo indica: ``Na cadeira de Moisés estão sentados os fariseus\footnote{Mateus 23.2}.''\cite[Artigo VII, p. 32]{livroConcordia}
\end{citacao}

Esta definição rompe com o conceito escolástico predominante no período anterior, segundo o qual a Igreja era identificada primordialmente pela hierarquia e pelos sacramentos operados \textit{\foreignlanguage{latin}{ex opere operato}}\footnote{A expressão latina ``\textit{\foreignlanguage{latin}{ex opere operato}}'', em tradução literal ``obra realizada'', refere-se à crença de que os sacramentos conferem graça automaticamente, independentemente da fé do ministrante ou do participante, ou seja, basta que o rito seja realizado de forma válida, para que ele seja eficaz em comunicar a graça. Sobre a instituição dos sacramentos, Calvino afirma: \textit{``Acerca dos sacramentos seria suficiente para convencer todas as pessoas sóbrias e ensináveis a não abraçarem quaisquer outros sacramentos [...] com a exceção daqueles dois [...] instituídos pelo Senhor''} \cite[v. 4, Capítulo XIX, p. 2605]{calvino2022}}. Enquanto Tomás de Aquino havia integrado essa visão institucionalizada da graça e da comunhão, a Confissão de Augsburgo devolve a centralidade à Palavra e à correta administração dos sacramentos como critérios visíveis da verdadeira Igreja. Assim, reafirma que a comunhão dos santos é sustentada pela ação soberana de Deus através dos meios da graça, e não pela mediação hierárquica da Igreja institucional.

A Reforma restaurou o entendimento de que a comunhão dos santos é fruto do agir direto de Deus entre seu povo, tanto na partilha espiritual quanto na prática comunitária em amor e serviço. Essa restauração oferece o fundamento necessário para a compreensão da ética relacionada à comunhão dos santos, fornecendo o alicerce teológico para entender a vida comunitária da Igreja como um espaço de mútuo cuidado, responsabilidade e edificação.

\subsection{Nos símbolos de fé}

\subsubsection{A Confissão de Fé de Westminster} 
\begin{citacao}
Todos os santos que pelo seu Espírito e pela fé estão unidos a Jesus Cristo, seu Cabeça, têm com Ele comunhão nas suas graças, nos seus sofrimentos, na sua morte, na sua ressurreição e na sua glória, e, estando unidos uns aos outros no amor, participam dos mesmos dons e graças e estão obrigados ao cumprimento dos deveres públicos e particulares que contribuem para o seu mútuo proveito, tanto no homem interior como no exterior. \cite{cfw}.
\end{citacao}

\subsubsection{Catecismo Maior} 
\begin{citacao}
P. 63. O que é a comunhão dos santos?\\
R. A comunhão dos santos é aquela comunhão que os crentes têm com Cristo, e uns com os outros, na graça, sofrimentos, morte, ressurreição e glória dele; e, sendo assim, são obrigados a amar uns aos outros como membros de um mesmo corpo e a exercer comunhão uns com os outros em coisas espirituais e temporais segundo suas possibilidades e necessidades. \cite{catecismoMaior}.
\end{citacao}

\subsubsection{Catecismo de Heidelberg}
\begin{citacao}
P. 55. Como você entende as palavras: ``a comunhão dos santos''?\\
R. Primeiro: entendo que todos os crentes, juntos e cada um por si, têm, como membros, comunhão com Cristo, o Senhor, e todos os seus ricos dons. Segundo: que todos devem sentir-se obrigados a usar seus dons com vontade e alegria para o bem dos outros membros \cite{heidelberg}
\end{citacao}

\subsubsection{Segunda Confissão Helvética} 
\textit{``A Igreja é chamada de comunhão dos santos, porque, reunidos na verdadeira fé, todos os crentes participam de todas as bênçãos espirituais que Deus outorga à sua Igreja.''} \cite{helvetica}

\subsection{Contribuições de Louis Berkhof}

\subsubsection{Igreja como corpo espiritual e comunidade dos santos}
\begin{citacao}
    a igreja da qual a Bíblia diz coisas tão gloriosas não é a igreja considerada como instituição externa, mas a igreja como corpo espiritual de Jesus Cristo, que é essencialmente invisível no presente, [...] e esteja destinada a ter uma perfeita encarnação visível no fim dos séculos. \cite[p. 644]{berkhof2012}.
\end{citacao}

\subsubsection{Igreja visível e invisível como uma só essência}
\begin{citacao}
Estas não são duas igrejas, mas uma somente e, portanto, têm apenas uma única essência. [...] A igreja visível é a igreja como o homem a vê [...] julgados como sendo a comunidade dos santos. \cite[p. 642]{berkhof2012}.
\end{citacao}

\subsection{Contribuições de Herman Bavinck}

\subsubsection{Igreja como organismo e instituição}

\subsubsection{Diversidade de dons e mutualidade}
\begin{citacao}
Toda igreja local [...] é uma comunicação de santos na qual todos sofrem e se alegram uns com os outros e usam seus dons especiais ``pronta e agradavelmente para o serviço e o enriquecimento dos outros membros''. \cite[p. 380]{bavinck2012}.
\end{citacao}

\subsubsection{A comunhão como graça presente e escatológica}

\section{Desdobramentos éticos contemporâneos}

\subsection{Comunhão real: confissão, escuta, presença}
\begin{citacao}
O primeiro serviço que alguém deve ao outro na comunidade é ouvi-lo. Assim como o amor a Deus começa quando ouvimos a sua Palavra, assim também o amor ao irmão começa quando aprendemos a escutá-lo. [...] Esquecem que ouvir pode ser um serviço maior do que falar. \cite[pp. 75--76]{bonhoeffer1997}.
\end{citacao}

\begin{citacao}O pecado oculto separava-o da comunhão, desmentia toda a comunhão aparente; o pecado professado ajudou-o a encontrar a verdadeira comunhão com os irmãos em Jesus Cristo.'' \cite[p. 80]{bonhoeffer1997}.
\end{citacao}

\subsection{A comunhão como responsabilidade mútua}

\subsection{Relevância para a era digital}

\subsection{Síntese das compreensões ao longo da história}

\subsection{A Comunhão dos Santos como realidade}

\begin{citacao}
\foreignlanguage{english}{Essentially the church can be understood only as a divine act, that is, in the utterance of faith; only upon this basis can it be understood as an 'experience'; only faith comprehends the church as a community established by God.}\footnote{Tradução: ``Essencialmente, a igreja só pode ser compreendida como um ato divino, ou seja, na expressão da fé; somente sobre essa base ela pode ser entendida como 'experiência'; só a fé compreende a igreja como uma comunidade estabelecida por Deus.''} \cite[p. 195]{bonhoeffer1963}
\end{citacao}

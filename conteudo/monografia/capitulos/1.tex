\chapter{A Comunhão dos Santos na história da Igreja}

\section{Perspectivas bíblicas}

\subsection{A Igreja em Atos como ponto de partida}

Esta pesquisa parte do pressuposto de que a comunhão dos santos, enquanto realidade espiritual e relacional, não pode ser compreendida de forma abstrata ou desconectada do texto bíblico, mas sim a partir da Igreja como comunidade formada pelo Espírito. Mais especificamente, o recorte adotado neste trabalho privilegia a Igreja conforme apresentada no livro de Atos dos Apóstolos, com ênfase na narrativa de Atos 2.41–47. Tal escolha de delimitação bíblica se justifica porque, embora o termo \textit{\foreignlanguage{greek}{ἐκκλησία}} apareça pela primeira vez em Mateus 16.18, é em Atos 2 que a Igreja se torna uma realidade histórica e visível, reunida e capacitada pelo Espírito Santo, dando início à sua missão no mundo.

Como afirma Bavinck, foi ali que a comunidade de discípulos se tornou uma \textit{``assembleia religiosa independente''} \cite[p.~284]{bavinck2012}, marcada por um novo princípio de vida e unidade no Espírito. No entanto, essa ancoragem neotestamentária da Igreja não está desconectada da história da redenção. Louis Berkhof adverte que, embora o Pentecostes represente uma nova fase, a essência da Igreja já estava presente no povo da antiga aliança:

\begin{citacao}
    Permanece o fato de que, tanto no Velho Testamento como no Novo, a palavra denota uma congregação ou assembléia [sic] do povo de Deus, e, como tal, serve para designar a essência da igreja. \cite[p.~650]{berkhof2012}
\end{citacao}

Clowney reforça esse entendimento ao afirmar que \textit{``sua vinda\footnote{do Espírito} encheu a assembléia [sic] de discípulos no Pentecostes e estabeleceu a igreja da nova aliança''} \cite[p.~24]{clowney2007}. E conclui:

\begin{citacao}
    A igreja não vive com uma memória desbotada da presença do Senhor, mas com a realidade de sua vinda no Espírito. O povo de Deus, reivindicado por Cristo no sangue da nova aliança, tornou-se a comunhão do Espírito enquanto espera a volta do Senhor. \cite[p.~46]{clowney2007}
\end{citacao}

\subsubsection{Base textual}

\subsection{Base textual: o uso do Texto Majoritário}

Esta pesquisa utiliza o \textit{Texto Majoritário} como base grega do Novo Testamento, conforme apresentado na obra \textit{``Novo Testamento Interlinear Analítico Grego-Português''} \cite{interlinear2008}. A escolha se justifica pelo compromisso com uma tradição textual mais ampla e preservada, em contraste com edições baseadas em um número reduzido de manuscritos antigos.

\begin{citacao}
    O Texto Majoritário é um texto que emprega a evidência disponível da completa gama de manuscritos remanescentes, ao invés de apoiar-se primariamente na evidência propiciada por poucos textos. Para nós não é científico praticamente ignorar oitenta a noventa por cento da evidência em qualquer disciplina \cite[p.~ix]{interlinear2008}.
\end{citacao}

Esse posicionamento se distancia da tradição crítica de Westcott e Hort, cujas edições favoreciam leituras apoiadas em dois textos egípcios do quarto século, como o Sinaítico e o Vaticano. Os editores do Interlinear argumentam, no entanto, que leituras do Texto Majoritário têm recebido novo reconhecimento, inclusive com apoio de papiros do século II antes considerados incompatíveis com essa tradição. Sobre esta edição de \citeonline{interlinear2008}, na introdução também é dito que:

\begin{citacao}
    representa um primeiro passo na direção do reconhecimento do valor e da autoridade da grande massa de documentos gregos remanescentes [...] em direção a um Novo Testamento grego que mais acuradamente reflita os inspirados autógrafos \cite[p.~xiii]{interlinear2008}.
\end{citacao}

Ao adotar o Texto Majoritário como base para a análise de Atos 2.41–47, este trabalho se alinha a uma tradição textual difundida na cristandade e busca também colaborar com o debate atual sobre crítica textual e fidelidade aos manuscritos do Novo Testamento.

\subsection{Dimensões éticas da comunhão em Atos 2.41-47}

A comunhão dos santos não é uma ideia acessória no cristianismo, mas uma verdade central da fé cristã ao longo da história. Está presente em credos antigos como o Credo Apostólico, como veremos mais à frente, ao lado da proclamação da Igreja universal e do perdão dos pecados. O termo \textit{``comunhão''} traduz o grego \textit{\foreignlanguage{greek}{κοινωνία}}, palavra que aparece no Novo Testamento para descrever tanto o relacionamento dos crentes com Deus (\gls{1jo} 1.3) quanto entre si (\gls{at} 2.42). \textit{\foreignlanguage{greek}{κοινωνία}} carrega o sentido de participação, partilha e vínculo. Essa realidade tem base bíblica e também foi aprofundada por autores cristãos ao longo dos séculos, como se verá na leitura de Atos.

A comunhão cristã não é um projeto individualista. O Novo Testamento não contempla uma fé solitária. No livro \textit{``Vida em Comunhão''} é afirmado por \citeonline[p.~13]{bonhoeffer1997} que: \textit{``O cristão precisa do cristão que lhe diga a Palavra de Deus''}. A salvação nos une a Cristo, e essa união nos vincula também ao seu corpo, que é a Igreja. Nesse sentido, pode-se dizer que a comunhão é o acréscimo da pessoa à Igreja: não existe comunhão no ``eu sozinho''. Bonhoeffer também diz que \textit{``a pessoa que não se encontra na comunhão, que tome cuidado com a solidão.''} \cite[p.~59]{bonhoeffer1997} O próprio termo \textit{koinonia} pressupõe relação, vínculo, partilha. Isso significa que a espiritualidade cristã, ainda que tenha momentos íntimos e pessoais, é essencialmente comunitária. Ser parte do corpo de Cristo é participar da vida uns dos outros em amor e responsabilidade.

A passagem de \gls{at} 2.41-47 abre uma janela para o cotidiano da igreja em seus primeiros passos. Antes mesmo de descrever as práticas da comunidade, o versículo 41 informa que \textit{``os que aceitaram a palavra foram batizados, havendo um acréscimo naquele dia de quase três mil pessoas.''} A comunhão, portanto, começa com o acréscimo da pessoa à igreja, fruto da resposta à pregação e do batismo. Trata-se de uma inserção relacional e espiritual que rompe com qualquer ideia de fé isolada. Vale refletir sobre essa inserção, pois certamente ela implicava responsabilidades quanto ao cuidado com informações compartilhadas, evitando a exposição de questões sensíveis sobre a vida daquelas pessoas que estavam sendo acrescidas à Igreja. Em seguida, os versículos 42 a 47 revelam práticas que não surgiram de regras externas, mas de algo que brotava de dentro, fruto da presença do Espírito entre eles manifestada na vida comum e em gestos concretos. Era um grupo que levava a sério a fé, que se dedicava com constância ao ensino, à vida em comum, às refeições compartilhadas e à oração. Mas, além de nos oferecer uma visão histórica e teológica, esse trecho também convida a refletir sobre a maneira como lidamos com a convivência cristã hoje, especialmente em tempos tão marcados pela exposição e pela quebra de confiança.

O verbo \textit{``perseveravam''}, do grego \textit{\foreignlanguage{greek}{προσκαρτερέω}}, aparece no tempo imperfeito, o que indica uma ação constante, habitual. O termo tem o sentido de dedicar-se a algo com firmeza, com continuidade e propósito. Ele não sugere um ato pontual ou esporádico, mas uma postura de entrega regular, marcada por constância. Esse mesmo verbo é usado em outros trechos de Atos (1.14; 6.4), sempre associado a uma atitude de compromisso coletivo. Aqui, em 2.42, ele aparece antes de quatro práticas essenciais: o ensino dos apóstolos, a comunhão, o partir do pão e as orações. Trata-se de um retrato completo da espiritualidade cristã como prática comunitária e perseverante. João Calvino observa que \textit{``a doutrina é o vínculo da comunhão fraterna entre nós''} \cite{calvinoAtos2}, ressaltando que essa fidelidade prática não nascia de mera formalidade religiosa, mas de um coração moldado pela Palavra. Eles não se encontravam por conveniência, mas porque sabiam que seguir a Cristo envolvia também caminhar com os irmãos, dividindo a vida como ela é, com seus pesos e respiros, sem máscaras, mas com disposição para caminhar juntos. A oração, nesse contexto, não era uma formalidade litúrgica, mas um momento de confidência, onde pessoas confiavam umas às outras o que carregavam diante de Deus. Isso exigia respeito mútuo e sensibilidade. Afinal, quem se dispõe a orar em comunidade se permite ser visto como realmente é. Tiago nos lembra: \textit{``Confessem os seus pecados uns aos outros e orem uns pelos outros para serem curados''} \gls{tg} 5.16 \gls{naa}. Há cura quando há confiança, e confiança só se constrói quando sabemos que o que é dito em oração será guardado com zelo. Onde não há respeito por aquilo que é dito em oração, a comunhão enfraquece. Se esse ambiente não fosse marcado por zelo, cuidado e maturidade, a comunhão não sobreviveria.

Mesmo que naquele tempo ninguém falasse em “privacidade de dados”, havia um senso claro de que certas informações exigiam zelo. O contexto hostil à fé cristã tornava a comunhão algo precioso, mas também arriscado. O próprio governador romano Plínio, o Jovem, descreve que os cristãos eram acusados simplesmente por se reunirem para cantar hinos e comprometer-se com uma vida íntegra:

\begin{citacao}
    Afirmavam, no entanto, que toda a sua culpa ou erro se resumia em ter o hábito de se reunir num dia marcado antes do amanhecer e de cantar, alternadamente, um hino a Cristo como a um deus; de se comprometerem com um juramento — não para algum crime, mas para não cometerem furto, roubo ou adultério, para não enganarem a palavra dada, nem se negarem a devolver um depósito quando solicitados \cite[p.~272]{plinioCartasX}.
\end{citacao}

Essa descrição mostra que, mesmo com uma expressão de fé sincera, a má interpretação era bem comum. Por isso, os cristãos precisavam ter cautela com quem participava das reuniões, com o que era dito e como era dito. A comunhão não podia ser ingênua. Assim como hoje há riscos em ambientes digitais, ali também havia riscos nos encontros presenciais. Zelar por comunhão era também proteger quem dela participava. Essa mesma tensão ainda é vivida por cristãos perseguidos em diversos países hoje. Em muitos lugares, ser igreja ainda exige sigilo, cautela e oração constante por proteção. Como relatado em um livro chamado \textit{O contrabandista de Deus}, que conta a história de um homem que levava Bíblias para cristãos em países da Igreja Perseguida:

\begin{citacao}
    Éramos sete, naquela noite, sete crentes reunindo-se quase da mesma forma que os cristãos se têm reunido desde que a Igreja teve início — em segredo, com dificuldades — orando para que, pela miraculosa intervenção do próprio Deus, nós fôssemos poupados de enfrentar as autoridades. \cite[p.~166]{brotherAndrew}
\end{citacao}

Esse testemunho de algo ocorrido em 1957 mostra que a prática da comunhão, mesmo sob vigilância, não desaparece; ela apenas se torna mais zelosa. E isso continua sendo um sinal da ação do Espírito na preservação do corpo de Cristo em qualquer época. O zelo, portanto, não é sinal de incredulidade, nem produto de teorias conspiratórias. É responsabilidade com o próximo e com o corpo de Cristo.

Outro ponto importante é a partilha de bens. Calvino, ao comentar \gls{at} 4.32, escreve que \textit{``a unidade interior das mentes precede como raiz, e o fruto dela é a partilha de bens''} \cite{calvinoAtos4}, revelando que a comunhão começa na transformação interior e se manifesta em atitudes práticas. O texto diz que os cristãos vendiam propriedades e ajudavam quem precisava. Essa generosidade, por mais pública que fosse, não significa que houvesse exposição desnecessária de quem dava ou recebia. Os ensinamentos de Jesus, como aquele em que Ele diz para a mão esquerda não saber o que faz a direita, apontam para sermos discretos, a termos cuidado que protege a dignidade das pessoas. Transportando esse princípio para os dias de hoje, há um alerta contra o exibicionismo nas redes, contra a coleta exagerada de informações e contra o desrespeito a contextos íntimos.

A comunhão descrita em Atos também envolvia refeições, alegria e simplicidade. John Wesley comenta que \textit{``cada pessoa era de um só coração e uma alma – seu amor, suas esperanças, suas paixões se uniam [...] essa era uma consequência necessária dessa união de coração''} \cite{wesleyAtos4}, destacando que esse tipo de convivência se expressava no cuidado mútuo e na sensibilidade com a vida uns dos outros. Não era um teatro de aparências. Era uma vida real, com vínculos reais. Em contraste, as interações digitais muitas vezes são breves, filtradas, performáticas. A singeleza dos primeiros cristãos nos lembra que a vida em comunidade não precisa ser ostentada, só precisa ser sincera.

No fim da passagem, vemos que esse modo de vida gerava louvor a Deus e também atraía pessoas de fora. A comunhão, quando vivida com integridade, não apenas fortalece os de dentro, mas desperta interesse em quem está de fora. Num tempo em que tanta coisa vaza, escandaliza ou decepciona, uma comunidade que sabe cuidar, guardar e acolher pode ser um sinal forte da graça de Deus no mundo.

Essa leitura não tenta transformar Atos 2 num manual sobre tecnologia, mas busca ouvir seus princípios com atenção e honestidade. O modo como tratamos as informações dos outros, como nos expomos e como cuidamos dos vínculos que temos, tudo isso é expressão da comunhão que dizemos viver. E se é assim, então o cuidado com a privacidade, com o que é dito, com o que é mostrado ou escondido, faz parte da fidelidade a esse chamado. O próximo capítulo vai se dedicar justamente a isso: como os meios digitais desafiam, e podem até distorcer, essa comunhão que os primeiros cristãos cultivaram com tanta alegria e reverência.

\section{Período Patrístico}

A comunhão dos santos, vivida e testemunhada no Novo Testamento, continuou a ser refletida nos séculos seguintes, à medida que a Igreja enfrentava novos desafios teológicos e estruturais. No período patrístico, essa realidade desenvolveu-se também no campo da formulação doutrinária, enquanto as comunidades cristãs buscavam expressar de maneira clara sua fé comum e preservar a unidade da Igreja.

Já nos documentos mais antigos, observa-se essa preocupação comunitária. A \textit{Didaché}\footnote{Em grego clássico: \foreignlanguage{greek}{Διδαχὴ}; significa ensino, doutrina ou instrução, sendo também chamada de Instrução dos Doze Apóstolos \cite{wikididaque}. Entre as traduções disponíveis em português, utilizamos a da Paulus \cite{didachePaulus2013}, embora haja outras versões publicadas, com variações mínimas no conteúdo e na forma.}, que é um dos primeiros escritos cristãos pós-apostólicos, orientava: \textit{`Vigie sobre a vida uns dos outros. Não deixe que sua lâmpada se apague, nem afrouxe o cinto dos rins. Fique preparado porque você não sabe a que horas nosso Senhor chegará''} \cite{didachePaulus2013}. Essa exortação reflete a consciência de que a comunhão dos santos implicava não apenas convivência, mas vigilância mútua e responsabilidade espiritual.

A necessidade de consolidar a fé comum também se evidencia na reflexão teológica de Santo Agostinho: \textit{``o cristianismo da primeira hora, no entanto, não tinha um símbolo propriamente dito; no século I, fórmulas trinitárias e cristológicas o precedem''}\footnote{Tradução, compilação e organização moderna realizadas pela equipe editorial da Paulus, conforme volume 32 da Coleção Patrística.} \cite[p.~8]{santos2013}. Essas fórmulas, usadas especialmente em contextos batismais e catequéticos, pavimentaram o caminho para o surgimento, nos séculos seguintes, das confissões que deram origem ao Credo Apostólico, tal como o conhecemos hoje.

A consciência da comunhão entre os santos, fundamentada na fé em Cristo, impulsionou também a necessidade de uma identidade compartilhada e publicamente confessada, tanto para fortalecer a vida interna da Igreja quanto para protegê-la diante das heresias.

Essa consciência levou à concepção da Igreja como mais do que uma simples reunião de crentes. Desde os dias de Tertuliano, ela passou a ser reconhecida como \textit{mater fidelium}\footnote{Tradução nossa: mãe dos crentes}, não como fruto da iniciativa humana, mas como obra do próprio Cristo:

\begin{citacao}
    Desde os dias de Tertuliano, todos os cristãos chamavam a igreja não apenas de comunidade reunida (\textit{coetus}), mas também de mãe dos crentes (\textit{mater fidelium}). [...] A instituição da igreja, pelo menos de acordo com a confissão reformada, absolutamente não é um produto da comunidade de fé, mas uma obra do próprio Cristo. \cite[p.~335]{bavinck2012}.
\end{citacao}

A preservação da comunhão também exigia um ordenamento ético baseado no amor e na paciência. Santo Agostinho, ao refletir sobre a convivência dos crentes na Igreja visível, aconselha:

\begin{citacao}
    Portanto, que corrija misericordiosamente o que pode, mas suporte pacientemente o que não pode, e com amor, que se lamente ou se empenhe, até que ou se emende e corrija do alto, ou adie até a colheita a remoção do joio e a separação da palha\footnote{Tradução nossa, a partir do texto latino: \textit{``\foreignlanguage{latin}{Misericorditer igitur corripiat homo quod potest, quod autem non potest patienter ferat et cum dilectione gemat aut lugeat, donec aut ille desuper emendet et corrigat aut usque ad messem differat eradicare zizania et paleam.}''} \cite[livro~III, capítulo~1, seção~15]{agostinhoContraParmeniani}. João Calvino também cita esse trecho, no contexto em que combate o rigor de puristas extremados e ressalta que a comunhão da Igreja inclui suportar as fraquezas dos irmãos, rejeitando uma visão perfeccionista da comunidade cristã \cite[v.~4, cap.~1, seção~16, p.~1887]{calvino2022}.}.
\end{citacao}

Dessa forma, a comunhão dos santos, já nos primeiros séculos da Igreja, não era apenas uma experiência espiritual, mas envolvia responsabilidade mútua (\gls{jo} 17.21), identidade comunitária e compromisso ético diante de Deus e dos irmãos.

\section{Período Escolástico}

No desenvolvimento da fé cristã durante a Idade Média, comentando o Credo Apostólico, Tomás de Aquino reflete sobre a natureza da comunhão dos santos à luz da analogia paulina do corpo de Cristo. Ele afirma:

\begin{citacao}
    Assim como no corpo natural a atividade de um membro subordina-se ao bem de todo o corpo, também no corpo espiritual acontece o mesmo, isto é, na Igreja. E porque todos os fiéis são um só corpo, o bem de um comunica-se ao outro. \cite[p.~79]{aquinoCredo2004}
\end{citacao}

Dessa maneira, ao citar explicitamente Paulo aos Romanos, Tomás de Aquino reafirma que a comunhão dos santos não é apenas uma unidade invisível, mas uma realidade relacional: cada membro, com seus dons e responsabilidades, deve atuar em benefício de todo o corpo eclesiástico. A negligência ou a decisão isolada de um membro contra a ordem estabelecida compromete o bem comum, ressaltando a importância do zelo comunitário e do respeito às estruturas que regem a vida da comunidade dos santos.

Embora a escolástica\footnote{A escolástica foi o método dominante de ensino teológico e filosófico na Idade Média, buscando sistematizar a fé cristã segundo categorias filosóficas, especialmente aristotélicas.} tenha refinado a compreensão da fé cristã e aprofundado a reflexão sobre a comunhão dos santos, progressivamente certos desvios e excessos comprometeram a simplicidade e a pureza da vida comunitária da Igreja. No próprio comentário de Tomás de Aquino sobre a comunhão dos santos, observa-se a associação desta realidade com a necessidade dos sete sacramentos para a participação na graça de Deus\footnote{Tomás de Aquino, ao tratar da comunhão dos santos no \textit{\foreignlanguage{latin}{Expositio in Symbolum Apostolorum}}, conecta esta comunhão aos sete sacramentos adotados pela \gls{icar} \cite{aquinoCredo2004}, apresentando-os como meios necessários de acesso à graça e à edificação do corpo eclesial.}, o que revela um processo de crescente sacramentalização da vida cristã. Diante desse cenário, a Reforma Protestante emergiu no século XVI como um movimento de retorno à centralidade da Palavra e à verdadeira comunhão dos crentes em Cristo, buscando restaurar a vida eclesial conforme o padrão bíblico.

\section{Na Reforma do séc. XVI}

A Reforma do século XVI representou, entre outras dimensões, uma recuperação da compreensão bíblica da comunhão dos santos. Ao invés de entender a comunhão como algo restrito à mediação hierárquica e sacramental da Igreja institucional, como predominava na escolástica, os Reformadores enfatizaram a comunhão como uma realidade espiritual operada diretamente por Deus, por meio da fé e da atuação do Espírito Santo.

João Calvino, ao tratar da Igreja visível\footnote{Calvino define Igreja visível como mãe dos fiéis. \cite[v.~4, p.~1865]{calvino2022}}, destaca que a comunhão dos santos não é uma mera associação externa, mas a expressão viva da partilha mútua dos benefícios espirituais concedidos por Deus. Ele afirma que:

\begin{citacao}
    A comunhão dos santos [...] exprime excelentemente a natureza da Igreja [...] que todos e quaisquer benefícios que Deus lhes confira, entre si, mutuamente, compartilhem \cite[v.~4, p.~1863]{calvino2022}.
\end{citacao}

Essa compreensão fundamenta-se na certeza de que os dons espirituais não são propriedades individuais ou institucionais, mas sim destinados à edificação do corpo de Cristo.

A segurança dessa comunhão está na união com Cristo. Calvino reforça que pertencer à Igreja não se baseia na força humana, mas na eleição divina e na firmeza do Salvador:

\begin{citacao}
    nossa salvação se apoia em suportes seguros e sólidos, de forma que, ainda quando seja abalada toda a máquina do universo, ela própria não se mova e tombe por terra: primeiro, ela se sustenta com a divina eleição, não pode variar ou falhar mais do que sua eterna providência; então, associada, de certo modo, com a firmeza de Cristo, que não permitirá que seus fiéis sejam afastados de si mais do que permitirá que sejam arrancados e despedaçados seus membros \cite[v.~4, pp.~1863--1864]{calvino2022}.
\end{citacao}

A Igreja, portanto, como comunhão dos santos, é o meio ordinário pelo qual Deus sustenta os crentes na fé e no amor.

Ao enfatizar a gravidade do abandono da comunhão visível, Calvino adverte:

\begin{citacao}
    o abandono da Igreja é negação de Deus e de Cristo. Portanto, devemos evitar ainda mais essa ímpia separação. Pois enquanto nos esforçamos, com todas as nossas forças, em arruinar a verdade de Deus, merecemos que ele lance os seus raios com todo o ímpeto de sua ira, a fim de nos fazer em pedaços. Não se pode imaginar algo mais abominável do que o de violar com sacrílega traição o matrimônio que o Unigênito Filho de Deus contraiu conosco \cite[v.~4, pp.~1877--1878]{calvino2022}.
\end{citacao}

Tal advertência ressoa com os desafios contemporâneos enfrentados pela Igreja na gestão dos vínculos comunitários, especialmente em tempos de crescente individualismo e virtualização das relações. Preservar a comunhão dos santos, para Calvino, não é apenas um dever institucional, mas uma expressão de fidelidade a Deus e de zelo pelo próximo.

Assim, para Calvino, a comunhão dos santos não é mero conceito institucional, mas uma realidade viva na qual cada crente participa ativamente do bem espiritual do outro. A comunhão verdadeira é tanto dom quanto responsabilidade.

Essa compreensão da comunhão como expressão concreta da vida cristã é também enfatizada por Martinho Lutero. Comentando o Terceiro Artigo do Credo no \textit{Catecismo Maior}, Lutero afirma:

\begin{citacao}
    Creio que existe na terra um santo grupinho e uma congregação compostos apenas de santos, sob uma só cabeça, Cristo, grupo congregado pelo Espírito Santo, em uma só fé, mente e um entendimento, com diversidade de dons, mas unânimes no amor, sem seitas e sem cismas \cite[p.~454]{livroConcordia}.
\end{citacao}

Lutero destaca que essa unidade espiritual é formada pela ação do Espírito Santo, e não por estruturas humanas. A comunhão dos santos, para Lutero, envolve a participação ativa dos crentes na vida e nos dons uns dos outros, unificados sob Cristo como cabeça da Igreja. Trata-se de uma comunhão que, ao mesmo tempo que é dom recebido, exige responsabilidade concreta em amor e serviço.

A Confissão de Augsburgo, importante símbolo de fé luterana, também reflete essa recuperação bíblica da natureza da Igreja ao afirmar:

\begin{citacao}
    Além disso, ainda que a igreja cristã, propriamente falando, outra coisa não é senão a congregação de todos os crentes e santos, todavia, já que nesta vida continuam entre os piedosos muitos falsos cristãos e hipócritas, também, pecadores manifestos, os sacramentos nada obstante são eficazes, embora os sacerdotes que os administram não sejam piedosos. Conforme o próprio Cristo indica: ``Na cadeira de Moisés estão sentados os fariseus\footnote{Mateus 23.2}.''\cite[Artigo~VII, p.~32]{livroConcordia}
\end{citacao}

Esta definição rompe com o conceito escolástico predominante no período anterior, segundo o qual a Igreja era identificada primordialmente pela hierarquia e pelos sacramentos operados \textit{\foreignlanguage{latin}{ex opere operato}}\footnote{A expressão latina ``\textit{\foreignlanguage{latin}{ex opere operato}}'' (Tradução nossa: obra realizada), refere-se à crença de que os sacramentos conferem graça automaticamente, independentemente da fé do ministrante ou do participante, ou seja, basta que o rito seja realizado de forma válida, para que ele seja eficaz em comunicar a graça. Sobre a instituição dos sacramentos, Calvino afirma: \textit{``Acerca dos sacramentos seria suficiente para convencer todas as pessoas sóbrias e ensináveis a não abraçarem quaisquer outros sacramentos [...] com a exceção daqueles dois [...] instituídos pelo Senhor''} \cite[v.~4, Capítulo~XIX, p.~2605]{calvino2022}}. Enquanto Tomás de Aquino havia integrado essa visão institucionalizada da graça e da comunhão, a Confissão de Augsburgo devolve a centralidade à Palavra e à correta administração dos sacramentos como critérios visíveis da verdadeira Igreja. Assim, reafirma que a comunhão dos santos é sustentada pela ação soberana de Deus através dos meios da graça, e não pela mediação hierárquica da Igreja institucional.

A Reforma restaurou o entendimento de que a comunhão dos santos é fruto do agir direto de Deus entre seu povo, tanto na partilha espiritual quanto na prática comunitária em amor e serviço. Essa restauração oferece o fundamento necessário para a compreensão da ética relacionada à comunhão dos santos, fornecendo o alicerce teológico para entender a vida comunitária da Igreja como um espaço de mútuo cuidado, responsabilidade e edificação.

\subsection{Nos símbolos de fé}

O desenvolvimento da doutrina da comunhão dos santos também se refletiu nos grandes símbolos confessionais da tradição reformada, que procuraram sistematizar os principais ensinamentos das Escrituras\footnote{Na tradição cristã, os símbolos de fé são exposições oficiais da fé elaboradas por igrejas ou concílios, visando resumir e declarar as verdades centrais da fé cristã. São conhecidos também como confissões ou credos.}. Em resposta aos desafios doutrinários dos séculos XVI e XVII, diversas igrejas reformadas formularam confissões de fé e catecismos que se tornaram referenciais para a identidade teológica e prática da comunidade cristã\footnote{Confissões de fé são declarações sistemáticas de doutrina produzidas por assembleias eclesiásticas. Catecismos, por sua vez, são manuais de instrução cristã em formato de perguntas e respostas, destinados ao ensino da fé, especialmente a novos convertidos e jovens.}.

\subsubsection{A \gls{cfw}} 
A \gls{cfw}, elaborada pela Assembleia de Westminster na Inglaterra, tornou-se um dos documentos confessionais mais influentes da tradição reformada\footnote{A \gls{cfw} foi redigida por teólogos puritanos ingleses durante a Assembleia de Westminster (1643–1653) e é um dos principais símbolos de fé do presbiterianismo histórico.}. Ela descreve a comunhão dos santos como uma união espiritual com Cristo e, por consequência, entre os próprios crentes:

\begin{citacao}
    Todos os santos que pelo seu Espírito e pela fé estão unidos a Jesus Cristo, seu Cabeça, têm com Ele comunhão nas suas graças, nos seus sofrimentos, na sua morte, na sua ressurreição e na sua glória, e, estando unidos uns aos outros no amor, participam dos mesmos dons e graças e estão obrigados ao cumprimento dos deveres públicos e particulares que contribuem para o seu mútuo proveito, tanto no homem interior como no exterior \cite[Cap.~XXVI, §~1º]{cfw}.
\end{citacao}

Este entendimento ressalta que a comunhão dos santos não é apenas uma realidade espiritual invisível, mas se manifesta em obrigações concretas de amor, serviço e edificação mútua no seio da Igreja.

\subsubsection{Catecismo Maior de Westminster}
O \textit{Catecismo Maior de Westminster}\footnote{O Catecismo Maior de Westminster, também fruto da Assembleia de Westminster, foi destinado ao ensino aprofundado da doutrina cristã, especialmente para a formação de adultos na fé reformada.}, complementa a Confissão ao explicar a natureza da comunhão dos santos:

\begin{citacao}
    \textbf{P. 63.} O que é a comunhão dos santos?\\
    \textbf{R.} A comunhão dos santos é aquela comunhão que os crentes têm com Cristo, e uns com os outros, na graça, sofrimentos, morte, ressurreição e glória dele; e, sendo assim, são obrigados a amar uns aos outros como membros de um mesmo corpo e a exercer comunhão uns com os outros em coisas espirituais e temporais segundo suas possibilidades e necessidades \cite{catecismoMaior}.
\end{citacao}

Este catecismo enfatiza a dinâmica horizontal da comunhão dos santos: a responsabilidade concreta dos crentes em servirem uns aos outros tanto em questões espirituais quanto nas necessidades práticas.

\subsubsection{Catecismo de Heidelberg}
O \textit{Catecismo de Heidelberg}, é uma das mais respeitadas confissões da tradição reformada continental, também trata do tema da comunhão dos santos.\footnote{O Catecismo de Heidelberg foi redigido na cidade de Heidelberg, na Alemanha, por comissão do eleitor Frederico III, com o objetivo de ensinar a fé reformada de maneira pastoral e acessível em 1563.} Em sua pergunta 55, ensina:

\begin{citacao}
    \textbf{P. 55.} Como você entende as palavras: ``a comunhão dos santos''?\\
    \textbf{R.} Primeiro: entendo que todos os crentes, juntos e cada um por si, têm, como membros, comunhão com Cristo, o Senhor, e todos os seus ricos dons. Segundo: que todos devem sentir-se obrigados a usar seus dons com vontade e alegria para o bem dos outros membros \cite{heidelberg}.
\end{citacao}

O Catecismo destaca não apenas o privilégio da comunhão com Cristo, mas também o dever alegre de empregar os dons recebidos em benefício da comunidade dos santos.

\subsubsection{Segunda Confissão Helvética}
A \textit{Segunda Confissão Helvética}, é outra confissão reformada clássica que aborda a comunhão dos santos\footnote{A Segunda Confissão Helvética, escrita por Heinrich Bullinger em 1566, sucedeu à Primeira Confissão Helvética e se tornou uma das confissões reformadas mais influentes na Europa continental, especialmente entre igrejas de tradição suíça e alemã.}. Ela afirma:

\begin{citacao}
    A Igreja é chamada de comunhão dos santos, porque, reunidos na verdadeira fé, todos os crentes participam de todas as bênçãos espirituais que Deus outorga à sua Igreja \cite{helvetica}.
\end{citacao}

Essa formulação destaca a comunhão como uma realidade espiritual coletiva, fundamentada na fé comum e na participação conjunta nas bênçãos da redenção.

A abordagem dos símbolos de fé reforça a compreensão de que a comunhão dos santos não é apenas uma doutrina abstrata, mas um princípio que estrutura a vida da Igreja, moldando o relacionamento dos crentes entre si sob a soberania de Cristo.

\subsection{Movimento pietista}

No desenvolvimento histórico da teologia cristã, é importante ressaltar diferenças entre a doutrina reformada da \textit{Communio Sanctorum} \footnote{Versão em latim para ``Comunhão dos Santos''} e a concepção de comunhão espiritual defendida por correntes pietistas e pelo metodismo. Enquanto a tradição reformada, especialmente na teologia de João Calvino, ao tratar da Igreja visível como ``mãe dos fiéis'' \cite[p.~1865]{calvino2022}, afirma de maneira clara que:

\begin{citacao}
    não há outro modo de entrar na vida a não ser que ela nos conceba no ventre, a não ser que nos dê à luz, a não ser que nos nutra em seus seios. \cite[p.~1865]{calvino2022}
\end{citacao}

Calvino destaca também que a remissão de pecados e a salvação não podem ser esperadas fora da Igreja. Em outras palavras, abandonar a Igreja é sempre desastroso. Essa concepção corrige e confronta a chamada ``teologia do eu sozinho'', própria do individualismo religioso moderno, reafirmando o papel essencial da Igreja visível e da comunhão dos santos.

O pietismo luterano, iniciado por Philipp Jakob Spener, nasceu como uma reação legítima à secularização e à politização da Igreja na Europa continental. No entanto, essa reação gerou uma tendência à interiorização da fé, deixando de lado a comunhão dos santos, que passou a ser entendida primariamente como uma experiência subjetiva, emocional e particular. A ênfase prática de Spener na piedade pessoal acabou fragilizando o comprometimento institucional com a igreja visível e suas práticas coletivas, contribuindo para uma espiritualidade mais mística e desvinculada da vida eclesial. Como destaca \citeonline{brenner_pietism}:

\begin{citacao}
    The conventicles began to split churches because the people in the \textit{collegia pietatis} thought that it was necessary to separate from those whom they considered to be unconverted or second-class Christians in the established congregations. \cite[p.~4]{brenner_pietism}\footnote{Tradução minha: Os conventículos começaram a dividir as igrejas porque as pessoas nos \textit{collegia pietatis} achavam que era necessário se separar daqueles que consideravam não convertidos ou cristãos de segunda classe nas congregações estabelecidas.}
\end{citacao}

Essa herança pietista exerceu grande influência no protestantismo ocidental, inclusive em contextos brasileiros contemporâneos, moldando uma compreensão de fé centrada na resolução de questões pessoais e no cultivo de uma espiritualidade desvinculada da responsabilidade comunitária e descentralizada.

Michael Horton também observa:

\begin{citacao}
    O pietismo evangélico começou como um movimento de renovação nas igrejas da Reforma, contudo, cada vez mais, tendeu a reduzir a fé a uma experiência subjetiva interna. A verdadeira ação acontecia em devoções particulares, em reuniões religiosas secretas ou clubes santos (o que hoje chamaríamos de pequenos grupos). \cite[p.~178]{horton2010}
\end{citacao}

Essa deformação do conceito de comunhão, cada vez mais centrada na autonomia do sujeito e na busca por experiências pessoais de espiritualidade, foi amplificada na cultura contemporânea. Como observa \citeonline[p.~164]{horton2010}, comentando a previsão de George Barna:

\begin{citacao}
    veremos que milhões de pessoas não vão se dirigir fisicamente para uma igreja, mas, em vez disso, vão navegar pela internet em busca de experiências espirituais significativas. \cite[p.~164]{horton2010}
\end{citacao}

O culto íntimo, diz Barna, não exigiria ``um culto congregacional'', apenas um compromisso pessoal com a Bíblia e com a oração. Horton conclui:

\begin{citacao}
    Os revolucionários descobriram que, a fim de exercer uma fé autêntica, têm de abandonar a igreja. Esta é a situação na qual a espiritualidade contemporânea nos deixa finalmente: sozinhos, navegando na internet, tentando achar treinadores e companheiros de equipe, tentando nos salvar do cativeiro desta época presente pela descoberta de estímulos que induzirão uma vida transformada. \cite[p.~164]{horton2010}
\end{citacao}

Dessa forma, percebemos que a comunhão dos santos, segundo a teologia reformada, implica em relacionamento não apenas entre irmãos na fé, mas também com Cristo, por meio da participação na vida da Igreja. Trata-se de uma relação orgânica e visível que vem sendo gradualmente desconstruída nas formas modernas de espiritualidade influenciadas pelo pietismo.

\subsection{Contribuições de Louis Berkhof}

Louis Berkhof, em sua renomada \textit{Teologia Sistemática}, trata da natureza da Igreja e de sua expressão visível e invisível. Em sua obra, enfatiza que a verdadeira Igreja, aquela da qual a Escritura fala com tanta glória, não se refere primariamente à instituição externa, mas ao corpo espiritual de Cristo:

\begin{citacao}
    a igreja da qual a Bíblia diz coisas tão gloriosas não é a igreja considerada como instituição externa, mas a igreja como corpo espiritual de Jesus Cristo, que é essencialmente invisível no presente, [...] e esteja destinada a ter uma perfeita encarnação visível no fim dos séculos \cite[p.~644]{berkhof2012}.
\end{citacao}

Essa perspectiva reafirma que a identidade da Igreja transcende suas expressões administrativas e organizacionais. Toda prática de gestão eclesiástica deve ser orientada pelo reconhecimento da dignidade espiritual de seus membros, considerados como parte do corpo de Cristo\footnote{Sobre a compreensão da Igreja como corpo espiritual de Cristo, ver também \gls{ef} 1.22-23, onde a Igreja é descrita como ``o seu corpo, a plenitude daquele que a tudo enche em todas as coisas'' \gls{naa}.}. Berkhof distingue a Igreja como organismo e como instituição: como organismo, manifesta a vida espiritual e os dons carismáticos dos crentes em sua comunhão e serviço mútuo; como instituição, é o meio ordenado por Deus para a preservação e edificação da fé.\footnote{Berkhof distingue entre a Igreja como organismo: a comunhão viva e carismática dos crentes, e como instituição: estrutura visível com ofícios e governo. A primeira é fim, a segunda, meio (cf. \cite[p.~638]{berkhof2012}).}

Berkhof também esclarece que a distinção entre Igreja visível e invisível não implica a existência de duas igrejas distintas, mas diferentes modos de percepção da mesma realidade espiritual:

\begin{citacao}
    Estas não são duas igrejas, mas uma somente e, portanto, têm apenas uma única essência. [...] A igreja visível é a igreja como o homem a vê [...] julgados como sendo a comunidade dos santos \cite[p.~642]{berkhof2012}.
\end{citacao}

Essa unidade de essência traz consigo a responsabilidade ética da administração eclesiástica: a Igreja, mesmo enquanto visível e organizada, não pode se distanciar de sua realidade espiritual. Sua visibilidade, segundo Berkhof, não se manifesta apenas em sua estrutura formal, mas sobretudo na vida comunitária e no testemunho público dos crentes\footnote{Berkhof afirma que, ainda que ofícios e instituições falhem, a Igreja permanece visível na vida comunitária e no testemunho dos fiéis (cf. \cite[p.~645]{berkhof2012}).}.

A compreensão de Berkhof sobre a Igreja como corpo espiritual e comunidade dos santos reforça o princípio de que as relações internas da Igreja devem refletir sua natureza espiritual e o respeito mútuo entre os membros. Essa visão orgânica da Igreja não apenas fundamenta a doutrina da comunhão dos santos, mas exige uma vivência coerente com a dignidade espiritual de cada membro. O cuidado ético é um reflexo prático da unidade vital entre os membros do corpo de Cristo\footnote{Para Berkhof, a unidade da Igreja não é apenas institucional, mas orgânica: \textit{``todas as partes constituintes se relacionam vitalmente umas com as outras''}. (cf. \cite[p.~646]{berkhof2012}).}

\subsection{Contribuições de Herman Bavinck}

Herman Bavinck\footnote{Herman Bavinck (1854–1921) foi um dos mais influentes teólogos reformados do final do século XIX e início do século XX. Sua obra monumental, \textit{Dogmática Reformada}, permanece como uma das mais completas expressões da teologia calvinista, marcada por um esforço constante de integrar piedade e ortodoxia, fé reformada e engajamento com o mundo moderno. Bavinck sustentava a ideia de que ``a graça restaura a natureza'' como princípio unificador de sua teologia.} enfatiza que a igreja possui uma dupla natureza: é tanto um organismo quanto uma instituição. Como organismo, é a comunidade viva dos crentes unidos pelo Espírito; como instituição, é a estrutura organizada com ministérios e ofícios visíveis. Ambos os aspectos são inseparáveis e se relacionam dinamicamente para a edificação dos santos e o testemunho no mundo. A igreja, como ``mãe'' dos fiéis, é anterior ao indivíduo, gerando e nutrindo sua fé na comunhão visível dos santos. \cite{bavinck2012}

A unidade da igreja não anula a diversidade. O Espírito Santo distribui diferentes dons aos crentes, não de maneira arbitrária, mas conforme a medida da fé, a posição de cada um na igreja e a missão que lhe foi confiada. \cite{bavinck2012} Cada membro é chamado a servir utilizando seu dom para a edificação mútua. A mutualidade é central: todos são necessários e se enriquecem reciprocamente. Esses dons são muitos e variados, e devem ser usados ``prontamente e agradavelmente'' para o serviço e o enriquecimento da comunidade.\cite{bavinck2012}

\begin{citacao}
    Toda igreja local [...] é uma comunicação de santos na qual todos sofrem e se alegram uns com os outros e usam seus dons especiais ``pronta e agradavelmente para o serviço e o enriquecimento dos outros membros''. \cite[p.~380]{bavinck2012}
\end{citacao}

O amor fraternal ocupa posição central nesta dinâmica comunitária. Bavinck destaca que, embora a diversidade de dons seja rica e essencial, o amor é o dom supremo que governa todos os demais. \cite[p.~304]{bavinck2012} Trata-se de um amor espiritual, enraizado em Cristo, que transcende até mesmo os vínculos naturais. \cite[p.~304]{bavinck2012}

\section{Desdobramentos éticos contemporâneos}

Bonhoeffer, ao refletir sobre o serviço mútuo na comunidade cristã, apresenta três formas de vivenciar o amor fraternal: escutar, servir com as mãos e carregar os fardos uns dos outros. Esses serviços não são opcionais, mas expressões práticas de uma comunhão real, construída não apenas com palavras ou de forma estética e farisaica, mas por meio da humildade e de entrega cotidiana.

O primeiro serviço, segundo ele, é o da escuta. Em tempos de comunicação acelerada e dispersa, ouvir com atenção se torna um testemunho raro e de muito valor do amor cristão:

\begin{citacao}
    O primeiro serviço que alguém deve ao outro na comunidade é ouvi-lo. Assim como o amor a Deus começa quando ouvimos a sua Palavra, assim também o amor ao irmão começa quando aprendemos a escutá-lo. [...] Quem não consegue mais ouvir o irmão, em breve também não conseguirá mais ouvir a Deus. \cite[pp.~75--76]{bonhoeffer1997}
\end{citacao}

Bonhoeffer associa essa escuta ao ministério da confissão, que rompe com a superficialidade e possibilita encontros verdadeiros com nossos irmãos e com Deus:

\begin{citacao}
    O pecado oculto separava-o da comunhão, desmentia toda a comunhão aparente; o pecado professado ajudou-o a encontrar a verdadeira comunhão com os irmãos em Jesus Cristo. \cite[p.~88]{bonhoeffer1997}
\end{citacao}

Essa escuta não é tarefa exclusiva de pastores ou terapeutas: é um ministério para todo cristão. Sem ela, confundimos o outro, interpretamos mal suas palavras, ou falamos antes de compreender. Com toda esta sobrecarga digital que vivemos e diversos elementos que nos levam à dispersão mental, escutar alguém com atenção total é um ato espiritual e um sinal de resistência à desumanização.

O segundo serviço é o da ajuda prática. Trata-se de dispor-se ao serviço em tarefas reais, mesmo que pareçam pequenas. Bonhoeffer nos lembra que:

\begin{citacao}
    Não há serviço que seja demasiadamente modesto para alguém. [...] Temos que nos dispor e permitir que Deus nos interrompa. \cite[p.~77]{bonhoeffer1997}
\end{citacao}

Essa disponibilidade se opõe a uma vida pietista e focada apenas em nós mesmos , que espiritualiza tudo vivendo um cristianismo estético e ignorando o irmão à beira do caminho. Em um mundo hiperconectado, onde somos constantemente bombardeados por notificações e agendas, consumindo a vida alheia em redes sociais, parar para ajudar com atenção plena é, muitas vezes, uma forma fiel de adoração e expressão de amor ao próximo que faz toda a diferença na vida de quem caminha ao nosso lado.

O terceiro serviço é o de carregar o outro, com suas fragilidades, sua liberdade e até seus pecados. Isso exige paciência, empatia e o compromisso de preservar a confidencialidade como parte do cuidado cristão:

\begin{citacao}
    Antes de mais nada é a liberdade do outro [...] que se torna um fardo para o cristão. [...] A liberdade alheia inclui tudo o que entendemos por natureza, individualidade, predisposição, inclusive as fraquezas e as esquisitices que tanto exigem nossa paciência. \cite[p.~78]{bonhoeffer1997}
\end{citacao}

Aceitar esse fardo é acolher o outro como ele é, sem tentar moldá-lo aos nossos gostos ou à nossa forma de compreender a realidade à nossa volta. E quando alguém nos confia suas dores ou pecados, isso deve ser tratado com respeito e oração. Este ato não deve ser feito como algo que desejamos de alguma forma levar para fora da intimidade do momento, mas como um peso a levarmos juntos diante de Deus que torna o nosso fardo leve e suave.

Esses três serviços ganham uma nova perspectiva no contexto digital. A virtualização das relações criou formas inéditas de comunhão, isto é inquestionável, mas também novos riscos: superficialidade, exposição indevida, e quebra de confidencialidade de forma nunca vista antes na história onde o que é compartilhado é eternizado em meios digitais. Escutar sem atenção, interromper com distrações digitais ou compartilhar sem permissão ferem a ética do cuidado fraterno.

A comunhão cristã exige uma presença real mesmo quando mediada por tecnologia e também requer um compromisso com a escuta, em ajudarmos de verdade quem está se abrindo conosco, dando e recebendo suporte espiritual. Às vezes, o maior gesto de cuidado que podemos oferecer é silenciar o celular, colocá-lo em modo avião ou fora do alcance, para que o outro saiba: ``estou totalmente com você''.

Desde os primeiros séculos, a Igreja testemunha que a comunhão dos santos é dom recebido e responsabilidade cultivada:

\begin{citacao}
    \foreignlanguage{english}{Essentially the church can be understood only as a divine act, that is, in the utterance of faith; only upon this basis can it be understood as an 'experience'; only faith comprehends the church as a community established by God.}\footnote{Tradução nossa: ``Essencialmente, a igreja só pode ser compreendida como um ato divino, ou seja, na expressão da fé; somente sobre essa base ela pode ser entendida como 'experiência'; só a fé compreende a igreja como uma comunidade estabelecida por Deus.''} \cite[p.~195]{bonhoeffer1963}
\end{citacao}

Assim, a Igreja é uma comunhão de pessoas distintas e livres, de diferentes culturas, que foram escolhidas por Deus para caminharem juntas. A comunhão digital, portanto, não pode ser neutra ou automática; ela deve refletir nossa vocação espiritual, marcada pela escuta, demonstrar nosso serviço e amor ao próximo e pelo compromisso de carregar uns aos outros, mesmo quando conectados por telas.

\newpage
\section{A comunhão dos santos de Gênesis a teologia contemporânea}

\subsection{Definições}
\begin{itemize}
    \item Contextualização do tema e sua relevância contemporânea
    \item A tensão entre comunhão espiritual e responsabilidade institucional
    \item Propósito do capítulo
\end{itemize}

\subsection{Perspectivas bíblicas}
\subsubsection{Comunhão na aliança e da vida do povo de Deus}
\begin{itemize}
  \item Criação, aliança e comunidade
  \item Expressão de unidade e justiça
\end{itemize}

\subsubsection{Expressão da vida em Cristo}
\begin{itemize}
  \item Unidade na fé, partilha de dons e vida comunitária
  \item A comunhão como sinal escatológico
\end{itemize}

\subsection{A comunhão dos santos na igreja primitiva}
\subsubsection{Partilha e solidariedade entre os santos}
\subsubsection{Tertuliano e a expressão \textit{mater fidelium}}
\subsubsection{A formulação do credo apostólico}

\subsection{A comunhão dos santos na reforma protestante}
\subsubsection{Comunhão como sustentação da igreja visível}
\subsubsection{A importância da igreja visível}
\subsubsection{A Confissão de Fé de Westminster}
\subsubsection{Catecismo Maior}

\subsection{A comunhão dos santos na teologia reformada pós-reforma}
\subsubsection{Contribuições de Louis Berkhof}
\begin{itemize}
  \item Igreja como corpo espiritual e comunidade dos santos
  \item Igreja visível e invisível como uma só essência
\end{itemize}

\subsubsection{Contribuições de Herman Bavinck}
\begin{itemize}
  \item Igreja como organismo e instituição
  \item Diversidade de dons e mutualidade
  \item A comunhão como graça presente e escatológica
\end{itemize}

\subsection{Desdobramentos éticos contemporâneos}
\subsubsection{Comunhão real: confissão, escuta, presença}
\subsubsection{A comunhão como responsabilidade mútua}
\subsubsection{Relevância para a era digital}
\subsubsection{Síntese das compreensões ao longo da história}
\subsubsection{A comunhão dos santos como realidade}

\chapter{A Comunhão dos Santos de Gênesis a teologia contemporânea}

\section{Perspectivas bíblicas}

\subsection{No Antigo Testamento}

Antes da queda, essa comunhão perfeita era caracterizada pela harmonia entre Criador e criatura, refletida no fato de que "o SENHOR Deus andava pelo jardim" em relacionamento próximo com o homem (\gls{gn} 3.8), em um ambiente de plena transparência, onde "ambos estavam nus e não se envergonhavam" (\gls{gn} 2.25). A ordem dada diretamente por Deus (\gls{gn} 2.16-17) e a responsabilidade confiada ao homem de cuidar do Éden (\gls{gn} 2.15) evidenciam uma vida partilhada em amor, verdade e justiça, que foi declarada "muito boa" pelo próprio Criador (\gls{gn} 1.31). Esta condição original de santidade e unidade é reconhecida também pelos Cânones de Dort, que descrevem que:

\begin{citacao}
    no princípio, o homem foi criado à imagem de Deus, adornado em seu entendimento com o verdadeiro e salutar conhecimento de Deus e de todas as coisas espirituais, com a vontade e o coração retos, e os afetos puros; portanto, era o homem completamente santo. \cite{cdd}
\end{citacao}

Essa comunhão desimpedida expressava desde o princípio o chamado do ser humano à vida comunitária sob a orientação de Deus, apontando para aquilo que seria restaurado de maneira real em Cristo por meio da comunhão dos santos, embora ainda estejamos em um estado corruptível e em um processo contínuo de santificação \footnote{\gls{rm} 8.23 ou \gls{2co} 4.16}, e que alcançará sua plenitude na eternidade: um povo reunido em vínculo com Deus e entre si, crescendo no conhecimento do Senhor, até que esse conhecimento e essa unidade sejam plenamente consumados na glória futura.

Essa realidade é reafirmada com a constituição do povo de Israel. Em \gls{ex} 19.5-6, Deus chama os israelitas para serem um reino de sacerdotes e uma nação santa, estabelecendo a aliança como fundamento para uma vida comunitária regida pela justiça e pelo amor (\gls{lv} 19.18).

A Bíblia de Estudos de Genebra, comentando \gls{jz} 20.1, registra que \textit{pela primeira vez, Israel agiu unido, mas o propósito era guerrear contra seus irmãos''}\footnote{Trecho na \gls{naa}: \textit{como se fosse um só homem''}.} \cite[p. 443]{bibliaGenebra}, evidenciando que a unidade, embora visível, pode ser mal orientada quando desconectada dos princípios divinos.

O Salmo 133 enfatiza o valor da verdadeira comunhão entre os crentes. No versículo 1, Davi declara: \textit{``Oh! Como é bom e agradável viverem unidos os irmãos''} (\gls{naa}). A Bíblia de Estudos de Genebra observa que este cântico de romagem, embora possa aludir a irmãos biológicos, provavelmente se refere às famílias tribais e nacionais reunidas para adorar ao Senhor \cite[p. 1053]{bibliaGenebra}. Ainda segundo essa fonte, a união celebrada no salmo aponta para o relacionamento entre os crentes no contexto da adoração no monte Sião \cite[p. 1053]{bibliaGenebra}. Matthew Henry, ao comentar este salmo, destaca que, embora Davi tenha inicialmente em vista a unidade das tribos de Israel, o princípio enunciado se aplica à comunhão dos santos em todos os tempos, sendo esta a principal bênção exaltada no texto sagrado \cite[p. 1077]{mathewHenry}.

A literatura sapiencial reforça a importância do companheirismo na caminhada humana. Em \gls{ec} 4.9-12, destaca-se que \textit{``melhor é serem dois do que um''} (\gls{naa}), pois a cooperação proporciona apoio mútuo e maior resistência diante das adversidades.

A expectativa de uma unidade restaurada entre os povos de todas as nações já aparece nas visões proféticas. Isaías profetiza que \textit{os estrangeiros que se unirem ao Senhor''} (\gls{is} 56.6-7, \gls{naa}) seriam aceitos em sua casa de oração, e Miquéias profetiza que \textit{muitos povos subirão ao monte do SENHOR''} (\gls{mq} 4.1-2, \gls{naa}). Essas promessas não apenas anunciam a inclusão dos gentios, mas também apontam para a reconstrução da relação rompida no Éden, agora sendo restabelecida em Cristo, que reúne em si todos os que são chamados pelo nome do Senhor em uma única comunidade de fé.

A promessa da nova aliança fundamenta a consolidação de uma vida comunitária renovada. Em \gls{jr} 31.33-34, o profeta anuncia: \textit{``Não ensinará jamais cada um ao seu próximo, nem cada um ao seu irmão, dizendo: 'Conheça o SENHOR!' Porque todos me conhecerão''} (\gls{naa}). Ainda que, em Cristo, o conhecimento de Deus já tenha sido inaugurado entre o seu povo na igreja visível, o cultivo da comunhão mútua permanece essencial, pois é por meio do relacionamento entre os crentes que o conhecimento do Senhor é aprofundado e vivido na prática da fé.

\subsection{No Novo Testamento}

Nos primeiros escritos do Novo Testamento, a identidade dos cristãos como "os santos" (\textit{οἱ ἅγιοι}) já é estabelecida. Os crentes são constantemente referidos como santos em Cristo Jesus ou chamados a viver de maneira separada (santa), como se vê, por exemplo, em \textit{``a todos os que em qualquer lugar invocam o nome de nosso Senhor Jesus Cristo, Senhor deles e nosso''} (\gls{1co} 1.2, \gls{naa}) e em \textit{``a todos os amados de Deus que estão em Roma, chamados para serem santos''} (\gls{rm} 1.7, \gls{naa}). Essa designação revela que a comunhão dos santos, embora ainda não formalizada como expressão doutrinária, já está presente como realidade espiritual entre os fiéis. Ela abrange tanto a união entre os crentes em suas comunidades locais quanto a participação na igreja como um todo, fundada na fé comum e na vocação santa recebida em Cristo Jesus.

Essa realidade espiritual da vocação comum dos crentes se afirma na história da redenção com a formação da igreja primitiva. Em Atos dos Apóstolos, logo após o Pentecostes, a nova comunidade de fé é descrita como perseverante \textit{``no ensino dos apóstolos, na comunhão, no partir do pão e nas orações''} (\gls{at} 2.42, \gls{naa}), o que mostra que a comunhão cristã não é algo acessório ou restrito a momentos específicos, mas elemento essencial que molda a vida e a identidade do povo de Deus. Nesse contexto, Calvino observa que \textit{``a doutrina é o vínculo da comunhão fraterna entre nós''}\cite{calvinoAtos2}, ressaltando que a verdadeira comunhão nasce da fidelidade ao ensino apostólico e é sustentada pela verdade de Deus. A palavra grega utilizada para comunhão (\textit{κοινωνία}) denota, segundo Albert Barnes, \textit{``ter coisas em comum, ou participação, sociedade, amizade''} \cite{barnesAtos2}, indicando que a vida em comum entre os crentes não se limitava a uma união exterior, mas envolvia partilha real de vida e missão. Barnes ainda comenta que \textit{``eles se uniram aos apóstolos e participaram com eles de tudo o que lhes aconteceu''} \cite{barnesAtos2}, evidenciando a profundidade dessa união, e conclui que \textit{``o efeito de um reavivamento da religião é unir cada vez mais os cristãos e trazer aqueles que antes eram separados à união e ao amor''} \cite{barnesAtos2}, destacando o agir do Espírito Santo como fonte e sustentação dessa comunhão.

Percebe-se também quando Lucas afirma em Atos que \textit{``da multidão dos que criam, era um o coração e a alma''} (\gls{at} 4.32, \gls{naa}), que esta unidade transcende aparências e brota de uma transformação interior. João Calvino observa que \textit{``a unidade interior das mentes precede como raiz, e o fruto dela é a partilha de bens''} \cite{calvinoAtos4}, enfatizando que a verdadeira comunhão nasce de um coração transformado e se expressa em ações práticas. John Wesley comenta que \textit{``cada pessoa era de um só coração e uma alma – seu amor, suas esperanças, suas paixões se uniam [...] essa era uma consequência necessária dessa união de coração''} \cite{wesleyAtos4}, indicando que amar também é cuidar, e cuidar é se importar com tudo o que possa de alguma forma tocar a vida do próximo. Tal princípio lança luz sobre a responsabilidade ética do cristão, inclusive no zelo pelo que diz respeito às informações e dados que envolvem a vida comunitária, pois a verdadeira comunhão exige uma atitude de proteção, respeito e edificação mútua em todas as esferas da convivência cristã.

O Novo Testamento apresenta tanto a prática da oração individual, como ensinado por Jesus em Mateus 6.6, quanto a prática da oração comunitária, como registrado em Atos 1.14 e 2.42. Em ambas as situações, nota-se que a oração ocorre em um ambiente de confiança e comunhão, respeitando o caráter pessoal da relação entre o crente e Deus, e a edificação mútua entre os irmãos. Essa perspectiva evidencia que a vida comunitária cristã sempre foi marcada por responsabilidade e zelo na preservação da dignidade, da honra e da intimidade dos irmãos em Cristo, elementos que integram a realidade da comunhão dos santos.

A doutrina apostólica expande ainda mais essa compreensão da comunhão nos escritos de Paulo. Em sua primeira carta aos Coríntios, ele descreve a comunhão como participação real no corpo e no sangue de Cristo, afirmando: \textit{``o cálice da bênção que abençoamos não é a comunhão do sangue de Cristo? O pão que partimos não é a comunhão do corpo de Cristo?''} (\gls{1co} 10.16, \gls{naa}). Segundo Albert Barnes, este versículo ensina que \textit{``ao participar da Ceia, os cristãos se dedicam a Cristo e reconhecem sua união com Ele e uns com os outros''} \cite{barnes1co10}. João Calvino também comenta que \textit{``o cálice da bênção é separado para ser um emblema do sangue de Cristo, pela palavra da promessa, quando os crentes se reúnem para celebrar a lembrança de sua morte''} \cite{calvino1co10}, ressaltando que a comunhão, como um sacramento, não é um mero ritual simbólico, mas um selo espiritual de união com o Salvador.

Esta verdade confere à comunhão dos santos uma dimensão de ainda maior responsabilidade e zelo: participar da comunhão implica não apenas desfrutar dos benefícios da redenção, mas também viver em coerência com essa união, de forma santa e diligente. A Ceia do Senhor, como expressão da comunhão mística com Cristo e da comunhão visível entre os crentes, exige discernimento e reverência, conforme Paulo adverte adiante: \textit{``pois quem come e bebe sem discernir o corpo, come e bebe juízo para si''} (\gls{1co} 11.29, \gls{naa}). Assim, a comunhão sacramental chama o cristão a um viver ético, em que o cuidado e a proteção da comunhão visível — inclusive em aspectos sensíveis como a gestão de informações e relações — tornam-se expressões práticas da fidelidade a Cristo e do amor fraterno entre os santos.

A comunhão não é apenas uma relação horizontal entre crentes, mas está enraizada na união vital com o próprio Cristo. Essa verdade é ilustrada pela analogia do corpo: \textit{``assim como o corpo é um e tem muitos membros, e todos os membros, mesmo sendo muitos, constituem um só corpo, assim também é com respeito a Cristo''} (\gls{1co} 12.12, \gls{naa}). Paulo enfatiza que, embora haja diversidade de dons e funções, todos são necessários para a edificação do corpo (cf. \gls{1co} 12.12-27).

Essa diversidade de dons não se limita apenas às funções estritamente teológicas ou pastorais, mas abrange todo saber e capacidade que possa servir para o cuidado da vida comunitária. Assim como o zelo pastoral visa proteger e edificar espiritualmente, também o zelo ético — no trato de informações, dados sensíveis e responsabilidades administrativas — é expressão legítima da vida em comunhão. Ignorar conhecimentos específicos que Deus concede a alguns membros do corpo, seja na área jurídica, tecnológica ou organizacional, é desprezar dons que foram dados para a edificação mútua. Como Paulo ensina, \textit{``o olho não pode dizer à mão: 'Não preciso de você' ''} (\gls{1co} 12.21, \gls{naa}); portanto, reconhecer e valorizar a contribuição de cada membro é fazer com que a comunhão dos santos se manifeste de maneira prática, cumprindo a vontade de Deus em todas as áreas da vida. Em contrapartida, negligenciar essas contribuições é negligenciar a diversidade do agir de Deus em seu próprio corpo, desprezando dons que foram concedidos para a edificação mútua.

Em Efésios, o apóstolo destaca que essa unidade é fruto da reconciliação realizada por Cristo: \textit{``Ele é a nossa paz, o qual de ambos fez um e quebrou a parede de separação que estava no meio''} (\gls{ef} 2.14, \gls{naa}). João Calvino comenta que Cristo \textit{``unindo judeus e gentios em um só corpo, removeu a separação estabelecida pela lei cerimonial''} \cite{calvinoEf2}, ressaltando que a comunhão dos santos é uma realidade concreta, e não meramente um símbolo ou conceito distante. Essa realidade demanda a remoção prática de todo tipo de preconceito, bloqueio, resistência ou barreira.

É importante reconhecer que, assim como Cristo derrubou a parede que separava judeus e gentios, também somos chamados a derrubar as barreiras que nos impedem de exercer a comunhão plena no corpo de Cristo. Um bom teólogo, ou qualquer membro da igreja, precisa estar atento às próprias limitações e remover as divisões internas que desconectam a prática teológica e pastoral da vida real da comunidade. Quando se despreza a contribuição de irmãos que possuem dons e saberes que nem sempre são estritamente teológicos, ergue-se novamente, de maneira sutil, a parede que Cristo já derrubou. A comunhão dos santos exige diálogo, humildade e disposição para reconhecer que a edificação do corpo depende da contribuição de todos, sob a direção daquele que é a nossa paz.

A prática da comunhão também se manifesta no modo de viver dos crentes. Paulo dá graças pela \textit{``participação no evangelho desde o primeiro dia até agora''} (\gls{fp} 1.5, \gls{naa}), mostrando que a comunhão inclui a cooperação ativa na missão de Deus. Aos colossenses, ele instrui a vivência comunitária marcada por compaixão, humildade e gratidão: \textit{``revesti-vos, pois, como eleitos de Deus, santos e amados, de ternos afetos de misericórdia, de bondade, de humildade, de mansidão, de longanimidade''} (\gls{cl} 3.12, \gls{naa}), culminando na exortação para que \textit{``a paz de Cristo seja o árbitro no coração de vocês''} (\gls{cl} 3.15, \gls{naa}).

O autor de Hebreus reforça a importância da comunhão para a perseverança na fé: \textit{``consideremo-nos também uns aos outros, para nos estimularmos ao amor e às boas obras, não abandonando a nossa congregação''} (\gls{hb} 10.24-25, \gls{naa}). De modo semelhante, Pedro exorta: \textit{``sejam todos de igual ânimo, compadecidos, fraternalmente amigos, misericordiosos, humildes''} (\gls{1pe} 3.8, \gls{naa}). João, em sua primeira carta, destaca que a comunhão com Deus e com os irmãos é marca essencial da vida cristã: \textit{``o que vimos e ouvimos anunciamos também a vocês, para que vocês tenham comunhão conosco; e a nossa comunhão é com o Pai e com seu Filho, Jesus Cristo''} (\gls{1jo} 1.3, \gls{naa}).

Essa comunhão, ainda imperfeita neste mundo, será plenamente consumada na eternidade. Em Apocalipse, Cristo assegura à sua igreja: \textit{``Não tenha medo. Eu sou o Primeiro e o Último. Eu sou aquele que vive; estive morto, mas eis que estou vivo pelos séculos dos séculos''} (\gls{ap} 1.17-18, \gls{naa}). Os que estão em Cristo, mesmo enfrentando a morte, permanecem vivos n'Ele e participarão da comunhão perfeita na nova criação, onde não haverá mais separação, dor ou pecado. Assim, a comunhão, que hoje é vivida de maneira imperfeita, já é um testemunho antecipado da unidade perfeita que será plenamente revelada na eternidade.

\section{Na Igreja Primitiva}

A comunhão dos santos na igreja primitiva manifestou-se de maneira visível e concreta através da partilha e solidariedade entre os crentes. Os primeiros cristãos viam a necessidade material e espiritual do próximo como sua própria responsabilidade, praticando a comunhão não apenas em palavras, mas em ações.

Essa vida comunitária levou à concepção da Igreja como mais do que uma simples reunião de crentes. Desde os dias de Tertuliano, a Igreja passou a ser reconhecida como \textit{mater fidelium}\footnote{Tradução: mãe dos crentes}, não como fruto da iniciativa humana, mas como obra do próprio Cristo:
\begin{citacao}
Desde os dias de Tertuliano, todos os cristãos chamavam a igreja não apenas de comunidade reunida (\textit{coetus}), mas também de mãe dos crentes (\textit{mater fidelium}). [...] A instituição da igreja, pelo menos de acordo com a confissão reformada, absolutamente não é um produto da comunidade de fé, mas uma obra do próprio Cristo. \cite[335]{bavinck2012}.
\end{citacao}

Com esta citação de Bavink, vemos que a comunhão dos santos não é fruto da criação humana mas é \textit{``uma obra do próprio Cristo''}.

A consciência da comunhão e da unidade também conduziu à necessidade de ordenamento doutrinário e ético. Agostinho aconselhava a correção mútua, com espírito de amor e paciência, preservando a comunhão mesmo diante das imperfeições humanas:
\begin{citacao}
Agostinho dá este conselho: ``que corrijam compassivamente o que podem; o que não podem, tolerem pacientemente, e com amor, deplorem e lamentem, até que Deus ou emende e corrija, ou, na colheita, arranque as cizânias e joeire as palhas.'' \cite[v. 4, p. 102]{calvino2022}
\end{citacao}

Dessa forma, a comunhão dos santos na igreja primitiva não era apenas uma experiência espiritual, mas envolvia responsabilidade mútua, identidade comunitária e compromisso ético diante de Deus e dos irmãos.

\section{Na Reforma Protestante}

\subsection{Comunhão como sustentação da igreja visível}
\textit{``A comunhão dos santos [...] exprime excelentemente a natureza da Igreja [...] que todos e quaisquer benefícios que Deus lhes confira, entre si, mutuamente, compartilhem''} \cite[v. 4, p. 75-76]{calvino2022}.

\begin{citacao}
``Por esta razão cremos na Igreja, que estejamos seguramente persuadidos de que somos seus membros. [...] nossa salvação se apoia em suportes seguros e sólidos, de sorte que, ainda quando seja abalada toda a máquina do orbe, ela própria não se mova e tombe por terra: primeiro, ela se sustém com a divina eleição [...]; então, de certo modo associada com a firmeza de Cristo, que não mais permitirá que seus fiéis sejam de si alijados [...]'' \cite[v. 4, p. 75-76]{calvino2022}
\end{citacao}

\begin{citacao}
``O abandono da Igreja é negação de Deus e de Cristo [...] Porque, enquanto nos esforçamos, quanto está em nós, por fomentar a ruína da verdade de Deus, somos dignos de que ele dardeje seus raios com todo o ímpeto de sua ira, a fim de fazer-nos em pedaços.'' \cite[v. 4, p. 92]{calvino2022}
\end{citacao}

\textit{``não renunciemos à comunhão da Igreja, nem perturbemos nela a paz e a disciplina devidamente exercitada''} \cite[v. 4, p. 95]{calvino2022}

\subsection{Nos símbolos de fé}

\subsubsection{A Confissão de Fé de Westminster} 
\begin{citacao}
Todos os santos que pelo seu Espírito e pela fé estão unidos a Jesus Cristo, seu Cabeça, têm com Ele comunhão nas suas graças, nos seus sofrimentos, na sua morte, na sua ressurreição e na sua glória, e, estando unidos uns aos outros no amor, participam dos mesmos dons e graças e estão obrigados ao cumprimento dos deveres públicos e particulares que contribuem para o seu mútuo proveito, tanto no homem interior como no exterior. \cite{cfw}.
\end{citacao}

\subsubsection{Catecismo Maior} 
\begin{citacao}
P. 63. O que é a comunhão dos santos?\\
R. A comunhão dos santos é aquela comunhão que os crentes têm com Cristo, e uns com os outros, na graça, sofrimentos, morte, ressurreição e glória dele; e, sendo assim, são obrigados a amar uns aos outros como membros de um mesmo corpo e a exercer comunhão uns com os outros em coisas espirituais e temporais segundo suas possibilidades e necessidades. \cite{catecismoMaior}.
\end{citacao}

\subsubsection{Catecismo de Heidelberg}
\begin{citacao}
P. 55. Como você entende as palavras: ``a comunhão dos santos''?\\
R. Primeiro: entendo que todos os crentes, juntos e cada um por si, têm, como membros, comunhão com Cristo, o Senhor, e todos os seus ricos dons. Segundo: que todos devem sentir-se obrigados a usar seus dons com vontade e alegria para o bem dos outros membros \cite{heidelberg}
\end{citacao}

\subsubsection{Segunda Confissão Helvética} 
\textit{``A Igreja é chamada de comunhão dos santos, porque, reunidos na verdadeira fé, todos os crentes participam de todas as bênçãos espirituais que Deus outorga à sua Igreja.''} \cite{helvetica}

\subsection{Contribuições de Louis Berkhof}

\subsubsection{Igreja como corpo espiritual e comunidade dos santos}
\begin{citacao}
    a igreja da qual a Bíblia diz coisas tão gloriosas não é a igreja considerada como instituição externa, mas a igreja como corpo espiritual de Jesus Cristo, que é essencialmente invisível no presente, [...] e esteja destinada a ter uma perfeita encarnação visível no fim dos séculos. \cite[p. 644]{berkhof2012}.
\end{citacao}

\subsubsection{Igreja visível e invisível como uma só essência}
\begin{citacao}
Estas não são duas igrejas, mas uma somente e, portanto, têm apenas uma única essência. [...] A igreja visível é a igreja como o homem a vê [...] julgados como sendo a comunidade dos santos. \cite[p. 642]{berkhof2012}.
\end{citacao}

\subsection{Contribuições de Herman Bavinck}

\subsubsection{Igreja como organismo e instituição}

\subsubsection{Diversidade de dons e mutualidade}
\begin{citacao}
Toda igreja local [...] é uma comunicação de santos na qual todos sofrem e se alegram uns com os outros e usam seus dons especiais ``pronta e agradavelmente para o serviço e o enriquecimento dos outros membros''. \cite[p. 380]{bavinck2012}.
\end{citacao}

\subsubsection{A comunhão como graça presente e escatológica}

\section{Desdobramentos éticos contemporâneos}

\subsection{Comunhão real: confissão, escuta, presença}
\begin{citacao}
O primeiro serviço que alguém deve ao outro na comunidade é ouvi-lo. Assim como o amor a Deus começa quando ouvimos a sua Palavra, assim também o amor ao irmão começa quando aprendemos a escutá-lo. [...] Esquecem que ouvir pode ser um serviço maior do que falar. \cite[pp. 75--76]{bonhoeffer1997}.
\end{citacao}

\begin{citacao}O pecado oculto separava-o da comunhão, desmentia toda a comunhão aparente; o pecado professado ajudou-o a encontrar a verdadeira comunhão com os irmãos em Jesus Cristo.'' \cite[p. 80]{bonhoeffer1997}.
\end{citacao}

\subsection{A comunhão como responsabilidade mútua}

\subsection{Relevância para a era digital}

\subsection{Síntese das compreensões ao longo da história}

\subsection{A Comunhão dos Santos como realidade}

\begin{citacao}
\foreignlanguage{english}{Essentially the church can be understood only as a divine act, that is, in the utterance of faith; only upon this basis can it be understood as an 'experience'; only faith comprehends the church as a community established by God.}\footnote{Tradução: ``Essencialmente, a igreja só pode ser compreendida como um ato divino, ou seja, na expressão da fé; somente sobre essa base ela pode ser entendida como 'experiência'; só a fé compreende a igreja como uma comunidade estabelecida por Deus.''} \cite[p. 195]{bonhoeffer1963}
\end{citacao}

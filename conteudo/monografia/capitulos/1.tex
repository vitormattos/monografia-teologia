\newpage
\section{A comunhão dos santos de Gênesis a teologia contemporânea}

\subsection{Definições}
\subsubsection*{Contextualização do tema e sua relevância contemporânea}
\subsubsection*{A tensão entre comunhão espiritual e responsabilidade institucional}
\subsubsection*{Propósito do capítulo}
Delinear como a comunhão dos santos foi compreendida e vivida ao longo da história da revelação e da teologia cristã.

\subsection{Perspectivas bíblicas sobre a comunhão dos santos}
\subsubsection{A comunhão no contexto da aliança e da vida do povo de Deus}
\begin{itemize}
  \item Criação, aliança e comunidade
  \item Comunhão como expressão de unidade e justiça
\end{itemize}

\subsubsection{A comunhão como expressão da vida em Cristo e no corpo}
\begin{itemize}
  \item Unidade na fé, partilha de dons e vida comunitária
  \item A comunhão como sinal escatológico
\end{itemize}

\subsection{A comunhão dos santos na igreja primitiva}
\subsubsection*{Tertuliano e a expressão \textit{mater fidelium}}
\subsubsection*{A formulação do credo apostólico}
\subsubsection*{A compreensão de comunhão como partilha e solidariedade entre os santos}

\subsection{A comunhão dos santos na reforma protestante}
\subsubsection*{A Confissão de Fé de Westminster -- capítulo XXVI}
\subsubsection*{Catecismo Maior -- pergunta 63}
\subsubsection*{João Calvino: comunhão como partilha dos dons espirituais e sustentação da igreja visível}
\subsubsection*{A importância da disciplina e da permanência na igreja visível}

\subsection{A comunhão dos santos na teologia reformada pós-reforma}
\subsubsection{Contribuições de Louis Berkhof}
\begin{itemize}
  \item Igreja como corpo espiritual e comunidade dos santos
  \item Igreja visível e invisível como uma só essência
\end{itemize}

\subsubsection{Contribuições de Herman Bavinck}
\begin{itemize}
  \item Igreja como organismo e instituição
  \item Diversidade de dons e mutualidade
  \item A comunhão como graça presente e escatológica
\end{itemize}

\subsection{Desdobramentos éticos contemporâneos}
\subsubsection*{Dietrich Bonhoeffer: comunhão real, confissão, escuta, presença}
\subsubsection*{A comunhão como responsabilidade mútua em tempos de crise}
\subsubsection*{Relevância para a era digital: comunhão ameaçada por exposição indevida de dados e práticas institucionais negligentes}
\subsubsection*{Síntese das compreensões ao longo da história}
\subsubsection*{A comunhão dos santos como realidade espiritual que se manifesta visivelmente e requer zelo ético, tecnológico e pastoral}

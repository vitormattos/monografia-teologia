\newpage
\section{A Comunhão dos Santos de Gênesis a teologia contemporânea}

\subsection{Perspectivas bíblicas}

\subsubsection{No Antigo Testamento}
A comunhão dos santos é uma realidade que se desenvolve desde a criação, onde o ser humano é feito à imagem de Deus com vocação relacional. Já no Antigo Testamento, a comunhão não é apenas relacional, mas institucional, pois Deus forma um povo que vive unido em leis e propósitos. A unidade comunitária era esperada em práticas de justiça, amor e culto. Os profetas anunciaram uma comunhão futura ampliada, abrangendo todos os povos.

\begin{itemize}
\item A imagem de Deus no ser humano fundamenta a relacionalidade (Gênesis 1:27)
\item ``Não é bom que o homem esteja só'' — a comunhão é necessária à condição humana (Gênesis 2:18)
\item A aliança forma um povo santo, unido por leis de justiça e amor (Êxodo 19:5-6; Levítico 19:18)
\item A comunhão expressa-se como unidade prática (Salmo 133)
\item O relacionamento entre os fiéis fortalece a perseverança e ajuda mútua (Eclesiastes 4:9-12)
\item A visão profética aponta para a comunhão futura de todos os povos no monte do Senhor (Isaías 2:2-4)
\item A promessa de uma nova aliança fundamenta uma comunhão mais profunda (Jeremias 31:33-34)
\end{itemize}

\subsubsection{No Novo Testamento}

\begin{itemize}
\item A igreja primitiva perseverava na comunhão (Atos 2:42)
\item ``Da multidão dos que criam, era um o coração e a alma'' (Atos 4:32)
\item Paulo descreve a comunhão como partilha no corpo e sangue de Cristo (1 Coríntios 10:16-17)
\item Unidade no Espírito, diversidade de membros, todos necessários (1 Coríntios 12:12-27)
\item A unidade e edificação da igreja como expressão da comunhão (Efésios 2:14-22; 4:1-6)
\item Participação ativa no evangelho como comunhão (Filipenses 1:5-7)
\item Vivência da comunhão com gratidão, ensino e amor (Colossenses 3:12-17)
\item A comunhão é vivida como edificação mútua e perseverança na fé (Hebreus 10:24-25)
\item A harmonia cristã como forma de comunhão (1 Pedro 3:8)
\item Comunhão com Deus e com os irmãos como marca da vida cristã (1 João 1:3-7)
\item A comunhão é escatológica: os que estão em Cristo vivem mesmo que morram (Apocalipse 1:17-18)
\end{itemize}

\subsection{Na igreja primitiva}

\subsubsection{Partilha e solidariedade entre os santos}
A multidão dos que criam era um só coração e uma só alma (Atos 4:32). Havia partilha de bens, oração e ensino. A comunhão dos santos era prática e diária.

\subsubsection{Tertuliano e a expressão \textit{mater fidelium}}
Desde Tertuliano, a igreja é reconhecida como \textit{mater fidelium}, mãe dos crentes \cite{bavinck2012}. Essa formulação patrística foi resgatada pela teologia reformada como fundamento para o papel visível, formativo e protetor da igreja na comunhão cristã.

\subsubsection{A formulação do credo apostólico}
A inclusão da ``comunhão dos santos'' no credo apostólico expressa a consciência histórica da igreja como corpo unido pela fé em Cristo e pela partilha da vida cristã.

\subsection{Na Reforma Protestante}

\subsubsection{Comunhão como sustentação da igreja visível}
Calvino define a comunhão dos santos como ``que todos e quaisquer benefícios que Deus lhes confira, entre si, mutuamente, compartilhem'' \cite{calvino2022}.

\subsubsection{A importância da igreja visível}
A igreja visível é chamada por Calvino de necessária: ``não outro nos é o ingresso à vida, a não ser que ela nos conceba no ventre [...] nos nutra [...] e nos governe'' \cite{calvino2022}.

\subsubsection{Nos símbolos de fé}

\paragraph{A Confissão de Fé de Westminster} 
Todos os santos têm comunhão com Cristo e uns com os outros; compartilham dons e graças e estão obrigados a deveres públicos e particulares \cite{cfw}.

\paragraph{Catecismo Maior} 
Pergunta 63: a comunhão dos santos é aquela que os crentes têm com Cristo e entre si, sendo obrigados a amar e a compartilhar espiritualmente e materialmente \cite{catecismoMaior}.

\paragraph{Catecismo de Heidelberg} 
Na pergunta 55, o Catecismo afirma: “Creio que todos e cada um, como membros de Cristo, têm comunhão entre si e participam de todos os seus dons e riquezas. Cada um deve considerar como dever seu usar com alegria os seus dons para o bem e salvação dos outros membros.” \cite{heidelberg}

\subsubsection{Contribuições de Louis Berkhof}
\begin{itemize}
  \item Igreja como corpo espiritual e comunidade dos santos
Berkhof entende que a igreja, conforme a concepção reformada, é o corpo espiritual de Cristo, uma comunhão dos fiéis reunidos pelo Espírito Santo \cite{berkhof2012}.
  \item Igreja visível e invisível como uma só essência
Para Berkhof, a igreja invisível — composta por todos os eleitos — e a igreja visível — formada por todos os que professam a fé — são essencialmente uma só igreja \cite{berkhof2012}.
\end{itemize}

\subsubsection{Contribuições de Herman Bavinck}
\begin{itemize}
  \item Igreja como organismo e instituição
Bavinck afirma que a igreja é tanto uma realidade espiritual quanto uma estrutura visível \cite{bavinck2012}.
  \item Diversidade de dons e mutualidade
O Espírito Santo concede diferentes dons aos membros da igreja para que se edifiquem mutuamente \cite{bavinck2012}.
  \item A comunhão como graça presente e escatológica
A comunhão dos santos é, para Bavinck, tanto uma realidade presente quanto uma antecipação da comunhão final na nova criação \cite{bavinck2012}.
\end{itemize}

\subsection{Desdobramentos éticos contemporâneos}

\subsubsection{Comunhão real: confissão, escuta, presença}
Bonhoeffer afirma que “o primeiro serviço que alguém deve ao outro na comunidade é ouvi-lo” \cite{bonhoeffer2009}.

\subsubsection{A comunhão como responsabilidade mútua}
A comunhão se fragiliza quando há omissão, exposição indevida ou quebra de confiança. A ética comunitária exige corresponsabilidade.

\subsubsection{Relevância para a era digital}
A comunhão dos santos pode ser comprometida quando dados, imagens ou histórias dos irmãos são tratados de forma negligente pela igreja institucional.

\subsubsection{Síntese das compreensões ao longo da história}
A doutrina da comunhão dos santos perpassa o Antigo e o Novo Testamento, a patrística, a Reforma e a teologia sistemática, sendo central à identidade da Igreja.

\subsubsection{A Comunhão dos Santos como realidade}
A comunhão dos santos é graça concedida por Deus, vivida em Cristo, sustentada pelo Espírito e encarnada na vida da igreja visível. Exige zelo pastoral, ético e comunitário, inclusive nos meios digitais.

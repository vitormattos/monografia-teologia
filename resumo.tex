\documentclass[12pt,a4paper]{article}
\usepackage[brazil]{babel}
\usepackage[top=3cm,bottom=2cm,left=3cm,right=2cm]{geometry}
\usepackage{indentfirst}
\usepackage{ragged2e}
\usepackage{fontspec}
\setmainfont{DejaVu Sans}
\renewcommand{\baselinestretch}{1.5}

\begin{document}

\newcommand{\instituicao}{Seminário Teológico Presbiteriano rev. Ashbel Green Simonton}
\newcommand{\tema}{A comunhão cristã frente aos dilemas da ética digital}
\newcommand{\autor}{Vitor Mattos de Souza}
\newpage
\thispagestyle{empty}

\begin{center}
    \textbf{\MakeUppercase{\instituicao}}
    \vskip 6cm
    \textbf{\MakeUppercase{\autor}}
    \vskip 6cm

    \textbf{\MakeUppercase{\tema}:} \subtema

    \vfill
    \localPublicacao \\
    \anoPublicacao
\end{center}
% Contracapa
\newpage
\pagestyle{empty}
\begin{center}
    \large \textbf{\MakeUppercase{\autor}}
    \vskip 6cm
    \Huge \textbf{\large \MakeUppercase{\tema}}
    \vskip 5cm
\end{center}
    \hfill{\vbox{\hsize=8.5cm\noindent\strut
    Resumo do Projeto de Pesquisa apresentado ao \instituicao, em cumprimento a exigências da disciplina \textbf{Monografia 1}, do curso de Bacharel em Teologia.}\\
    \strut}
    \vskip 3.0cm
\begin{center}
    \vfill
    \large
    Rio de Janeiro \\
    2025
\end{center}

\begin{center}
    \large \textbf{RESUMO}
\end{center}

\vspace{1cm}

\begin{justify}
    Esta pesquisa investiga os desafios e responsabilidades das igrejas no uso da tecnologia, com ênfase na proteção de dados sensíveis e na privacidade no contexto da comunhão cristã. Partindo de princípios bíblicos sobre ética e dignidade humana, o estudo explora como a fidelidade às Escrituras pode orientar uma gestão ética e segura das informações no ambiente eclesiástico. Aborda os riscos do uso inadequado da tecnologia, como a negligência na proteção de dados e seus impactos na vida comunitária. Analisa-se a relação entre ética, tecnologia e comunhão cristã, considerando as implicações legais e pastorais na administração de informações sensíveis, apoiando-se na Confissão de Fé de Westminster ao tratar da comunhão cristã e das responsabilidades cristãs de refletir-se os ensinos bíblicos. Por fim, propõe-se diretrizes para uma governança digital ética, reforçando o compromisso das igrejas com a privacidade e a promoção de uma comunhão segura e responsável.
\end{justify}

\vspace{0.5cm}

\noindent
\textbf{Palavras-chave:} Tecnologia. Privacidade. Igreja. Ética cristã. Segurança de dados.

\end{document}

\documentclass[12pt,a4paper]{article}
\usepackage[brazil]{babel}
\usepackage[autostyle=true]{csquotes}
\usepackage{graphicx}
\usepackage[colorlinks=true, linkcolor=black, citecolor=black, urlcolor=blue, pdfborder={0 0 0}]{hyperref}
\usepackage[alf]{abntex2cite}
\usepackage[top=3cm,bottom=2cm,left=3cm,right=2cm]{geometry}
\usepackage{indentfirst}
\usepackage{tabularx}
\usepackage{fontspec}
\setmainfont{DejaVu Sans}

\newcommand{\instituicao}{Seminário Teológico Presbiteriano rev. Ashbel Green Simonton}
\newcommand{\tema}{A comunhão cristã frente aos dilemas da ética digital}
\newcommand{\subtema}{Uma abordagem bíblica sobre a gestão de dados sensíveis na igreja}
\newcommand{\autor}{Vitor Mattos de Souza}
\newcommand{\anoPublicacao}{2025}
\newcommand{\localPublicacao}{Rio de Janeiro}
\newcommand{\palavrasChave}{igreja, ética cristã, tecnologia, privacidade, segurança de dados.}
\newcommand{\keywords}{church, Christian ethics, technology, privacy, data security.}


\begin{document}

\hypersetup{pageanchor=false}  % Evita âncoras duplicadas

\newpage
\thispagestyle{empty}

\begin{center}
    \textbf{\MakeUppercase{\imprimirinstituicao}}
    \vskip 6cm
    \textbf{\MakeUppercase{\imprimirautor}}
    \vskip 6cm

    \textbf{\MakeUppercase{\imprimirtitulo}:} \subtema

    \vfill
    \imprimirlocal \\
    \imprimirdata
\end{center}

% Contracapa
\newpage
\pagestyle{empty}
\begin{center}
    \large \textbf{\MakeUppercase{\autor}}
    \vskip 6cm
    \Huge \textbf{\large \MakeUppercase{\tema}}
    \vskip 6cm
\end{center}
    \hfill{\vbox{\hsize=8.5cm\noindent\strut
    Projeto de Pesquisa apresentado ao \instituicao, em cumprimento a exigências da disciplina \textbf{Monografia 1}, do curso de Bacharel em Teologia.}\\
    \strut}
    \vskip 3.0cm
\begin{center}
    \vfill
    \large
    Rio de Janeiro \\
    2025
\end{center}

\include{conteudo/licença}

% DESENVOLVIMENTO
\newpage
\pagestyle{plain}
\pagenumbering{arabic}
\renewcommand{\baselinestretch}{1.5}
\normalsize
\noindent
\textbf{Tema:} \tema

\noindent
\textbf{Sinopse:} Este estudo reflete sobre a interação entre tecnologia e a vida comunitária cristã e como a proteção de dados sensíveis pode ser um reflexo da fidelidade ao chamado divino para preservar a dignidade humana. A pesquisa propõe práticas éticas que promovam tanto a privacidade quanto a comunhão.

\noindent
\textbf{Problema:} Como a fidelidade aos princípios bíblicos pode orientar as igrejas na gestão ética de dados sensíveis no contexto de uma sociedade altamente tecnológica?

\noindent
\textbf{Formulação de hipóteses:}
Decorrem do tema as seguintes hipóteses:
\begin{enumerate}
    \item O uso inadequado da tecnologia nas igrejas pode ser uma forma contemporânea de negligência, comprometendo a fidelidade aos princípios bíblicos.
    \item A responsabilidade ética no uso da tecnologia pelas igrejas reflete a obediência ao chamado bíblico de preservar a dignidade humana e demonstra o compromisso pastoral com o cuidado mútuo.
    \item A implementação de políticas claras de proteção de dados sensíveis, fundamentadas em princípios bíblicos, fortalece o testemunho cristão e promove uma gestão ética no contexto de uma sociedade tecnológica.
\end{enumerate}

\noindent
\textbf{Justificativa:} A escolha do tema "\tema" se fundamenta na relevância de refletir sobre os desafios éticos e administrativos que as igrejas enfrentam em um mundo cada vez mais tecnológico. Considerando a crescente utilização da tecnologia pelas igrejas, seja em atividades administrativas, na promoção da comunhão cristã ou na proclamação do evangelho, é necessário utilizar-se de princípios bíblicos para uso consciente, ético e legal de dados no contexto das igrejas e suas atividades.

O uso inadequado da tecnologia compromete valores essenciais da fé cristã e pode levar ao descumprimento de leis, impactando a vida cristã de líderes e membros. Isso reforça a necessidade de uma gestão ética e responsável dos dados sensíveis. A segurança, o respeito às leis e a implementação de políticas claras precisam ser debatidos nas igrejas por suas lideranças para fortalecer o testemunho cristão e assegurar que a igreja local reflita princípios bíblicos e cristãos em todos os aspectos da vida comunitária.

\noindent
\textbf{Objetivo geral:} Investigar os desafios e responsabilidades das igrejas no uso da tecnologia, considerando princípios bíblicos e a necessidade de proteger dados sensíveis de forma ética e legal.

\noindent
\textbf{Objetivo específico:} Identificar os principais riscos e desafios enfrentados pelas igrejas no uso da tecnologia, especialmente no que se refere à gestão de dados sensíveis, à luz de princípios éticos e bíblicos.

\noindent
\textbf{Fundamentação teórica:}

A fundamentação deste trabalho está ancorada nos princípios bíblicos sobre ética e dignidade humana, conforme ensinados pelas Escrituras e aplicados aos desafios éticos e administrativos na gestão de dados sensíveis pelas igrejas.

A \textit{Confissão de Fé de Westminster (CFW)} estabelece, na Seção VI do Capítulo I, que todos os aspectos da vida devem ser regidos pelos princípios bíblicos, incluindo questões éticas e administrativas. Assim, o uso da tecnologia nas igrejas deve refletir valores cristãos, promovendo a glória de Deus e o cuidado com o próximo \cite{cfw}. O Capítulo XXIII da CFW, "\textit{Do Magistrado Civil}", reforça o dever cristão de obedecer às leis civis, destacando que a gestão ética de tecnologia nas igrejas demonstra fidelidade ao testemunho cristão e protege a dignidade dos membros.

Conforme Calvino afirma, \textit{"amar ao próximo é reconhecidamente um mandamento, não um conselho evangélico aleatório"} \cite[p. 453]{calvino2022}. Proteger dados sensíveis no contexto eclesiástico é uma expressão prática desse amor cristão, e sua negligência constitui uma falha ética, comprometendo o mandamento do amor.

O avanço tecnológico abriu a possibilidade de propagação do evangelho em meios digitais, mas também trouxe desafios relacionados à privacidade e à proteção de dados sensíveis. Como alertou Francis Schaeffer:

\begin{quote}
\textit{"A possibilidade de armazenamento de informações... faz com que literalmente não reste mais espaço para alguém se esconder ou ter qualquer privacidade"} \cite[p. 180]{schaeffer2002}.
\end{quote}

Essa realidade exige que as igrejas adotem práticas éticas e responsáveis no uso da tecnologia para evitar comprometer valores cristãos e prejudicar seu testemunho público.

Na aplicação prática dos princípios bíblicos, a gestão de dados deve ser direcionada pela responsabilidade de proteger a comunhão e a dignidade humanas. Em \textit{Igreja Centrada}, Timothy Keller reflete sobre o compromisso da igreja com a justiça e sua responsabilidade com a comunidade \cite[p. 209]{keller2014}. John Stott reforça a importância da ética cristã, afirmando que \textit{"sem ética, não há arrependimento genuíno"} \cite[p. 51]{stott2008}. Isso evidencia que a gestão ética de dados sensíveis vai além das exigências legais, sendo uma expressão prática do cuidado cristão.

O Capítulo XXVI da CFW, "\textit{Da Comunhão dos Santos}", conecta a gestão de dados à unidade da igreja. Proteger informações sensíveis promove transparência, edificação e confiança mútua, fortalecendo a comunhão e o impacto social da igreja \cite{cfw}.

A gestão ética de tecnologia e dados sensíveis pelas igrejas não é apenas uma questão administrativa ou da área tecnológica e não teológica, mas uma expressão prática de princípios bíblicos de amor, justiça e responsabilidade, refletindo o compromisso da igreja com sua missão e integridade no mundo contemporâneo.

\noindent
\textbf{Estrutura da pesquisa (sumário preliminar):}

\noindent
Capítulo 1: Princípios bíblicos e éticos para a dignidade humana e a proteção de dados sensíveis

\noindent
Capítulo 2: Desafios éticos e administrativos na gestão de dados pelas igrejas

\noindent
Capítulo 3: Boas práticas de segurança de dados no contexto eclesiástico

\noindent
\textbf{Metodologia:}

A Metodologia a ser empregada no trabalho é a pesquisa bibliográfica, e será desenvolvida a partir de materiais publicadas em livros, artigos, dissertações e teses.
\begin{enumerate}
    \item Pesquisa bibliográfica que aborde o tema proposto dentro do meio cristão e na área de segurança da informação
    \item Contextualizar a bibliografia com princípios bíblicos e a CFW
    \item Interagir as hipóteses com o contexto bíblico e bibliográfico
    \item Elaboração de propostas para solução do problema
\end{enumerate}

\noindent
\textbf{Cronograma:}

\begin{table}[h]
    \centering
    \renewcommand{\arraystretch}{1.3}
    \begin{tabularx}{\textwidth}{|c|X|}
        \hline
        \textbf{Data} & \textbf{Etapa do Cronograma} \\
        \hline
        Janeiro  & Etapa de orientação do trabalho para definição de conteúdo e forma \\
        \hline
        6 a 7 de fevereiro  & Apresentação projeto de monografia \\
        \hline
        15 de março  & Entrega do 1º capítulo \\
        \hline
        15 de abril  & Entrega do 2º capítulo \\
        \hline
        15 de maio  & Entrega do 3º capítulo \\
        \hline
        15 de junho  & Entrega das Considerações Finais e Introdução \\
        \hline
        15 de julho  & Entrega completa ao orientador para revisão parcial \\
        \hline
        15 de agosto  & Entrega completa ao orientador para revisão final \\
        \hline
        15 de setembro  & Entrega completa ao orientador \\
        \hline
        30 de setembro  & Entrega ao seminário \\
        \hline
        1 a 15 de outubro  & Defesa em Banca \\
        \hline
        28 de novembro  & Entrega do trabalho após as correções da Banca Examinadora \\
        \hline
    \end{tabularx}
    \caption{Cronograma do trabalho}
    \label{tab:cronograma}
\end{table}

\bibliography{referencias/bibliografia}

\end{document}

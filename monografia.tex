\documentclass[12pt,a4paper]{article}
\usepackage[brazil]{babel}
\usepackage[autostyle=true]{csquotes}
\usepackage{graphicx}
\usepackage[colorlinks=true, linkcolor=black, citecolor=black, urlcolor=blue, pdfborder={0 0 0}]{hyperref}
\usepackage[alf]{abntex2cite}
\usepackage[top=3cm,bottom=2cm,left=3cm,right=2cm]{geometry}
\usepackage{indentfirst}
\usepackage{tabularx}
\usepackage{fontspec}
\usepackage{ragged2e}
\setmainfont{DejaVu Sans}

\begin{document}

\newcommand{\instituicao}{Seminário Teológico Presbiteriano rev. Ashbel Green Simonton}
\newcommand{\tema}{A comunhão cristã frente aos dilemas da ética digital}
\newcommand{\subtema}{Uma abordagem bíblica sobre a gestão de dados sensíveis na igreja}
\newcommand{\autor}{Vitor Mattos de Souza}
\newcommand{\anoPublicacao}{2025}
\newcommand{\localPublicacao}{Rio de Janeiro}
\newcommand{\palavrasChave}{igreja, ética cristã, tecnologia, privacidade, segurança de dados.}
\newcommand{\keywords}{church, Christian ethics, technology, privacy, data security.}


\hypersetup{pageanchor=false}  % Evita âncoras duplicadas

\newpage
\thispagestyle{empty}

\begin{center}
    \textbf{\MakeUppercase{\imprimirinstituicao}}
    \vskip 6cm
    \textbf{\MakeUppercase{\imprimirautor}}
    \vskip 6cm

    \textbf{\MakeUppercase{\imprimirtitulo}:} \subtema

    \vfill
    \imprimirlocal \\
    \imprimirdata
\end{center}

% Contracapa
\newpage
\pagestyle{empty}
\begin{center}
    \large \textbf{\MakeUppercase{\autor}}
    \vskip 6cm

    \textbf{\MakeUppercase{\tema}:}

    \vspace{0.5cm}
    \subtema

    \vskip 6cm
\end{center}
    \hfill{\vbox{\hsize=8.5cm\noindent\strut
    Monografia apresentada ao \instituicao, como requisito parcial para a conclusão do curso de Bacharelado em Teologia.\\
    \strut}
    \vskip 3.0cm
\begin{center}
    \vfill
    \large
    \localPublicacao \\
    \anoPublicacao
\end{center}

\include{conteudo/licença}

\newpage
\pagestyle{plain}
\pagenumbering{arabic}
\renewcommand{\baselinestretch}{1.5}
\normalsize

% Reesumo
\newpage
\begin{center}
    \large \textbf{\MakeUppercase{Resumo}}
\end{center}

\vspace{1cm}

\justifying
\input{conteudo/resumo/pt_BR.txt}

\vspace{0.5cm}

\noindent
\textbf{Palavras-chave:} \textit{\palavrasChave}

% Abstact
\newpage
\begin{center}
    \large \textbf{\MakeUppercase{Abstract}}
\end{center}

\vspace{1cm}

\justifying
\input{conteudo/resumo/en_US.txt}

\vspace{0.5cm}

\noindent
\textbf{Keywords:} \textit{\keywords}

% Conteúdo
\newpage
\noindent
Capítulo 1: Fundamentos bíblicos e históricos da comunhão cristã

\noindent
Capítulo 2: Riscos éticos da era digital que ameaçam a vivência da comunhão

\noindent
Capítulo 3: Propostas teológicas e éticas para preservar a comunhão em contextos digitais

\bibliography{referencias/bibliografia}

\end{document}

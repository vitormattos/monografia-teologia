\documentclass[12pt,a4paper]{article}
\usepackage[brazil]{babel}
\usepackage[autostyle=true]{csquotes}
\usepackage{graphicx}
\usepackage[colorlinks=true, linkcolor=black, citecolor=black, urlcolor=blue, pdfborder={0 0 0}]{hyperref}
\usepackage[alf]{abntex2cite}
\usepackage[top=3cm,bottom=2cm,left=3cm,right=2cm]{geometry}
\usepackage{indentfirst}
\usepackage{tabularx}
\usepackage{fontspec}
\setmainfont{DejaVu Sans}

\begin{document}

\instituicao{Seminário Teológico Presbiteriano rev. Ashbel Green Simonton}
\titulo{A comunhão dos santos frente aos dilemas da ética digital}
\newcommand{\subtema}{Uma abordagem bíblica sobre a gestão de dados sensíveis na igreja}
\autor{Vitor Mattos de Souza}
\data{2025}
\local{Rio de Janeiro}
\newcommand{\dataAprovacao}{\rule{1cm}{0.4pt} de \rule{3cm}{0.4pt} de \imprimirdata}
\newcommand{\palavrasChave}{igreja, ética cristã, tecnologia, privacidade, segurança de dados.}
\newcommand{\keywords}{church, Christian ethics, technology, privacy, data security.}
\orientador{Rev. André Monteiro}
\tipotrabalho{Monografia}
\preambulo{Trabalho monográfico apresentado ao \textbf{\imprimirinstituicao}, como parte das exigências para obtenção do título de Bacharel em Teologia, sob a orientação do \imprimirorientador.}
\newcommand{\notaApresentacaoProjeto}{Pré-projeto monográfico apresentado ao \textbf{\imprimirinstituicao}, como parte das exigências para obtenção do título de Bacharel em Teologia, sob a orientação do \imprimirorientador.}


\hypersetup{pageanchor=false}  % Evita âncoras duplicadas

\newpage
\thispagestyle{empty}

\begin{center}
    \textbf{\MakeUppercase{\instituicao}}
    \vskip 6cm
    \textbf{\MakeUppercase{\autor}}
    \vskip 6cm

    \textbf{\MakeUppercase{\tema}:} \subtema

    \vfill
    \localPublicacao \\
    \anoPublicacao
\end{center}

\newpage
\thispagestyle{empty}

\begin{center}
    \textbf{\MakeUppercase{\autor}}
    \vskip 5.5cm

    \textbf{\MakeUppercase{\tema}:} \subtema

    \vskip 5.5cm
\end{center}
    \hfill{\vbox{\hsize=8.5cm\noindent\strut\notaApresentacao\\
    \strut}
    \vskip 3.0cm
\begin{center}
    \vfill
    \localPublicacao \\
    \anoPublicacao
\end{center}

\include{conteudo/licença}

\newpage
\pagestyle{plain}
\pagenumbering{arabic}
\renewcommand{\baselinestretch}{1.5}
\normalsize

\noindent
Capítulo 1: Fundamentos bíblicos e históricos da comunhão cristã

\noindent
Capítulo 2: Riscos éticos da era digital que ameaçam a vivência da comunhão

\noindent
Capítulo 3: Propostas teológicas e éticas para preservar a comunhão em contextos digitais

\bibliography{referencias/bibliografia}

\end{document}
